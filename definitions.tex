% Clase del documento
\documentclass[a4paper,12pt,twoside,openright,titlepage]{book}

%
% Paquetes necesarios
%

\usepackage{eurosym} 			% Símbolo del euro
\usepackage[utf8]{inputenc} 	% Codificación UTF8
\usepackage[spanish,english]{babel} 	% Caracteres del español
\usepackage{listings} 		% Código, algoritmos, etc.
\usepackage{color} 			% Definición de colores
\usepackage[table,xcdraw]{xcolor} 		% Extensión del paquete color
\usepackage{anysize} 		% Márgenes
\usepackage{fancyhdr} 		% Cabecera y pie de página
\usepackage{quotchap} 		% Estilo título capítulos
\usepackage{algorithmic} 	% Algoritmos (expresarlos mejor)
\usepackage{titlesec} 		% Títulos de secciones
\usepackage{amsmath} 		% Fórmulas matemáticas
\usepackage{enumerate} 		% Enumeraciones
\usepackage{emptypage} 		% Páginas en blanco
\usepackage{float} 			% Separación entre cajas
\usepackage{graphicx} 		% Imágenes
\usepackage{array} 			% Mejora de las tablas
\usepackage{mdwmath} 		% Mejora de los símbolos matemáticos
\usepackage[caption=false,font=footnotesize]{subfig} 		% Separar figuras en subfiguras
\usepackage{pdfpages} 		% Incluir pdfs externos
\usepackage{fancybox} 		% Mejoras sobre las cajas
\usepackage{appendix} 		% Apéndices
\usepackage{bookmark} 		% Marcadores (para el pdf)
\usepackage{enumitem} 		% Estilo de enumeraciones
\usepackage{setspace} 		% Espacio entre líneas y párrafos
\usepackage[acronym]{glossaries} 		% Glosario/Acrónimos
\usepackage[T1]{fontenc} 	% Fuentes
\usepackage[sorting=none,natbib=true,backend=bibtex,bibencoding=ascii]{biblatex} 		% Bibliografía
\usepackage{csquotes} 		% Fix biblatex+babel warning
\usepackage{exmath}
\usepackage{MathUnicode}
\usepackage{imakeidx}
\usepackage{definitions}
\usepackage{breqn}
\usepackage{hyperref}

\usetikzlibrary{arrows}
\tikzset{every node/.append style={minimum size=0.5cm, draw,circle,font=\sffamily\Large\bfseries,inner sep=0.05cm}}%



%\PrerenderUnicode{ÁáÉéÍíÓóÚúÑñ} % Para que salgan las tildes y demás mierdas en el título.

% Enlaces
\hypersetup{hidelinks,pageanchor=true,colorlinks,citecolor=Fuchsia,urlcolor=black,linkcolor=Cerulean}

% Euro (€)
\DeclareUnicodeCharacter{20AC}{\euro}

% Inclusión de gráficos
\graphicspath{{./graphics/}}

% Extensiones de gráficos
\DeclareGraphicsExtensions{.pdf,.jpeg,.jpg,.png}

% Definiciones de colores (para hidelinks)
\definecolor{LightCyan}{rgb}{0,0,0}
\definecolor{Cerulean}{rgb}{0,0,0}
\definecolor{Fuchsia}{rgb}{0,0,0}

% Keywords (español e inglés)
\def\keywordsEn{\vspace{.5em}
{\textbf{\textit{Key words ---}}\,\relax%
}}
\def\endkeywordsEn{\par}

\def\keywordsEs{\vspace{.5em}
{\textbf{\textit{Palabras clave ---}}\,\relax%
}}
\def\endkeywordsEs{\par}


% Abstract (español e inglés)
\def\abstractEs{\vspace{.5em}
{\textbf{\textit{Resumen ---}}\,\relax%
}}
\def\endabstractEs{\par}

\def\abstractEn{\vspace{.5em}
{\textbf{\textit{Abstract ---}}\,\relax%
}}
\def\endabstractEn{\par}

% Estilo páginas de capítulos
\fancypagestyle{plain}{
\fancyhf{}
\fancyfoot[CO]{\footnotesize\emph{\workname}}
\fancyfoot[RO]{\thepage}
\renewcommand{\footrulewidth}{.6pt}
\renewcommand{\headrulewidth}{0.0pt}
}

% Estilo resto de páginas
\pagestyle{fancy}

% Estilo páginas impares
\fancyfoot[CO]{\footnotesize\emph{\workname}}
\fancyfoot[RO]{\thepage}
\rhead[]{\leftmark}

% Estilo páginas pares
\fancyfoot[CE]{\emph{\pieparcen}}
\fancyfoot[LE]{\thepage}
\fancyfoot[RE]{\pieparizq}
\lhead[\leftmark]{}

% Guía del pie de página
\renewcommand{\footrulewidth}{.6pt}

% Nombre de los bloques de código
\renewcommand{\lstlistingname}{Code}

% Estilo de los lstlistings
\lstset{
    frame=tb,
    breaklines=true,
    postbreak=\raisebox{0ex}[0ex][0ex]{\ensuremath{\color{gray}\hookrightarrow\space}}
}

% Definiciones de funciones para los títulos
\newlength\salto
\setlength{\salto}{3.5ex plus 1ex minus .2ex}
\newlength\resalto
\setlength{\resalto}{2.3ex plus.2ex}

% Estilo de los acrónimos

\pretolerance=2000
\tolerance=3000

% Texto índice de tablas
%\addto\captionsspanish{
%\def\tablename{Tabla}
%\def\listtablename{\'Indice de tablas}
%}


% Comando code (lstlisting sin cambio de página)
\lstnewenvironment{code}[1][]%
  { \noindent\minipage{0.935\linewidth}\medskip
    \vspace{5mm}
    \lstset{basicstyle=\ttfamily\footnotesize,#1}}
  {\endminipage}

%\newcommand{\implies}{\rightarrow}
%\newcommand{\impliedby}{\leftarrow}
%\newcommand{\dimplies}{\leftrightarrow}
\newcommand{\orcond}{\vee}
\newcommand{\andcond}{\wedge}
\newcommand{\true}{\top}
\newcommand{\false}{\perp}
\newcommand{\axioms}{\mathcal{A}}

% % % % From Alejandro Sánchez
	\newcommand{\figscale}{0.36}
	\newcommand{\funcscale}{0.34}
	\newcommand{\queuescale}{0.32}
	\newcommand{\leapscale}{0.42}
	\newcommand{\listscale}{0.30}
	\newcommand{\robotscale}{0.28}
	\newcommand{\longtablesize}{22em}
	\newcommand{\medtablesize}{25em}
	\newcommand{\smalltablesize}{6em}
	\newcommand{\intersize}{30em}
	\newcommand{\quadsize}{\quad}

\renewcommand{\pc}{pc}
\newcommand{\tbf}[1]{\textbf{#1}}



%%%%%%%%%%%%%%%%%%%%%%%%%%%%%%%%%%%%%%
%%                                  %%
%%  Basic terminology for theories  %%
%%                                  %%
%%%%%%%%%%%%%%%%%%%%%%%%%%%%%%%%%%%%%%

%% Signatures
\newcommand{\Sig}{\ensuremath{\Sigma}\xspace}
\newcommand{\SigInter}{\ensuremath{\Sig}-interpretation\xspace}
\newcommand{\SigStruct}{\ensuremath{\Sig}-structure\xspace}
\newcommand{\SigStructs}{\ensuremath{\Sig}-structures\xspace}
\newcommand{\SigTerm}{\ensuremath{\Sig}-term\xspace}
\newcommand{\SigFormula}{\ensuremath{\Sig}-formula\xspace}
\newcommand{\SigLiteral}{\ensuremath{\Sig}-literal\xspace}
\newcommand{\SigTheory}{\ensuremath{\Sig}-theory\xspace}
\newcommand{\SigFormulaSet}{\ensuremath{F_{\Sig}}\xspace}
\newcommand{\SigFormulaSetVar}[1]{\ensuremath{F_{\Sig}^{(#1)}}\xspace}

%% Interpretations
\newcommand{\Omg}{\ensuremath{\Omega}\xspace}
\newcommand{\OmgInter}{\ensuremath{\Omg}-interpretation\xspace}
\newcommand{\AiSig}{\ensuremath{\Ai^{\Sig}}\xspace}
\newcommand{\SigClass}{\ensuremath{\mathbf{A}}\xspace}

% Variable sets
\newcommand{\VarV}{\ensuremath{V}\xspace}
\newcommand{\VarU}{\ensuremath{U}\xspace}

%% Generic theories names
\newcommand{\Theo}{\ensuremath{T}\xspace}
\newcommand{\TheoInter}{\ensuremath{\Theo}-interpretation\xspace}
\newcommand{\TheoSat}{\ensuremath{\Theo}-satisfiable\xspace}


%% Symbols for theory combination
\newcommand{\sort}{\ensuremath{\mathit{sort}}\xspace}


%% Theory names
\newcommand{\addr}{\ensuremath{\mathsf{addr}}\xspace}
\newcommand{\elem}{\ensuremath{\mathsf{elem}}\xspace}
\newcommand{\tid}{\ensuremath{\mathsf{tid}}\xspace}
\newcommand{\thId}{\ensuremath{\mathsf{tid}}\xspace}
\newcommand{\cell}{\ensuremath{\mathsf{cell}}\xspace}
\newcommand{\cells}{\ensuremath{\mathsf{cell}}\xspace}
\newcommand{\cellsK}{\ensuremath{\mathsf{cell}_{\mathsf{K}}}\xspace}
\newcommand{\memory}{\ensuremath{\mathsf{mem}}\xspace}
\newcommand{\mem}{\ensuremath{\mathsf{mem}}\xspace}
\newcommand{\memoryK}{\ensuremath{\mathsf{mem}_{\mathsf{K}}}\xspace}
%\newcommand{\reach}{\ensuremath{\mathsf{reachability}}\xspace}
\newcommand{\reach}{\ensuremath{\mathsf{reach}}\xspace}
\newcommand{\reachK}{\ensuremath{\mathsf{reach}_{\mathsf{K}}}\xspace}
\providecommand*{\path}{}
\renewcommand{\path}{\ensuremath{\mathsf{path}}\xspace}
\newcommand{\bridge}{\ensuremath{\mathsf{bridge}}\xspace}
\newcommand{\bridgeK}{\ensuremath{\mathsf{bridge}_{\mathsf{K}}}\xspace}
\newcommand{\lists}{\ensuremath{\mathsf{Lists}}\xspace}
\newcommand{\cons}{\ensuremath{\mathsf{cons}}\xspace}
\newcommand{\listsBridge}{\ensuremath{\mathsf{ListsBridge}}\xspace}


%% Sorts
\newcommand{\sTID}{\textsf{tid}\xspace}
\newcommand{\sBool}{\ensuremath{\mathsf{boolean}}\xspace}
\newcommand{\sElem}{\ensuremath{\mathsf{elem}}\xspace}
\newcommand{\sAddr}{\ensuremath{\mathsf{addr}}\xspace}
\newcommand{\sTid}{\ensuremath{\mathsf{tid}}\xspace}
\newcommand{\sCell}{\ensuremath{\mathsf{cell}}\xspace}
\newcommand{\sCellK}{\ensuremath{\cellsK}\xspace}
\newcommand{\sMem}{\ensuremath{\mathsf{mem}}\xspace}
\newcommand{\sMemK}{\ensuremath{\memoryK}\xspace}
\newcommand{\sPath}{\ensuremath{\mathsf{path}}\xspace}
\newcommand{\sWhatever}{\ensuremath{\mathsf{\_}}\xspace}
\newcommand{\sSetWhatever}{\ensuremath{\mathsf{set}}\sWhatever\xspace}
\newcommand{\sSetAddr}{\ensuremath{\mathsf{setaddr}}\xspace}
\newcommand{\sSetElem}{\ensuremath{\mathsf{setelem}}\xspace}
\newcommand{\sSetTid}{\ensuremath{\mathsf{settid}}\xspace}
\newcommand{\sSet}{\ensuremath{\mathsf{setaddr}}\xspace}
\newcommand{\sSetE}{\ensuremath{\mathsf{setelem}}\xspace}
\newcommand{\sSetT}{\ensuremath{\mathsf{settid}}\xspace}
\newcommand{\sSeq}{\ensuremath{\mathsf{fseq}}\xspace}
\newcommand{\sList}{\ensuremath{\mathsf{list}}\xspace}
\newcommand{\sArray}{\ensuremath{\mathsf{array}}\xspace}
\newcommand{\sCList}{\ensuremath{\mathsf{clist}}\xspace}
\newcommand{\sLevel}{\ensuremath{\mathsf{level}}\xspace}
\newcommand{\sLevelK}{\ensuremath{\mathsf{level}_{\mathsf{K}}}\xspace}
\newcommand{\sOrd}{\ensuremath{\mathsf{ord}}\xspace}
\newcommand{\sAArray}{\ensuremath{\mathsf{aarr}}\xspace}
\newcommand{\sLArray}{\ensuremath{\mathsf{larr}}\xspace}
\newcommand{\sPair}{\ensuremath{\mathsf{pair}}\xspace}
\newcommand{\sSetPair}{\ensuremath{\mathsf{setpair}}\xspace}
\newcommand{\sMrgn}{\ensuremath{\mathsf{mrgn}}\xspace}
\newcommand{\sLock}{\ensuremath{\mathsf{sem}}\xspace}


%% Functions
\newcommand{\fNil}{\ensuremath{\mathit{nil}}\xspace}
\newcommand{\fCons}{\ensuremath{\mathit{cons}}\xspace}
\newcommand{\fHd}{\ensuremath{\mathit{hd}}\xspace}
\newcommand{\fTl}{\ensuremath{\mathit{tl}}\xspace}
\newcommand{\fError}{\ensuremath{\mathit{error}}\xspace}
\newcommand{\fMkcell}{\ensuremath{\mathit{mkcell}}\xspace}
\newcommand{\fData}{\ensuremath{\mathit{data}}\xspace}
\newcommand{\fSem}{\ensuremath{\mathit{lock}}\xspace}
\newcommand{\fKey}{\ensuremath{\mathit{key}}\xspace}
\newcommand{\fNext}{\ensuremath{\mathit{next}}\xspace}
\newcommand{\fLockID}{\ensuremath{\mathit{lockid}}\xspace}
\newcommand{\fLock}{\ensuremath{\mathit{lock}}\xspace}
\newcommand{\fUnlock}{\ensuremath{\mathit{unlock}}\xspace}
\newcommand{\fNull}{\ensuremath{\mathit{null}}\xspace}
\newcommand{\fNoThread}{\ensuremath{\oslash}\xspace}
\newcommand{\fUndef}{\ensuremath{\mathit{undef}}\xspace}
\newcommand{\fRd}{\ensuremath{\mathit{rd}}\xspace}
\newcommand{\fUpd}{\ensuremath{\mathit{upd}}\xspace}
\newcommand{\fRev}{\ensuremath{\mathit{rev}}\xspace}
\newcommand{\fEpsilon}{\ensuremath{\epsilon}\xspace}
\newcommand{\fPathToSet}{\ensuremath{\mathit{path2set}}\xspace}
\newcommand{\fPathToSetA}{\ensuremath{\mathit{\inter{path2set}{\Ai}}}\xspace}
\newcommand{\fPathToSetB}{\ensuremath{\mathit{\inter{path2set}{\Bi}}}\xspace}
\newcommand{\fSetToElemSet}{\ensuremath{\mathit{set2elemset}}\xspace}
\newcommand{\fAddrToSet}{\ensuremath{\mathit{addr2set}}\xspace}
\newcommand{\fAddrToSetK}{\ensuremath{\mathit{addr2set_{\K}}}\xspace}
\newcommand{\fGetp}{\ensuremath{\mathit{getp}}\xspace}
\newcommand{\fGetpK}{\ensuremath{\mathit{getp_{\K}}}\xspace}
\newcommand{\fPathToList}{\ensuremath{\mathit{path2list}}\xspace}
\newcommand{\fPathToListA}{\ensuremath{\inter{\fPathToList}{\Ai}}\xspace}
\newcommand{\fPathToListB}{\ensuremath{\inter{\fPathToList}{\Bi}}\xspace}
\newcommand{\fPathToCList}{\ensuremath{\mathit{path2clist}}\xspace}
\newcommand{\fPathToCListA}{\ensuremath{\inter{\fPathToCList}{\Ai}}\xspace}
\newcommand{\fPathToCListB}{\ensuremath{\inter{\fPathToCList}{\Bi}}\xspace}
%\newcommand{\fFirstLocked}{\ensuremath{\mathit{firstlocked}}\xspace}
\newcommand{\fFirstLocked}{\ensuremath{\mathit{fstlock}}\xspace}
\newcommand{\fLastLocked}{\ensuremath{\mathit{lstlock}}\xspace}
%\newcommand{\fFirstLockedK}{\ensuremath{\mathit{firstlocked_{\K}}}\xspace}
\newcommand{\fFirstLockedK}{\ensuremath{\mathit{fstlock_{\K}}}\xspace}
\newcommand{\fThreadSubset}{\ensuremath{\mathit{thSubset}}\xspace}
\newcommand{\fMinTh}{\ensuremath{\mathit{minTh}}\xspace}
\newcommand{\fMin}{\ensuremath{\mathit{min}}\xspace}

\newcommand{\fLUKSubset}{\ensuremath{\mathit{subset_{\mathit{LUH}}}}\xspace}
\newcommand{\fInterLUK}{\ensuremath{\mathit{inter_{\mathit{LUH}}}}\xspace}
\newcommand{\fEmptyLUK}{\ensuremath{\mathit{empty_{\mathit{LUH}}}}\xspace}
\newcommand{\fMinLUK}{\ensuremath{\mathit{min_{\mathit{LUH}}}}\xspace}


\newcommand{\fListToSet}{\ensuremath{\mathit{list2set}}\xspace}
\newcommand{\fListToSetA}{\ensuremath{\inter{\fListToSet}{\Ai}}\xspace}
\newcommand{\fListToSetB}{\ensuremath{\inter{\fListToSet}{\Bi}}\xspace}
\newcommand{\fSeqToList}{\ensuremath{\mathit{fseq2list}}\xspace}
\newcommand{\fSeqToCList}{\ensuremath{\mathit{fseq2clist}}\xspace}
%\newcommand{\fFirstMarked}{\ensuremath{\mathit{firstmarked}}\xspace}
\newcommand{\fFirstMarked}{\ensuremath{\mathit{fstmark}}\xspace}
\newcommand{\fIsPath}{\ensuremath{\mathit{ispath}}\xspace}
\newcommand{\fLevel}[1]{\ensuremath{\eta_{#1}}\xspace}
\newcommand{\fZero}{\ensuremath{0}\xspace}
\newcommand{\fMinusInfty}{\ensuremath{-\infty}\xspace}
\newcommand{\fInfty}{\ensuremath{+\infty}\xspace}
\newcommand{\fSucc}{\ensuremath{\mathsf{succ}}\xspace}
\newcommand{\fSkipKToList}{\ensuremath{\mathit{skiplist_{\mathsf{k}}2list}}\xspace}
\newcommand{\fApp}{\ensuremath{\mathit{app}}\xspace}
\newcommand{\fSeqToSet}{\ensuremath{\mathit{fseq2set}}\xspace}
\newcommand{\fLast}{\ensuremath{\mathit{last}}\xspace}
\newcommand{\fCont}{\ensuremath{\mathit{cont}}\xspace}
\newcommand{\pContAtLevelK}{\ensuremath{\mathit{cont_{\K}}}\xspace}
\newcommand{\pContAtLevel}{\ensuremath{\mathit{contAtLevel}}\xspace}
\newcommand{\fPair}[2]{\ensuremath{\langle{#1},{#2}\rangle}\xspace}

\newcommand{\fSingl}{\ensuremath{\mathit{singl}}\xspace}
\newcommand{\fUnion}{\ensuremath{\mathit{Union}}\xspace}
\newcommand{\fSetdiff}{\ensuremath{\mathit{setDiff}}\xspace}
\newcommand{\fEmptyset}{\ensuremath{\mathit{emptyset}}\xspace}
\newcommand{\pIn}{\ensuremath{\mathit{in}}\xspace}

\newcommand{\fSinglElem}{\ensuremath{\mathit{singlElem}}\xspace}
\newcommand{\fUnionElem}{\ensuremath{\mathit{UnionElem}}\xspace}
\newcommand{\fSetdiffElem}{\ensuremath{\mathit{setDiffElem}}\xspace}
\newcommand{\fEmptysetElem}{\ensuremath{\mathit{emptysetElem}}\xspace}
\newcommand{\pInElem}{\ensuremath{\mathit{inElem}}\xspace}

\newcommand{\fSinglTid}{\ensuremath{\mathit{singlTh}}\xspace}
\newcommand{\fUnionTid}{\ensuremath{\mathit{UnionTh}}\xspace}
\newcommand{\fSetdiffTid}{\ensuremath{\mathit{setDiffTh}}\xspace}
\newcommand{\fEmptysetTid}{\ensuremath{\mathit{emptysetTh}}\xspace}
\newcommand{\pInTid}{\ensuremath{\mathit{inTh}}\xspace}

\newcommand{\fHighest}{\ensuremath{\mathit{highestElem}}\xspace}
\newcommand{\fLowest}{\ensuremath{\mathit{lowestElem}}\xspace}
\newcommand{\fLselem}{\ensuremath{\mathit{ls\_elem}}\xspace}


%% Operations for the generic array manipulation
\newcommand{\fArrayRd}[2]
		{\ensuremath{{#1} \lbrack {#2} \rbrack}\xspace}
\newcommand{\fArrayUpd}[3]
		{\ensuremath{{#1} \{{#2} \gets {#3} \}}\xspace}
\newcommand{\fArrayRdInter}[3]
		{\ensuremath{{#1} \lbrack {#2} \rbrack^{#3}}\xspace}
\newcommand{\fArrayUpdInter}[4]
		{\ensuremath{{#1} \{{#2} \gets {#3} \}^{#4}}\xspace}

\newcommand{\contained}{\ensuremath{\mathit{contained}}\xspace}


%% Predicates
\newcommand{\pOrdPath}{\ensuremath{\mathit{ordPath}}\xspace}
\newcommand{\pOrdList}{\ensuremath{\mathit{ordList}}\xspace}
\newcommand{\pSubPath}{\ensuremath{\mathit{subPath}}\xspace}
\newcommand{\pAppend}{\ensuremath{\mathit{append}}\xspace}
\newcommand{\pReach}{\ensuremath{\mathit{reach}}\xspace}
\newcommand{\pReachK}{\ensuremath{\mathit{reach_{\K}}}\xspace}
%\newcommand{\pIsReachable}{\eniguremath{\mathit{isreachable}}\xspace}
\newcommand{\pIsReachable}{\eniguremath{\mathit{isreach}}\xspace}
\newcommand{\pOrd}{\ensuremath{\preceq}\xspace}
\newcommand{\pStrictOrd}{\ensuremath{\prec}\xspace}
\newcommand{\pSubList}{\ensuremath{\mathit{SubList}}\xspace}
\newcommand{\pSkiplistK}{\ensuremath{\mathit{Skiplist_{\mathsf{K}}}}\xspace}
\newcommand{\pDisjLevelK}{\ensuremath{\mathit{DisjLevel_{\mathsf{K}}}}\xspace}
%\newcommand{\pIsReach}{\ensuremath{\mathit{isreachable}}\xspace}
\newcommand{\pIsReach}{\ensuremath{\mathit{isreach}}\xspace}
%\newcommand{\pIsReachP}{\ensuremath{\mathit{isreachablep}}\xspace}
\newcommand{\pIsReachP}{\ensuremath{\mathit{isreachp}}\xspace}
%\newcommand{\pIsReachK}{\ensuremath{\mathit{isreachable_{\K}}}\xspace}
\newcommand{\pIsReachK}{\ensuremath{\mathit{isreach_{\K}}}\xspace}
%\newcommand{\pIsReachKP}{\ensuremath{\mathit{isreachablep_{\K}}}\xspace}
\newcommand{\pIsReachKP}{\ensuremath{\mathit{isreachp_{\K}}}\xspace}

\newcommand{\pInTidPair}{\ensuremath{\mathit{inTidPair}}\xspace}
\newcommand{\pUniqueInt}{\ensuremath{\mathit{uniqueInt}}\xspace}
\newcommand{\pUniqueTid}{\ensuremath{\mathit{uniqueTid}}\xspace}



%% Signatures
\newcommand{\sigAddr}{\ensuremath{\Sigma_{\addr}}\xspace}
\newcommand{\sigElem}{\ensuremath{\Sigma_{\elem}}\xspace}
\newcommand{\sigTid}{\ensuremath{\Sigma_{\thId}}\xspace}
\newcommand{\sigCells}{\ensuremath{\Sigma_{\cells}}\xspace}
\newcommand{\sigCellsK}{\ensuremath{\Sigma_{\cellsK}}\xspace}
\newcommand{\sigMemory}{\ensuremath{\Sigma_{\memory}}\xspace}
\newcommand{\sigMemoryK}{\ensuremath{\Sigma_{\memoryK}}\xspace}
\newcommand{\sigReach}{\ensuremath{\Sigma_{\reach}}\xspace}
\newcommand{\sigReachK}{\ensuremath{\Sigma_{\reachK}}\xspace}
\newcommand{\sigSets}{\ensuremath{\Sigma_{\sSet}}\xspace}
\newcommand{\sigSetsE}{\ensuremath{\Sigma_{\sSetE}}\xspace}
\newcommand{\sigSetsT}{\ensuremath{\Sigma_{\sSetT}}\xspace}
\newcommand{\sigBridge}{\ensuremath{\Sigma_{\bridge}}\xspace}
\newcommand{\sigBridgeK}{\ensuremath{\Sigma_{\bridgeK}}\xspace}
\newcommand{\sigLists}{\ensuremath{\Sigma_{\cons}}\xspace}
\newcommand{\sigListsBridge}{\ensuremath{\Sigma_{\listsBridge}}\xspace}
\newcommand{\sigSeq}{\ensuremath{\Sigma_{\sSeq}}\xspace}
\newcommand{\sigPath}{\ensuremath{\Sigma_{\sPath}}\xspace}
\newcommand{\sigMrgn}{\ensuremath{\Sigma_{\sMrgn}}\xspace}
\newcommand{\sigTLL}{\ensuremath{\Sigma_{\TLL}}\xspace}
\newcommand{\sigTLLh}{\ensuremath{\Sigma_{\TLLh}}\xspace}
\newcommand{\sigTLLpL}{\ensuremath{\Sigma_{\TLLpL}}\xspace}
\newcommand{\sigTLLpLh}{\ensuremath{\Sigma_{\TLLpLh}}\xspace}
\newcommand{\sigLevel}{\ensuremath{\Sigma_{\sLevel}}\xspace}
\newcommand{\sigLevelK}{\ensuremath{\Sigma_{\sLevelK}}\xspace}
\newcommand{\sigOrd}{\ensuremath{\Sigma_{\sOrd}}\xspace}
\newcommand{\sigBasepL}{\ensuremath{\Sigma_{\textsf{Base3}}}\xspace}
\newcommand{\sigAArray}{\ensuremath{\Sigma_{\sAArray}}\xspace}
\newcommand{\sigLArray}{\ensuremath{\Sigma_{\sLArray}}\xspace}
\newcommand{\sigPair}{\ensuremath{\Sigma_{\sPair}}\xspace}
\newcommand{\sigSetPair}{\ensuremath{\Sigma_{\sSetPair}}\xspace}
\newcommand{\sigLUK}{\ensuremath{\Sigma_{\mathsf{LUH}}}\xspace}
\newcommand{\sigTSLK}{\ensuremath{\Sigma_{\TSLK}}\xspace}
\newcommand{\sigTSLKh}{\ensuremath{\Sigma_{\TSLKh}}\xspace}
\newcommand{\sigTSLKhBar}{\ensuremath{\Sigma_{\TSLKhBar}}\xspace}
\newcommand{\sigTBaseSLK}{\ensuremath{\Sigma_{\TBaseSLK}}\xspace}
\newcommand{\sigTBaseSLKP}{\ensuremath{\Sigma_{\TBaseSLKP}}\xspace}


%% Interpretations
\newcommand{\interTLL}{\ensuremath{\textbf{TLL}}\xspace}
\newcommand{\interTLLpL}{\ensuremath{\textbf{TL3}}\xspace}
\newcommand{\interTOrd}{\ensuremath{\textbf{TOrd}}\xspace}
\newcommand{\interTSLK}{\ensuremath{\textbf{TSLK}}\xspace}


%% Theories
\newcommand{\thCells}{\ensuremath{\mathit{T}_{\cells}}\xspace}
\newcommand{\thElem}{\ensuremath{\mathit{T}_{\elem}}\xspace}
\newcommand{\thTid}{\ensuremath{\mathit{T}_{\thId}}\xspace}
\newcommand{\thAddr}{\ensuremath{\mathit{T}_{\addr}}\xspace}
\newcommand{\thMem}{\ensuremath{\mathit{T}_{\memory}}\xspace}
\newcommand{\thReach}{\ensuremath{\mathit{T}_{\reach}}\xspace}
\newcommand{\thPath}{\ensuremath{\mathit{T}_{\path}}\xspace}
\newcommand{\thSets}{\ensuremath{\mathit{T}_{\sets}}\xspace}
\newcommand{\thSetsE}{\ensuremath{\mathit{T}_{\setsE}}\xspace}
\newcommand{\thSetsT}{\ensuremath{\mathit{T}_{\setsT}}\xspace}
\newcommand{\thSeq}{\ensuremath{\mathit{T}_{\sSeq}}\xspace}
\newcommand{\thLists}{\ensuremath{\mathit{T}_{\cons}}\xspace}
\newcommand{\thMrgn}{\ensuremath{\mathit{T}_{\sMrgn}}\xspace}
\newcommand{\thAArray}{\ensuremath{\mathit{T}_{\sAArray}}\xspace}
\newcommand{\thLArray}{\ensuremath{\mathit{T}_{\sLArray}}\xspace}
\newcommand{\thLUK}{\ensuremath{\mathit{T}_{\mathsf{LUH}}}\xspace}
\newcommand{\thPair}{\ensuremath{\mathit{T}_{\sPair}}\xspace}
\newcommand{\thSetPair}{\ensuremath{\mathit{T}_{\sSetPair}}\xspace}
\newcommand{\thBridge}{\ensuremath{\mathit{T}_{\bridge}}\xspace}

\newcommand{\TLL}{\ensuremath{\textsf{TLL}}\xspace}
\newcommand{\TLLh}{\ensuremath{\widehat{\textsf{TLL}}}\xspace}
\newcommand{\TLLpL}{\ensuremath{\textsf{TL3}}\xspace}
\newcommand{\TLLpLh}{\ensuremath{\widehat{\textsf{TL3}}}\xspace}
\newcommand{\TBase}{\ensuremath{\mathit{T}_{\textsf{Base}}}\xspace}
\newcommand{\TBasepL}{\ensuremath{\mathit{T}_{\textsf{Base3}}}\xspace}
\newcommand{\TBaseSLK}{\ensuremath{\mathit{T}_{\textsf{SLKBase}}}\xspace}
\newcommand{\TBaseSLKP}{\ensuremath{\mathit{T}_{\textsf{SLKBase+}}}\xspace}
\newcommand{\TSL}{\ensuremath{\textsf{TSL}}\xspace}
\newcommand{\K}{\ensuremath{\textsf{K}}\xspace}
\newcommand{\TSLK}{\ensuremath{\TSL_{\K}}\xspace}
\newcommand{\TSLKplus}{\ensuremath{\TSL_{\K}+}\xspace}
\newcommand{\TSLKh}{\ensuremath{\widehat{\TSLK}}\xspace}
\newcommand{\TSLKhBar}{\ensuremath{\overline{\widehat{\TSLK}}}\xspace}
\newcommand{\TN}{\ensuremath{\T_{\Nat}}\xspace}
\newcommand{\TLevel}{\ensuremath{\mathit{T}_{\sLevel}}\xspace}
\newcommand{\TLevelK}{\ensuremath{\mathit{T}_{\sLevelK}}\xspace}
\newcommand{\TOrd}{\ensuremath{\mathit{T}_{\sOrd}}\xspace}


%% Lists operations for theory of ideal lists
\newcommand{\lempty}{\ensuremath{\langle \rangle}\xspace}
\newcommand{\lemptyA}{\ensuremath{\inter{\lempty}{\Ai}}\xspace}
\newcommand{\lemptyB}{\ensuremath{\inter{\lempty}{\Bi}}\xspace}
\newcommand{\lcons}{\ensuremath{::}\xspace}
\newcommand{\lconsA}{\ensuremath{\inter{\lcons}{\Ai}}\xspace}
\newcommand{\lconsB}{\ensuremath{\inter{\lcons}{\Bi}}\xspace}
\newcommand{\lcar}{\ensuremath{\mathsf{car}}\xspace}
\newcommand{\lcdr}{\ensuremath{\mathsf{cdr}}\xspace}


%% Set operations for the Theory of Set of Elements
\newcommand{\seSym}{\textsc{e}}
\newcommand{\seEmpty}{\ensuremath{\emptyset_{\seSym}}\xspace}
\newcommand{\seSingle}[1]{\ensuremath{\{{#1}\}_{\seSym}}\xspace}
\newcommand{\seCup}{\ensuremath{\cup_{\seSym}}\xspace}
\newcommand{\seCap}{\ensuremath{\cap_{\seSym}}\xspace}
\newcommand{\seSubst}{\ensuremath{\setminus_{\seSym}}\xspace}
\newcommand{\seIn}{\ensuremath{\in_{\seSym}}\xspace}
\newcommand{\seSubset}{\ensuremath{\subseteq_{\seSym}}\xspace}


%% Set operations for the Theory of Set of thread ids
\newcommand{\stSym}{\textsc{t}}
\newcommand{\stEmpty}{\ensuremath{\emptyset_{\stSym}}\xspace}
\newcommand{\stSingle}[1]{\ensuremath{\{{#1}\}_{\stSym}}\xspace}
\newcommand{\stCup}{\ensuremath{\cup_{\stSym}}\xspace}
\newcommand{\stCap}{\ensuremath{\cap_{\stSym}}\xspace}
\newcommand{\stSubst}{\ensuremath{\setminus_{\stSym}}\xspace}
\newcommand{\stIn}{\ensuremath{\in_{\stSym}}\xspace}
\newcommand{\stSubset}{\ensuremath{\subseteq_{\stSym}}\xspace}


%% Set operations for the Theory of Set of pairs
\newcommand{\spSym}{\textsc{p}}
\newcommand{\spEmpty}{\ensuremath{\emptyset_{\spSym}}\xspace}
\newcommand{\spSingle}[1]{\ensuremath{\{{#1}\}_{\spSym}}\xspace}
\newcommand{\spCup}{\ensuremath{\cup_{\spSym}}\xspace}
\newcommand{\spCap}{\ensuremath{\cap_{\spSym}}\xspace}
\newcommand{\spSubst}{\ensuremath{\setminus_{\spSym}}\xspace}
\newcommand{\spIn}{\ensuremath{\in_{\spSym}}\xspace}
\newcommand{\spSubset}{\ensuremath{\subseteq_{\spSym}}\xspace}


%% Set operations for the Theory of Masked Regions
\newcommand{\mrSym}{\ensuremath{\textsc{mr}}}
\newcommand{\mrgnEmpty}{\ensuremath{\emp_{\mrSym}}\xspace}
\newcommand{\mrgnSingle}[1]{\ensuremath{\langle{#1}\rangle_{\mrSym}}\xspace}
\newcommand{\mrgnCup}{\ensuremath{\cup_{\mrSym}}\xspace}
\newcommand{\mrgnCap}{\ensuremath{\cap_{\mrSym}}\xspace}
\newcommand{\mrgnMinus}{\ensuremath{-_{\mrSym}}\xspace}
\newcommand{\mrgnIn}{\ensuremath{\in_{\mrSym}}\xspace}
\newcommand{\mrgnSubset}{\ensuremath{\subseteq_{\mrSym}}\xspace}
\newcommand{\mrgnDisj}{\ensuremath{\#_{\mrSym}}\xspace}


%% Operations for the generic array manipulation
\newcommand{\arrRd}[2]
		{\ensuremath{{#1} \lbrack {#2} \rbrack}\xspace}
\newcommand{\arrUpd}[3]
		{\ensuremath{{#1} \{{#2} \gets {#3} \}}\xspace}


%% Operations for the Theory of Array of Addressee
\newcommand{\aarrSym}{\ensuremath{\textsc{a}}}
\newcommand{\aarrRd}[2]
		{\ensuremath{{#1} \lbrack {#2} \rbrack_{\aarrSym}}\xspace}
\newcommand{\aarrUpd}[3]
		{\ensuremath{{#1} \{{#2} \gets {#3} \}_{\aarrSym}}\xspace}


%% Operations for the Theory of Array of Addressee
\newcommand{\larrSym}{\ensuremath{\textsc{l}}}
\newcommand{\larrRd}[2]
		{\ensuremath{{#1} \lbrack {#2} \rbrack_{\larrSym}}\xspace}
\newcommand{\larrUpd}[3]
		{\ensuremath{{#1} \{{#2} \gets {#3} \}_{\larrSym}}\xspace}


%% Interpreted set operations for the Theory of Masked Regions
\newcommand{\mrgnEmptyI}[1]{\ensuremath{\emp_{\mrSym}^{#1}}\xspace}
\newcommand{\mrgnSingleI}[2]{\ensuremath{\langle{#1}\rangle_{\mrSym}^{#2}}\xspace}
\newcommand{\mrgnCupI}[1]{\ensuremath{\cup_{\mrSym}^{#1}}\xspace}
\newcommand{\mrgnCapI}[1]{\ensuremath{\cap_{\mrSym}^{#1}}\xspace}
\newcommand{\mrgnMinusI}[1]{\ensuremath{-_{\mrSym}^{#1}}\xspace}
\newcommand{\mrgnInI}[1]{\ensuremath{\in_{\mrSym}^{#1}}\xspace}
\newcommand{\mrgnSubsetI}[1]{\ensuremath{\subseteq_{\mrSym}^{#1}}\xspace}
\newcommand{\mrgnDisjI}[1]{\ensuremath{\#_{\mrSym}^{#1}}\xspace}


%% Interpreted operations for the Theory of Array of Addressee
\newcommand{\aarrRdI}[3]
		{\ensuremath{{#1} \lbrack {#2} \rbrack_{\aarrSym}^{#3}}\xspace}
\newcommand{\aarrUpdI}[4]
		{\ensuremath{{#1} \{{#2} \gets {#3} \}_{\aarrSym}^{#4}}\xspace}


%% Interpreted operations for the Theory of Array of Addressee
\newcommand{\larrRdI}[3]
		{\ensuremath{{#1} \lbrack {#2} \rbrack_{\larrSym}^{#3}}\xspace}
\newcommand{\larrUpdI}[4]
		{\ensuremath{{#1} \{{#2} \gets {#3} \}_{\larrSym}^{#4}}\xspace}


%% Auxiliary structures names for theories
\newcommand{\TGen}{\ensuremath{\mathsf{TGen}}\xspace}
\newcommand{\ETGen}{\ensuremath{\mathsf{ETGen}}\xspace}
\newcommand{\ETGenh}{\ensuremath{\widehat{\mathsf{ETGen}}}\xspace}
\newcommand{\Mod}[2]{\ensuremath{\mathit{Mod}^{#1}(#2)}\xspace}
\newcommand{\AxPATH}{\ensuremath{\mathit{PATH}}\xspace}
\newcommand{\GAP}{\ensuremath{\mathit{GAP}}\xspace}
\newcommand{\BASEP}{\ensuremath{\mathit{BASE+}}\xspace}
\newcommand{\Ax}[1]{\ensuremath{\mathsf{Ax}_{#1}}\xspace}
\newcommand{\kacyclic}{\ensuremath{\mathsf{k-acyclicity}}\xspace}
\newcommand{\generatedness}{\ensuremath{\mathsf{Generatedness}}\xspace}
\newcommand{\acyclic}{\ensuremath{\mathit{acyclic}}\xspace}


%% Interpretations names
\newcommand{\inter}[2]{\ensuremath{\mathit{#1}^{#2}}\xspace}
\newcommand{\Ai}{\ensuremath{\mathcal{A}}\xspace}
\newcommand{\Bi}{\ensuremath{\mathcal{B}}\xspace}
\newcommand{\Bip}{\ensuremath{\mathcal{B}^{+}}\xspace}
\newcommand{\Ais}[1]{\ensuremath{\Ai_{#1}}\xspace}
\newcommand{\Bis}[1]{\ensuremath{\Bi_{#1}}\xspace}
\newcommand{\BisTilde}[1]{\ensuremath{\tilde{\mathcal{B}}_{#1}}\xspace}
\newcommand{\SAi}{\ensuremath{\mathit{S}_{\hspace{-0.15em}\Ai}}\xspace}
\newcommand{\SBi}{\ensuremath{\mathit{S}_{\hspace{-0.05em}\Bi}}\xspace}


%% Variables names
\newcommand{\TVar}[1]{\ensuremath{\mathit{V}_{#1}}\xspace}
\newcommand{\VarA}[1]{\ensuremath{\mathit{V}_{#1}^{\Ai}}\xspace}
\newcommand{\VarB}[1]{\ensuremath{\mathit{V}_{#1}^{\Bi}}\xspace}


%% Auxiliary functions for theories definitions
\newcommand{\fa}{\ensuremath{f_a}\xspace}
\newcommand{\fb}{\ensuremath{f_b}\xspace}
\newcommand{\Marked}{\ensuremath{\mathit{marked}}\xspace}
\newcommand{\NoMarks}{\ensuremath{\mathit{NoMarks}}\xspace}
\newcommand{\SomeMark}{\ensuremath{\mathit{SomeMark}}\xspace}
\newcommand{\LastMarked}{\ensuremath{\mathit{LastMarked}}\xspace}
\newcommand{\MarkedList}{\ensuremath{\mathit{MarkedList}}\xspace}
\newcommand{\LastLockedAddr}{\ensuremath{\mathit{LastLockedAddr}}\xspace}
\newcommand{\NoLocks}{\ensuremath{\mathit{NoLocks}}\xspace}
\newcommand{\NoChange}{\ensuremath{\mathit{NoChange}}\xspace}
\newcommand{\AboutToChange}{\ensuremath{\mathit{AboutToChange}}\xspace}
\newcommand{\IsLast}{\ensuremath{\mathit{IsLast}}\xspace}
\newcommand{\NoIsLast}{\ensuremath{\mathit{IsNotLast}}\xspace}
\newcommand{\KEnds}{\ensuremath{\mathit{K_{ends}}}\xspace}
\newcommand{\tranEnd}{\ensuremath{\mathit{end}}\xspace}
\newcommand{\tPre}{\ensuremath{\mathit{t^{pre}}}\xspace}
\newcommand{\deltaL}{\ensuremath{\delta_{L}}\xspace}

\newcommand{\fstL}{\ensuremath{\mathit{fstL}}\xspace}
\newcommand{\unordered}{\ensuremath{\mathit{unordered}}\xspace}
\newcommand{\diseqPath}{\ensuremath{\mathit{diseq}}\xspace}
\newcommand{\common}{\ensuremath{\mathit{common}}\xspace}
\newcommand{\knownTID}{\ensuremath{\mathit{knownTID}}\xspace}
\newcommand{\NUM}[1]{\ensuremath{\overline{#1}}}

\newcommand{\first}{\ensuremath{\mathit{first}}\xspace}
\newcommand{\firstK}{\ensuremath{\mathit{first}_{\K}}\xspace}
\newcommand{\compress}{\ensuremath{\mathit{compress}}\xspace}
\newcommand{\compressA}{\ensuremath{\inter{\compress}{\Ai}}\xspace}
\newcommand{\compressB}{\ensuremath{\inter{\compress}{\Bi}}\xspace}
\newcommand{\compressList}{\ensuremath{\mathit{compressList}}\xspace}
\newcommand{\compressListA}{\ensuremath{\inter{\compressList}{\Ai}}\xspace}
\newcommand{\compressListB}{\ensuremath{\inter{\compressList}{\Bi}}\xspace}


%% Sets names
\newcommand{\setL}{\ensuremath{\mathit{L}}\xspace}
\newcommand{\setO}{\ensuremath{\mathit{O}}\xspace}
\newcommand{\setX}{\ensuremath{\mathit{X}}\xspace}
\newcommand{\setY}{\ensuremath{\mathit{Y}}\xspace}
\newcommand{\setZ}{\ensuremath{\mathit{Z}}\xspace}


%% Shortcuts for theory manipulation
\newcommand{\ai}[2]{\ensuremath{\mathit{a}_{#1}^{#2}}\xspace}
\newcommand{\hp}[2]{\ensuremath{\mathit{h}_{{#1} \mapsto {#2}}}\xspace}
\newcommand{\ma}{\ensuremath{\inter{m}{\Ai}}\xspace}
\newcommand{\mb}{\ensuremath{\inter{m}{\Bi}}\xspace}


%%%%%%%%%%%%%%%%%%%%%%%%%%%%%%%%%%%%%%%%%%%%%%%%%
%%                                             %%
%%  Concurrent lock-coupling example commands  %%
%%                                             %%
%%%%%%%%%%%%%%%%%%%%%%%%%%%%%%%%%%%%%%%%%%%%%%%%%

\newcommand{\ra}{\ensuremath{r_a}\xspace}
\newcommand{\rb}{\ensuremath{r_b}\xspace}
\newcommand{\setT}{\ensuremath{\mathit{set}_{T}}\xspace}
\newcommand{\thSetAP}{\ensuremath{\mathit{TSet}_{prev}}\xspace}
\newcommand{\thSetAC}{\ensuremath{\mathit{TSet}_{curr}}\xspace}
\newcommand{\thSetB}{\ensuremath{\mathit{TSet}_{bef}}\xspace}

\newcommand{\ThSetIn}{\ensuremath{\mathit{In}}\xspace}
\newcommand{\ThSetOut}{\ensuremath{\mathit{Out}}\xspace}
\newcommand{\SerIn}{\ensuremath{\mathit{Ser_{\ThSetIn}}}\xspace}
\newcommand{\SerOut}{\ensuremath{\mathit{Ser_{\ThSetOut}}}\xspace}
\newcommand{\kisin}{\ensuremath{\mathit{k\_is\_in}}\xspace}

\newcommand{\serB}{\ensuremath{\mathit{ser}_{b}}\xspace}
\newcommand{\serA}{\ensuremath{\mathit{ser}_{a}}\xspace}
\newcommand{\iisin}{\ensuremath{\mathit{i\_is\_in}}\xspace}
\newcommand{\myid}{\ensuremath{\mathit{my\_id}}\xspace}
\newcommand{\NLA}{\ensuremath{\mathit{NLA}}\xspace}
\newcommand{\myTicket}{\ensuremath{\mathit{myTicket}}\xspace}
\newcommand{\ahead}{\ensuremath{\mathit{ahead}}\xspace}


%%%%%%%%%%%%%%%%%%%%%%%%
%%                    %%
%%  Stack operations  %%
%%                    %%
%%%%%%%%%%%%%%%%%%%%%%%%

\newcommand{\StkEmpty}{\ensuremath{\mathit{isEmpty}}\xspace}
\newcommand{\StkSize}{\ensuremath{\mathit{size}}\xspace}
\newcommand{\StkPush}{\ensuremath{\mathit{push}}\xspace}
\newcommand{\StkPop}{\ensuremath{\mathit{pop}}\xspace}
\newcommand{\StkTop}{\ensuremath{\mathit{top}}\xspace}
\newcommand{\StkPretop}{\ensuremath{\mathit{pretop}}\xspace}
\newcommand{\StkIncTop}{\ensuremath{\mathit{incTop}}\xspace}
\newcommand{\Set}{\ensuremath{\mathit{Set}}\xspace}
\newcommand{\Stack}{\ensuremath{\mathit{Stack}}\xspace}
\newcommand{\IStack}{\ensuremath{\mathit{I_{\Stack}}}\xspace}
\newcommand{\IPrevStack}{\ensuremath{\mathit{I_{Prev\Stack}}}\xspace}
\newcommand{\IPrevStackP}{\ensuremath{\mathit{I'_{Prev\Stack}}}\xspace}
\newcommand{\IStackTop}{\ensuremath{\mathit{I_{\Stack}^{top}}}\xspace}
\newcommand{\IStackNTop}{\ensuremath{\mathit{I_{\Stack}^{nontop}}}\xspace}
\newcommand{\IStackP}{\ensuremath{\mathit{I'_{\Stack}}}\xspace}
\newcommand{\IStackTopP}{\ensuremath{\mathit{I_{\Stack}^{top'}}}\xspace}
\newcommand{\IStackNTopP}{\ensuremath{\mathit{I_{\Stack}^{nontop'}}}\xspace}



%%%%%%%%%%%%%%%%%%%%%%%%%%%%%
%%                         %%
%%  Region function names  %%
%%                         %%
%%%%%%%%%%%%%%%%%%%%%%%%%%%%%
\newcommand{\mr}[3]{\ensuremath{\vt{m}{#1}_{#2..#3}}}
\newcommand{\mskreg}[1]{\ensuremath{\vt{m}{#1}_{r}}}
\newcommand{\rgn}{\ensuremath{\mathbf{rgn}}\xspace}
\newcommand{\mrgn}{\ensuremath{\mathbf{mrgn}}\xspace}
\newcommand{\tdot}{\tbf{.}}
\newcommand{\wre}{\text{\textbf{wr }}}
\newcommand{\rde}{\text{\textbf{rd }}}
\newcommand{\sreg}[1]{\left\langle {#1} \right\rangle}
\newcommand{\mreg}[1]{({#1})}
\newcommand{\emp}{\text{\textbf{emp}}\xspace}
\newcommand{\luk}{\ensuremath{(L,U,H)}\xspace}
\newcommand{\me}{\ensuremath{\mathit{me}}\xspace}


%%%%%%%%%%%%%%%%%%%%%%%%%%%%%%%%%%
%%                              %%
%%  Lists names and operations  %%
%%                              %%
%%%%%%%%%%%%%%%%%%%%%%%%%%%%%%%%%%
\newcommand{\ListNode}{\ensuremath{\mathit{ListNode}}\xspace}
\newcommand{\UnboundedQueueNode}{\ensuremath{\mathit{UnboundedQueueNode}}\xspace}
\newcommand{\LockfreeStackNode}{\ensuremath{\mathit{LockfreeStackNode}}\xspace}
\newcommand{\LockfreeQueueNode}{\ensuremath{\mathit{LockfreeQueueNode}}\xspace}
\newcommand{\Node}{\ensuremath{\mathit{Node}}\xspace}
\newcommand{\NodeP}{\ensuremath{\mathit{Node^{*}}}\xspace}
\newcommand{\Nodes}{\ensuremath{\mathit{Nodes}}\xspace}
\newcommand{\DNode}{\ensuremath{\mathit{DoubleNode}}\xspace}
\newcommand{\DList}{\ensuremath{\mathit{DoubleList}}\xspace}
\newcommand{\Prev}{\ensuremath{\mathit{Prev}}\xspace}
\newcommand{\NodeSize}{\ensuremath{\mathit{NodeSize}}\xspace}
\newcommand{\NodeSizeUpTo}{\ensuremath{\mathit{NodeSizeUpTo}}\xspace}
\newcommand{\KeepSize}{\ensuremath{\mathit{KeepSize}}\xspace}
\newcommand{\EqN}{\ensuremath{\mathit{EqN}}\xspace}
\newcommand{\EqNode}{\ensuremath{\mathit{EqNode}}\xspace}
\newcommand{\EqNodeUpTo}{\ensuremath{\mathit{EqNodeUpTo}}\xspace}
\newcommand{\DisjList}{\ensuremath{\mathit{DisjList}}\xspace}
\newcommand{\ListUpTo}{\ensuremath{\mathit{ListUpTo}}\xspace}
\newcommand{\OrdListUpTo}{\ensuremath{\mathit{OrdListUpTo}}\xspace}
\newcommand{\OrdListUpToLevel}{\ensuremath{\mathit{OrdListUpToLevel}}\xspace}
\newcommand{\CircList}{\ensuremath{\mathit{CircList}}\xspace}
\newcommand{\RevInitUpTo}{\ensuremath{\mathit{RevInitUpTo}}\xspace}
\newcommand{\EqSkipUpTo}{\ensuremath{\mathit{EqSkipUpTo}}\xspace}
\newcommand{\RevConcat}{\ensuremath{\mathit{RevConcat}}\xspace}
\newcommand{\IsInList}{\ensuremath{\mathit{IsInList}}\xspace}
\newcommand{\NotInList}{\ensuremath{\mathit{NotInList}}\xspace}
\newcommand{\List}{\ensuremath{\mathit{List}}\xspace}
\newcommand{\RegInv}{\ensuremath{\mathit{RegInv}}\xspace}
\newcommand{\forward}{\ensuremath{\mathit{forward}}\xspace}
\newcommand{\nextptr}{\ensuremath{\mathit{nextptr}}\xspace}
\newcommand{\last}{\ensuremath{\mathit{last}}\xspace}
\newcommand{\size}{\ensuremath{\mathit{size}}\xspace}
\newcommand{\Head}{\ensuremath{\mathit{Head}}\xspace}
\newcommand{\head}{\ensuremath{\mathit{head}}\xspace}
\newcommand{\Tail}{\ensuremath{\mathit{Tail}}\xspace}
\newcommand{\tail}{\ensuremath{\mathit{tail}}\xspace}
\newcommand{\tNext}{\ensuremath{\mathit{next}}\xspace}
\newcommand{\tList}{\ensuremath{\mathit{list}}\xspace}
\newcommand{\rs}{\ensuremath{\mathit{rs}}\xspace}
\providecommand*{\ts}{} % Cesar: otherwise, it \ts is undefined, it crashes
\renewcommand{\ts}{\ensuremath{\mathit{ts}}\xspace}
\newcommand{\fs}{\ensuremath{\mathit{fs}}\xspace}

\newcommand{\ListToSet}{\ensuremath{\mathit{ListToSet}}\xspace}
\newcommand{\OrdList}{\ensuremath{\mathit{OrdList}}\xspace}
\newcommand{\Ordered}{\ensuremath{\mathit{Ordered}}\xspace}
\newcommand{\ChkP}{\ensuremath{\mathit{ChkSer}}\xspace}

\newcommand{\goess}{\ensuremath{\hookrightarrow}}
\newcommand{\goesn}{\ensuremath{\goess^{n}}}
\newcommand{\goesk}[1]{\ensuremath{\goess^{#1}}}

\newcommand{\Op}{\ensuremath{\mathit{Op}}\xspace}
\newcommand{\AddOp}{\ensuremath{\mathit{Add_{Op}}}\xspace}
\newcommand{\RemOp}{\ensuremath{\mathit{Rem_{Op}}}\xspace}
\newcommand{\FindTOp}{\ensuremath{\mathit{FindT_{Op}}}\xspace}
\newcommand{\FindFOp}{\ensuremath{\mathit{FindF_{Op}}}\xspace}

\newcommand{\preds}{\ensuremath{\mathit{preds}}\xspace}
\newcommand{\succs}{\ensuremath{\mathit{succs}}\xspace}
\newcommand{\lFound}{\ensuremath{\mathit{lFound}}\xspace}

\newcommand{\pred}{\ensuremath{\mathit{pred}}\xspace}
\newcommand{\curr}{\ensuremath{\mathit{curr}}\xspace}
\newcommand{\update}{\ensuremath{\mathit{update}}\xspace}
\newcommand{\upd}{\ensuremath{\mathit{upd}}\xspace}

\newcommand{\cover}{\ensuremath{\mathit{cover}}\xspace}
\newcommand{\lvl}{\ensuremath{\mathit{lvl}}\xspace}

%% Concurrent lists predicates
\newcommand{\Respect}{\ensuremath{\mathit{Respect}}\xspace}
\newcommand{\AfterMe}{\ensuremath{\mathit{AfterMe}}\xspace}
\newcommand{\BeforeMe}{\ensuremath{\mathit{BeforeMe}}\xspace}
\newcommand{\BeforeSer}{\ensuremath{\mathit{BeforeSer}}\xspace}
\newcommand{\AfterSer}{\ensuremath{\mathit{AfterSer}}\xspace}


%%%%%%%%%%%%%%%%%%%%%%%%%%%%%%%%%%%%%%
%%                                  %%
%%  Skiplists names and operations  %%
%%                                  %%
%%%%%%%%%%%%%%%%%%%%%%%%%%%%%%%%%%%%%%
\newcommand{\ConcSkiplistNode}{\ensuremath{\mathit{BoundedSkiplistNode}}\xspace}
\newcommand{\UnboundedSkiplistNode}{\ensuremath{\mathit{UnboundedSkiplistNode}}\xspace}
\newcommand{\SkipList}{\ensuremath{\mathit{SkipList}}\xspace}
\newcommand{\IsSkipList}{\ensuremath{\mathit{IsSkipList}}\xspace}
\newcommand{\ListUpToLevel}{\ensuremath{\mathit{ListUpToLevel}}\xspace}
\newcommand{\SubPath}{\ensuremath{\mathit{SubPath}}\xspace}
\newcommand{\SubList}{\ensuremath{\mathit{SubList}}\xspace}
\newcommand{\SubListUpTo}{\ensuremath{\mathit{SubListUpTo}}\xspace}
\newcommand{\SubListUpToLevel}{\ensuremath{\mathit{SubListUpToLevel}}\xspace}
\newcommand{\OList}{\ensuremath{\mathit{OList}}\xspace}
\newcommand{\skreg}{\ensuremath{\mathit{skreg}}\xspace}
\newcommand{\locks}{\ensuremath{\mathit{locks}}\xspace}
\newcommand{\newArray}{\ensuremath{\mathit{newArray}}\xspace}
\newcommand{\randomLevel}{\ensuremath{\mathit{randomLevel}}\xspace}
\newcommand{\randomKey}{\ensuremath{\mathit{randomKey}}\xspace}
\newcommand{\randomValue}{\ensuremath{\mathit{randomValue}}\xspace}
\newcommand{\CreateNode}{\ensuremath{\mathit{CreateNode}}\xspace}
\newcommand{\maxLevel}{\ensuremath{\mathit{maxLevel}}\xspace}
\newcommand{\level}{\ensuremath{\mathit{level}}\xspace}
\newcommand{\initialLevel}{\ensuremath{\mathit{initialLevel}}\xspace}
\newcommand{\currLevel}{\ensuremath{\mathit{currLevel}}\xspace}
\newcommand{\topLevel}{\ensuremath{\mathit{topLevel}}\xspace}
\newcommand{\DisjRegions}{\ensuremath{\mathit{DisjRegions}}\xspace}
\newcommand{\DisjSkLevel}{\ensuremath{\mathit{DisjSkLevel}}\xspace}
\newcommand{\DisjSkUpToLevel}{\ensuremath{\mathit{DisjSkUpToLevel}}\xspace}
\newcommand{\DisjSkRegs}{\ensuremath{\mathit{DisjSkRegs}}\xspace}


%%%%%%%%%%%%%%%%%%%%%%%%%%%%%%%%%
%%                             %%
%%  Algorithms and code names  %%
%%                             %%
%%%%%%%%%%%%%%%%%%%%%%%%%%%%%%%%%
\newcommand{\ghcode}[1]{\ensuremath{
			\left[ \begin{array}{l}
			#1
			\end{array} \right]}
}

\newlength{\mylength}
\newenvironment{derivation} {

	\newcommand{\rint}{\text{\textbf{int}}}
	\newcommand{\rnull}{\tbf{null}}
	\newcommand{\alloc}{\text{\textbf{alloc}}}
	\newcommand{\new}{\text{\textbf{new}}}
	\newcommand{\rif}{\text{\textbf{if}}}
	\newcommand{\rthen}{\text{\textbf{then}}}
	\newcommand{\relse}{\text{\textbf{else}}}
	\newcommand{\while}[1]{\text{\textbf{while }}{##1}\text{\textbf{ do }}}
	\newcommand{\var}[2]{\text{\textbf{var }}{##1}:{##2}\text{\textbf{ in }}}
	\newcommand{\varend}{\text{\textbf{end}}}	

%% New ghost command.
	\newcommand{\gcode}[1]{\code{@ \: ##1}}

	\newcommand{\code}[1]{\item{\hspace{0.5cm}\ensuremath{##1}}}
	
	\newcommand{\step}[3]{\item \item[\ensuremath{##1}]{
		\setlength{\mylength}{\linewidth}
		\addtolength{\mylength}{-0.7cm}
			\ensuremath{\Big\{}\parbox[t]{\mylength}{\ensuremath{##2}\ensuremath{\Big\}} \\ \big[\ensuremath{##3} \big] }\par \item
	}}
	
	\newcommand{\simplestep}[2]{\item \item[\ensuremath{##1}]{
		\setlength{\mylength}{\linewidth}
		\addtolength{\mylength}{-0.7cm}
			\ensuremath{\Big\{}\parbox[t]{\mylength}{\ensuremath{##2}\ensuremath{\Big\}} }\par \item
	}}

	\begin{list}{} {
		\setlength{\itemsep}{0cm}
		\setlength{\leftmargin}{1cm}
		\addtolength{\topsep}{-0.5\topsep}
	}
	\item
}{
	\end{list}
}

%%%%%%%%%%%%%%%%%%%%%%%%%%%%%%%%%%%%%%%%%%%%%%%%%%%%%%%
%%                                                   %%
%%  Operations for program declaration               %%
%%                                                   %%
%%%%%%%%%%%%%%%%%%%%%%%%%%%%%%%%%%%%%%%%%%%%%%%%%%%%%%%
\newcommand{\New}{\tbf{new }}
\newcommand{\SkipStm}{\tbf{skip}\xspace}
\newcommand{\Nondet}{\tbf{nondet choice}\xspace}
\newcommand{\NondetEnd}{\tbf{end choice}\xspace}
\newcommand{\Noncrit}{\tbf{noncritical}\xspace}
\newcommand{\HavocQueueElem}{\tbf{havocQueueElem}\xspace}
\newcommand{\HavocListElem}{\tbf{havocListElem}\xspace}
\newcommand{\HavocSkiplistElem}{\tbf{havocSkiplistElem}\xspace}
\newcommand{\CriticalStm}{\tbf{critical}\xspace}
\newcommand{\Noncritical}{\tbf{noncritical}\xspace}
\newcommand{\Await}{\tbf{await }\xspace}
\newcommand{\Call}{\tbf{call }}
\newcommand{\ProgOr}{\tbf{    or }}
\newcommand{\PrgIf}{\tbf{if }\xspace}
\newcommand{\PrgThen}{\tbf{ then }\xspace}
\newcommand{\PrgElse}{\tbf{else }\xspace}
\newcommand{\ProcedureBegin}{\tbf{procedure}\xspace}
\newcommand{\ProcedureEnd}{\tbf{end procedure}\xspace}
\newcommand{\Free}{\ensuremath{\textrm{\tbf{free }}}\xspace}
\newcommand{\Break}{\ensuremath{\textrm{\tbf{break }}}\xspace}

\newcommand{\result}{\ensuremath{\mathit{result}}\xspace}
\newcommand{\retval}{\ensuremath{\mathit{retval}}\xspace}
\newcommand{\newval}{\ensuremath{\mathit{newval}}\xspace}
\newcommand{\removeFrom}{\ensuremath{\mathit{removeFrom}}\xspace}
\newcommand{\valueWasIn}{\ensuremath{\mathit{valueWasIn}}\xspace}
\newcommand{\valueIsIn}{\ensuremath{\mathit{valueIsIn}}\xspace}
\newcommand{\fullyLinked}{\ensuremath{\mathit{fullyLinked}}\xspace}

\newcommand{\aInit}{\ensuremath{a_{\textit{init}}}\xspace}
\newcommand{\aEnd}{\ensuremath{a_{\textit{end}}}\xspace}


%%%%%%%%%%%%%%%%%%%
%%               %%
%%  Types names  %%
%%               %%
%%%%%%%%%%%%%%%%%%%
\newcommand{\NOT}{\ensuremath{\mathit{not}}\xspace}
\newcommand{\Tvector}{\ensuremath{\mathit{Vector}}\xspace}
\newcommand{\TArray}{\ensuremath{\mathit{Array}}\xspace}
\newcommand{\Tint}{\ensuremath{\mathit{Int}}\xspace}
\newcommand{\Taddr}{\ensuremath{\mathit{Addr}}\xspace}
\newcommand{\Ttid}{\ensuremath{\mathit{Tid}}\xspace}
\newcommand{\Telem}{\ensuremath{\mathit{Elem}}\xspace}
\newcommand{\TelemSet}{\ensuremath{\mathit{ElemSet}}\xspace}
\newcommand{\Tord}{\ensuremath{\mathit{Ord}}\xspace}
\newcommand{\Tsem}{\ensuremath{\mathit{Lock}}\xspace}
\newcommand{\Tlock}{\ensuremath{\mathit{Lock}}\xspace}
\newcommand{\Tset}{\ensuremath{\mathit{Set}}\xspace}
\newcommand{\TsetInt}{\ensuremath{\Tset\langle\Tint\rangle}\xspace}
\newcommand{\intt}{\ensuremath{\mathit{Int}}\xspace}
\newcommand{\thIdT}{\ensuremath{\mathit{thId}}\xspace}
\newcommand{\bool}{\ensuremath{\mathit{Bool}}\xspace}
\newcommand{\Tbool}{\bool}


%%%%%%%%%%%%%%%%%%%%%%%%%%
%%                      %%
%%  Locks manipulation  %%
%%                      %%
%%%%%%%%%%%%%%%%%%%%%%%%%%
\newcommand{\Lock}{\ensuremath{\mathit{Lock}}\xspace}
\newcommand{\Locked}{\ensuremath{\mathit{Locked}}\xspace}
\newcommand{\semwait}{\ensuremath{\mathit{lock}}\xspace}
\newcommand{\lock}{\ensuremath{\mathit{lock}}\xspace}
\newcommand{\sempost}{\ensuremath{\mathit{unlock}}\xspace}
\newcommand{\unlock}{\ensuremath{\mathit{unlock}}\xspace}
\newcommand{\owner}{\ensuremath{\mathit{owner}}\xspace}
\newcommand{\CallLock}[1]{\ensuremath{\semwait(#1)}}
\newcommand{\CallUnlock}[1]{\ensuremath{\sempost(#1)}}
\newcommand{\CallLLock}[1]{\ensuremath{\lock(#1)}}
\newcommand{\CallLUnlock}[1]{\ensuremath{\unlock(#1)}}



%%%%%%%%%%%%%%%%%%%
%%               %%
%%  Transitions  %%
%%               %%
%%%%%%%%%%%%%%%%%%%
\newcommand{\at}[1]{\ensuremath{\mathit{at\_\ell_{#1}}}\xspace}
\newcommand{\att}[1]{\ensuremath{\mathit{at'\_\ell_{#1}}}\xspace}

\newcommand{\atth}[2]{\ensuremath{\mathit{at\_\ell^{[{#2}]}_{#1}}}\xspace}
\newcommand{\attht}[2]{\ensuremath{\mathit{at'\_\ell^{[{#2}]}_{#1}}}\xspace}

\newcommand{\atthps}[2]{\ensuremath{\mathit{at\_{#1}_{#2}}}\xspace}
\newcommand{\atthpst}[2]{\ensuremath{\mathit{at'\_{#1}_{#2}}}\xspace}

\newcommand{\atthp}[3]{\ensuremath{\mathit{at\_{#1}^{[{#3}]}_{#2}}}\xspace}
\newcommand{\atthpt}[3]{\ensuremath{\mathit{at'\_{#1}^{[{#3}]}_{#2}}}\xspace}

\newcommand{\atEnd}[1]{\ensuremath{\mathit{at\_{\mit{end}}^{[{#1}]}}}\xspace}

\renewcommand{\tr}[1]{\ensuremath{\mathit{\rho_{\ell_{#1}}}}\xspace}
\newcommand{\tset}[1]{\ensuremath{\big\{ {#1} \big\}}\xspace}
\newcommand{\tra}[1]{\ensuremath{\tau_{#1}}\xspace}
\newcommand{\tranDef}[2]{\ensuremath{\tau^{#2}_{#1}}\xspace}
\newcommand{\tran}[3]{\ensuremath{\tau^{[{#3}]}_{{#1}_{#2}}}\xspace}
\renewcommand{\trans}[2]{\ensuremath{\tau_{{#1}_{#2}}}\xspace}
\newcommand{\ttran}[2]{\ensuremath{\tau^{[#1]}_{#2}}\xspace}

\newcommand{\pt}[1]{\pi^{[#1]}}

\newcommand{\rtau}[2]{\ensuremath{\mathit{\rho_{\tau^{[#2]}_{#1}}}}\xspace}
\newcommand{\rtaust}[2]{\ensuremath{\mathit{\rho_\tau(#1,#2)}}\xspace}
\newcommand{\rptaust}[3]{\ensuremath{\mathit{\rho^{[{#1}]}_\tau(#2,#3)}}\xspace}
\newcommand{\rhotran}[3]{\ensuremath{\rho_{\tau_{#1_{#2}}^{[#3]}}}\xspace}


%%%%%%%%%%%%%%%%%%%%%%%%%
%%                     %%
%%  Skiplists symbols  %%
%%                     %%
%%%%%%%%%%%%%%%%%%%%%%%%%
% Common words used in mathematical enviroments.
\renewcommand{\inv}[2]{\ensuremath{I_{\textrm{#1}}^{[#2]}}}
\newcommand{\Value}{\ensuremath{\mathit{Value}}\xspace}
\newcommand{\Key}{\ensuremath{\mathit{Key}}\xspace}
\newcommand{\Null}{\ensuremath{\mathit{Null}}\xspace}
\newcommand{\Thread}{\ensuremath{\mathit{Thread}}\xspace}


%%%%%%%%%%%%%%%%%%%%%%%%%%
%%                      %%
%%  Schorr-Waite names  %%
%%                      %%
%%%%%%%%%%%%%%%%%%%%%%%%%%
\newcommand{\Graph}{\ensuremath{\mathit{Graph}}\xspace}
\newcommand{\Rch}{\ensuremath{\mathit{Rch}}\xspace}
\newcommand{\Rall}{\ensuremath{\mathit{R_{all}}}\xspace}
\newcommand{\Rgraph}{\ensuremath{\mathit{R_{graph}}}\xspace}
\newcommand{\PreSW}{\ensuremath{\mathit{Pre_{SSW}}}\xspace}
\newcommand{\PostSW}{\ensuremath{\mathit{Post_{SSW}}}\xspace}
\newcommand{\PreNDTS}{\ensuremath{\mathit{Pre_{NDTS}}}\xspace}
\newcommand{\PostNDTS}{\ensuremath{\mathit{Post_{NDTS}}}\xspace}
\newcommand{\PreNDSW}{\ensuremath{\mathit{Pre_{NDSW}}}\xspace}
\newcommand{\PostNDSW}{\ensuremath{\mathit{Post_{NDSW}}}\xspace}
\newcommand{\edges}{\ensuremath{\mathit{edges}}\xspace}
\newcommand{\ptr}{\ensuremath{\mathit{Ptr}}\xspace}
\newcommand{\ptrChild}{\ensuremath{\mathit{Ptr_{child}}}\xspace}
\newcommand{\MptrChild}{\ensuremath{\mathit{MPtr_{child}}}\xspace}

\newcommand{\ptrLROnly}{\ensuremath{\mathit{Ptr_{LR\_only}}}\xspace}
\newcommand{\ptrLPOnly}{\ensuremath{\mathit{Ptr_{LP\_only}}}\xspace}
\newcommand{\Mptr}{\ensuremath{\mathit{MPtr}}\xspace}

\newcommand{\Forall}[1]{\ensuremath{\textsf{forall}_{#1}}\xspace}
\newcommand{\DUnion}{\ensuremath{\oplus}\xspace}
\newcommand{\BDUnion}{\ensuremath{\bigoplus}\xspace}

\newcommand{\TID}{\ensuremath{T_{ID}}\xspace}
\newcommand{\vt}[2]{\ensuremath{{#1}^{[{#2}]}}\xspace}
\newcommand{\Addr}{\ensuremath{\mathit{Addr}}\xspace}
\newcommand{\Values}{\ensuremath{\mathit{Values}}\xspace}
\newcommand{\Vars}{\ensuremath{\mathit{Vars}}\xspace}
\newcommand{\val}{\ensuremath{\mathit{val}}\xspace}
\newcommand{\key}{\ensuremath{\mathit{key}}\xspace}
\newcommand{\visits}{\ensuremath{\mathit{visits}}\xspace}
\newcommand{\aux}{\ensuremath{\mathit{aux}}\xspace}
\newcommand{\leftp}{\ensuremath{\mathit{left}}\xspace}
\newcommand{\rightp}{\ensuremath{\mathit{right}}\xspace}
\newcommand{\oleftp}{\ensuremath{\mathit{orig\_left}}\xspace}
\newcommand{\orightp}{\ensuremath{\mathit{orig\_right}}\xspace}
\newcommand{\vroot}{\ensuremath{\mathit{vroot}}\xspace}
\newcommand{\rootnode}{\ensuremath{\mathit{root}}\xspace}

\newcommand{\info}{\ensuremath{\mathit{info}}\xspace}
\newcommand{\infostr}{\ensuremath{\mathit{info\_structure}}\xspace}
\newcommand{\Step}{\ensuremath{\mathit{step}}\xspace}
\newcommand{\theresWork}{\ensuremath{\mathit{theresWork}}\xspace}
\newcommand{\prev}{\ensuremath{\mathit{prev}}\xspace}
\newcommand{\found}{\ensuremath{\mathit{found}}\xspace}
\newcommand{\visited}{\ensuremath{\mathit{visited}}\xspace}
\newcommand{\execPoint}{\ensuremath{\mathit{execPoint}}\xspace}

\newcommand{\TheresAvailableNodes}
	{\ensuremath{\mathit{TheresAvailableNodes}}\xspace}
\newcommand{\NondetDecide}{\ensuremath{\mathit{NondetDecide}}\xspace}
\newcommand{\NondetChooseThread}{\ensuremath{\mathit{NondetChooseThread}}\xspace}
\newcommand{\NondetChooseNode}{\ensuremath{\mathit{NondetChooseNode}}\xspace}
\newcommand{\NondetChooseElem}{\ensuremath{\mathit{NondetPickElem}}\xspace}
\newcommand{\NondetSchorrWaite}
	{\ensuremath{\mathit{NondeterministicShorrWaite}}\xspace}

\newcommand{\findNode}{\ensuremath{\mathit{findNode}}\xspace}
\newcommand{\okToDelete}{\ensuremath{\mathit{okToDelete}}\xspace}

\newcommand{\pn}{\ensuremath{\pi_{n}}}

%% Some names for graphs operations
\newcommand{\rotjump}{\textsf{rotate \& jump }}
\renewcommand{\rot}{\textsf{rotate }}


%%%%%%%%%%%%%%%%%%%%%
%%                 %%
%%  Lock drawings  %%
%%                 %%
%%%%%%%%%%%%%%%%%%%%%
\newcommand{\blacklocksym}{\ensuremath{
	\stackrel{\textrm{\raisebox{-2ex}{$\cap$}}}
		{\blacksquare}}}

\newcommand{\locksym}[1]{\tiny{\stackrel{\cap}{\Box}}
	\hspace{-2.8ex}\textrm{\raisebox{0.9ex}{#1}}}


%%%%%%%%%%%%%%%%%%%%%%%%%%%%%%%%%%%%%%%
%%                                   %%
%%  Fair Transition Systems Symbols  %%
%%                                   %%
%%%%%%%%%%%%%%%%%%%%%%%%%%%%%%%%%%%%%%%
\newcommand{\FTS}{\ensuremath{\textsc{fts}}\xspace}
\newcommand{\FTSs}{\textsc{fts}s\xspace}
\newcommand{\LTL}{\textsc{ltl}\xspace}
\newcommand{\PFTSof}[1]{\ensuremath{\mathcal{F}^{[{#1}]}}\xspace}
\newcommand{\TFof}[1]{\ensuremath{\varphi{({#1})}}\xspace}
\newcommand{\VD}{\ensuremath{\textsc{vd}}\xspace}
\newcommand{\GTVD}{\ensuremath{\textsc{gtvd}}\xspace}
\newcommand{\GVD}{\ensuremath{\textsc{gvd}}\xspace}
\newcommand{\GVDs}{\textsc{gvd}s\xspace}
\newcommand{\PVD}{\ensuremath{\textsc{pvd}}\xspace}
\newcommand{\PVDs}{\textsc{pvd}s\xspace}
\newcommand{\PVDsym}{\ensuremath{\mathcal{D}}\xspace}
\newcommand{\VC}{\textsc{vc}\xspace}
\newcommand{\FA}{\ensuremath{\textsc{fa}}\xspace}
\newcommand{\Prog}{\ensuremath{\mathit{P}}\xspace}
\newcommand{\V}{\ensuremath{\mathit{V}}\xspace}
\newcommand{\vzero}[1]{\ensuremath{{#1}_{0}}\xspace}
\newcommand{\vpre}[1]{\ensuremath{{#1}_{\mathit{pre}}}\xspace}
\newcommand{\VProg}{\ensuremath{\mathit{V_{prog}}}\xspace}
\newcommand{\Vtheta}{\ensuremath{\mathit{V_{\theta}}}\xspace}
\newcommand{\VTID}{\ensuremath{\mathit{V_{\TID}}}\xspace}
\newcommand{\VTIDp}{\ensuremath{\mathit{V_{\TID}^{+}}}\xspace}
\newcommand{\VSetT}{\ensuremath{\mathit{V_{\setT}}}\xspace}
\newcommand{\tDiff}{\ensuremath{\mathit{t_{diff}}}\xspace}
\newcommand{\J}{\ensuremath{\mathcal{J}}\xspace}
\ifx\C\undefined
	\newcommand{\C}{\ensuremath{\mathcal{C}}\xspace}
\else
	\renewcommand{\C}{\ensuremath{\mathcal{C}}\xspace}
\fi
\newcommand{\T}{\ensuremath{\mathcal{T}}\xspace}
\renewcommand{\F}{\ensuremath{\mathcal{F}}\xspace}
\newcommand{\D}{\ensuremath{\mathcal{D}}\xspace}
\newcommand{\tDelta}[1]{\ensuremath{\delta_{#1}}\xspace}
\ifx\k\undefined
	\newcommand{\k}{\ensuremath{\kappa}\xspace}
\else
	\renewcommand{\k}{\ensuremath{\kappa}\xspace}
\fi
\newcommand{\bpn}{\ensuremath{\beta_n}\xspace}
\newcommand{\bpv}{\ensuremath{\beta_v}\xspace}
\newcommand{\FTSState}{\ensuremath{\mathit{state}}\xspace}


%%%%%%%%%%%%%%%%%%%%%%%%%%%%%%%%%%%%%
%%                                 %%
%%  Verification Diagrams Symbols  %%
%%                                 %%
%%%%%%%%%%%%%%%%%%%%%%%%%%%%%%%%%%%%%
\newcommand{\validVD}{\ensuremath{(\Phi, \varphi)\mit{-valid}}\xspace}
\newcommand{\validPVD}
	{\ensuremath{(\PFTSof{M}, \TFof{k})\mit{-valid}}\xspace}

\newcommand{\pres}{\ensuremath{\mathit{pres}}\xspace}
% A preserved set of variables
\newcommand{\preset}[1]{\ensuremath{\pres(V - \{ #1 \})}}
\newcommand{\En}{\ensuremath{\mathit{En}}\xspace}
\newcommand{\lang}{\ensuremath{\mathcal{L}}\xspace}

\newcommand{\infn}{\ensuremath{\mathit{inf_n}}\xspace}
\newcommand{\infe}{\ensuremath{\mathit{inf_e}}\xspace}

\newcommand{\PFTSrun}{\ensuremath{\mathcal{L}_{R}(\PFTSof{M})}\xspace}
\newcommand{\PVDrun}{\ensuremath{\mathcal{L}_{R}(\PVDsym)}\xspace}
\newcommand{\PFTScomp}{\ensuremath{\mathcal{L}(\PFTSof{M})}\xspace}
\newcommand{\PVDcomp}{\ensuremath{\mathcal{L}(\PVDsym)}\xspace}
\newcommand{\PTFcomp}{\ensuremath{\mathcal{L}(\TFof{k})}\xspace}
\newcommand{\PVDcompP}{\ensuremath{\mathcal{L}^{p}(\PVDsym)}\xspace}
\newcommand{\PTFcompP}{\ensuremath{\mathcal{L}^{p}(\TFof{k})}\xspace}

\newcommand{\FTSrun}{\ensuremath{\mathcal{L}_{R}(\Phi)}\xspace}
\newcommand{\VDrun}{\ensuremath{\mathcal{L}_{R}(\Psi)}\xspace}
\newcommand{\FTScomp}{\ensuremath{\mathcal{L}(\Phi)}\xspace}
\newcommand{\VDcomp}{\ensuremath{\mathcal{L}(\Psi)}\xspace}
\newcommand{\TFcomp}{\ensuremath{\mathcal{L}(\varphi)}\xspace}
\newcommand{\VDcompP}{\ensuremath{\mathcal{L}^{p}(\Psi)}\xspace}
\newcommand{\TFcompP}{\ensuremath{\mathcal{L}^{p}(\varphi)}\xspace}

\newcommand{\LastTerm}{\ensuremath{\psi}\xspace}
\newcommand{\LastTermK}{\ensuremath{\psi(k)}\xspace}
\newcommand{\Term}{\ensuremath{\psi}\xspace}
\newcommand{\Sys}{\ensuremath{\mathcal{S}}\xspace}
\newcommand{\SysN}{\ensuremath{\mathcal{S}[N]}\xspace}

\newcommand{\any}{\ensuremath{\mathit{any}}\xspace}

%%%%%%%%%%%%%%%%%%%%%%%%
%%                    %%
%%  Class definition  %%
%%                    %%
%%%%%%%%%%%%%%%%%%%%%%%%
\newcommand{\class}{\tbf{class}}
\newcommand{\field}[2]{\hs{1} #1 \: #2; \\}
\newcommand{\mfield}[2]{\hs{1} @ #1 \: #2; \\}
\newenvironment{classdef}[1]
	{
		\begin{center}
		\begin{math}
		\begin{array}{l}
		\class \: #1 \: \{ \\
	}
	{
		\}
		\end{array}
		\end{math}
		\end{center}
	}

\newcommand{\mysection}[1]{\paragraph{\textbf{\textup{#1}}}}

\newcommand{\myframe}[1]{
	\begin{mdframed}[roundcorner=8pt]
		\centering
		{#1}
	\end{mdframed}
}

\newenvironment{myframeverb}[1]
	{
		\begin{mdframed}[roundcorner=8pt]
		\centering
		\small
		\begin{Verbatim}
		#1
	}
	{
		\end{Verbatim}
		\normalsize
		\end{mdframed}
	}

\newcommand{\descLabel}[2]{#1) {\large$\boldsymbol{#2}$}:}
\newcommand{\descLabelThree}[2]{#1) {#2}:}

\newcommand{\phiNonNorm}{\ensuremath{\varphi_{\mathsf{nonNorm}}}\xspace}
\newcommand{\pers}[1]{\textsf{\textbf{#1}}}
\newcommand{\cyr}[1]{\fontencoding{T2A}\selectfont{#1}\fontencoding{T1}\selectfont}


\newcommand{\Always}{\square}
\newcommand{\Var}{\mathit{Var}}
\newcommand{\Premise}[1]{\ensuremath{\mathsf{#1}}\xspace}

\newcommand{\SigmaParam}{\ensuremath{\Sigma_{\textsf{param}}}\xspace}
\newcommand{\VParam}{\ensuremath{V_{\textsf{param}}}\xspace}
\newcommand{\VbaseParam}{\ensuremath{V_{\textsf{bsp}}}\xspace} 
\newcommand{\ThetaParam}{\ensuremath{\Theta_{\textsf{param}}}\xspace}
\newcommand{\TauParam}{\ensuremath{\Tau_{\textsf{param}}}\xspace}
%\newcommand{\TParam}{\ensuremath{T_{\textit{param}}}\xspace}
\newcommand{\FParam}{\ensuremath{F_{\textsf{param}}}\xspace}
\newcommand{\Param}{\ensuremath{\mathcal{P}}\xspace}
\newcommand{\ParamP}{\ensuremath{\mathcal{P}_P}\xspace}

\newcommand{\ModelA}{\mathcal{A}}
\newcommand{\ModelB}{\mathcal{B}}
\newcommand{\Img}{\textit{Img}\xspace}

\newcommand{\supports}{\mathrel{\rhd}}
\newcommand{\Into}{\mathrel{\rightarrow}} %% ce (function definition)