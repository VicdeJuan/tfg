% Clase del documento

%
% Paquetes necesarios
%

\usepackage{eurosym} 			% Símbolo del euro
\usepackage[utf8]{inputenc} 	% Codificación UTF8
\usepackage[spanish,english]{babel} 	% Caracteres del español
\usepackage{listings} 		% Código, algoritmos, etc.
\usepackage{color} 			% Definición de colores
\usepackage[table,xcdraw]{xcolor} 		% Extensión del paquete color
\usepackage{anysize} 		% Márgenes
\usepackage{fancyhdr} 		% Cabecera y pie de página
\usepackage{quotchap} 		% Estilo título capítulos
\usepackage{algorithmic} 	% Algoritmos (expresarlos mejor)
\usepackage{titlesec} 		% Títulos de secciones
\usepackage{amsmath} 		% Fórmulas matemáticas
\usepackage{enumerate} 		% Enumeraciones
\usepackage{emptypage} 		% Páginas en blanco
\usepackage{float} 			% Separación entre cajas
\usepackage{graphicx} 		% Imágenes
\usepackage{array} 			% Mejora de las tablas
\usepackage{mdwmath} 		% Mejora de los símbolos matemáticos
\usepackage[caption=false,font=footnotesize]{subfig} 		% Separar figuras en subfiguras
\usepackage{pdfpages} 		% Incluir pdfs externos
\usepackage{fancybox} 		% Mejoras sobre las cajas
\usepackage{appendix} 		% Apéndices
\usepackage{bookmark} 		% Marcadores (para el pdf)
\usepackage{enumitem} 		% Estilo de enumeraciones
\usepackage{setspace} 		% Espacio entre líneas y párrafos
\usepackage[acronym]{glossaries} 		% Glosario/Acrónimos
\usepackage[T1]{fontenc} 	% Fuentes
\usepackage[sorting=none,natbib=true,backend=bibtex,bibencoding=ascii]{biblatex} 		% Bibliografía
\usepackage{csquotes} 		% Fix biblatex+babel warning
\usepackage{exmath}
\usepackage{MathUnicode}
\usepackage{imakeidx}
\usepackage{definitions}
\usepackage{breqn}
\usepackage{hyperref}

\usepackage{babel}

\usetikzlibrary{arrows}
\tikzset{
every picture/.append style={
  execute at begin picture={\deactivatequoting},
  execute at end picture={\activatequoting}
  }
}
%\tikzset{every node/.append style={minimum size=0.5cm, draw,circle,font=\sffamily\Large\bfseries,inner sep=0.05cm}}%



%\PrerenderUnicode{ÁáÉéÍíÓóÚúÑñ} % Para que salgan las tildes y demás mierdas en el título.

% Enlaces
\hypersetup{hidelinks,pageanchor=true,colorlinks,citecolor=Fuchsia,urlcolor=black,linkcolor=Cerulean}

% Euro (€)
\DeclareUnicodeCharacter{20AC}{\euro}

% Inclusión de gráficos
\graphicspath{{./graphics/}}

% Extensiones de gráficos
\DeclareGraphicsExtensions{.pdf,.jpeg,.jpg,.png}

% Definiciones de colores (para hidelinks)
\definecolor{LightCyan}{rgb}{0,0,0}
\definecolor{Cerulean}{rgb}{0,0,0}
\definecolor{Fuchsia}{rgb}{0,0,0}

% Keywords (español e inglés)
\def\keywordsEn{\vspace{.5em}
{\textbf{\textit{Key words ---}}\,\relax%
}}
\def\endkeywordsEn{\par}

\def\keywordsEs{\vspace{.5em}
{\textbf{\textit{Palabras clave ---}}\,\relax%
}}
\def\endkeywordsEs{\par}


% Abstract (español e inglés)
\def\abstractEs{\vspace{.5em}
{\textbf{\textit{Resumen ---}}\,\relax%
}}
\def\endabstractEs{\par}

\def\abstractEn{\vspace{.5em}
{\textbf{\textit{Abstract ---}}\,\relax%
}}
\def\endabstractEn{\par}

% Estilo páginas de capítulos
\fancypagestyle{plain}{
\fancyhf{}
\fancyfoot[CO]{\footnotesize\emph{\workname}}
\fancyfoot[RO]{\thepage}
\renewcommand{\footrulewidth}{.6pt}
\renewcommand{\headrulewidth}{0.0pt}
}

% Estilo resto de páginas
\pagestyle{fancy}

% Estilo páginas impares
\fancyfoot[CO]{\footnotesize\emph{\workname}}
\fancyfoot[RO]{\thepage}
\rhead[]{\leftmark}

% Estilo páginas pares
\fancyfoot[CE]{\emph{\pieparcen}}
\fancyfoot[LE]{\thepage}
\fancyfoot[RE]{\pieparizq}
\lhead[\leftmark]{}

% Guía del pie de página
\renewcommand{\footrulewidth}{.6pt}

% Nombre de los bloques de código
\renewcommand{\lstlistingname}{Code}

% Estilo de los lstlistings
\lstset{
    frame=tb,
    breaklines=true,
    postbreak=\raisebox{0ex}[0ex][0ex]{\ensuremath{\color{gray}\hookrightarrow\space}}
}

% Definiciones de funciones para los títulos
\newlength\salto
\setlength{\salto}{3.5ex plus 1ex minus .2ex}
\newlength\resalto
\setlength{\resalto}{2.3ex plus.2ex}

% Estilo de los acrónimos

\pretolerance=2000
\tolerance=3000

% Texto índice de tablas
%\addto\captionsspanish{
%\def\tablename{Tabla}
%\def\listtablename{\'Indice de tablas}
%}


