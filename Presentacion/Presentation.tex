\documentclass[10pt,notes,compress,usetitleprogressbar,aspectratio=1610]{beamer}

\usepackage[spanish, es-tabla]{babel}
\usepackage[T1]{fontenc}
\usepackage[utf8x]{inputenc}
\usepackage{tikztools}

\definecolor{palette1}{HTML}{1B9E77}
\definecolor{palette2}{HTML}{D95F02}
\definecolor{palette3}{HTML}{7570B3}
\definecolor{palette4}{HTML}{E7298A}
\definecolor{palette5}{HTML}{66A61E}
\definecolor{palette6}{HTML}{E6AB02}
\definecolor{palette7}{HTML}{A6761D}
\definecolor{palette8}{HTML}{666666}

\usetheme{metropolis}



\title{First order proofs for concurrent programs}
\subtitle{Bachelor thesis | Double Degree Computer Science and Mathematics}
\author{Víctor de Juan Sanz}
\date{July 2016}

% \setbeameroption{hide notes}
% \setbeameroption{show notes}
% \setbeameroption{show only notes}

\begin{document}

\maketitle \note{Buenos días, me llamo Guillermo Julián Moreno y voy a presentar mi TFG sobre (leer título).}

\begin{frame}{Tabla de contenidos}
  \setbeamertemplate{section in toc}[sections numbered]
  \tableofcontents[hideallsubsections]
  \note{En primer lugar, explicaré en qué contexto se desarrolla este TFG y la motivación del trabajo. Después, explicaré cómo se ha desarrollado e implementado la solución de captura. Lo siguiente serán las pruebas de rendimiento y la validación del sistema, y por último las conclusiones que se extraen de este trabajo.}
\end{frame}

\section{Introducción y motivación} \note{
	Como decía, empezaré motivando este trabajo y explicando el contexto, que en este caso es el de las redes Ethernet de alta velocidad.
}
\end{document}
