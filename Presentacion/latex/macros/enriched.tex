%%%%% %%%%% 1. for the enriched paper %%%%%
%\newcommand{\F}{\mathsf{F}} %\newcommand{\T}{\mathsf{T}}
%\newcommand{\F}{\mathbf{false}}
%\newcommand{\T}{\mathbf{true}}
%\newcommand{\true}{\T}
\providecommand*{\Always}{} %Ale: to compile with cesar-llncs
\renewcommand{\Always}{\LTLsquare}
\providecommand*{\Event}{} %Ale: to compile with cesar-llncs
\renewcommand{\Event}{\LTLdiamond}
\providecommand*{\Next}{} %Ale: to compile with cesar-llncs
\renewcommand{\Next}{\LTLcircle}
%\newcommand{\Prev}{\LTLcircleminus}
\newcommand{\PrevNoFirst}{\LTLcircletilde}
\newcommand{\HasAlwaysBeen}{\LTLsquareminus}
\providecommand*{\Once}{} %Ale: to compile with cesar-llncs
\renewcommand{\Once}{\LTLdiamondminus}
\providecommand*{\Since}{} %Ale: to compile with cesar-llncs
\renewcommand{\Since}{\mathbin{\mathcal{S}}}
\providecommand*{\BackTo}{} %Ale: to compile with cesar-llncs
\renewcommand{\BackTo}{\mathbin{\mathcal{B}}}
\newcommand{\WeakPrev}{\LTLcircletilde}


\newcommand{\ORsym}{{\scriptstyle{\pmb{+}}}}
\newcommand{\SEQsym}{{\scriptstyle{\cdot}}}
\newcommand{\ANDsym}{{\scriptstyle{\pmb{|}}}}
\newcommand{\STARsym}{*}
\newcommand{\EXPsym}{\widehat{\;\phantom{\cdot}\;}}
\newcommand{\POWsym}{\EXPsym}
\newcommand{\DPOWsym}{\_}
\newcommand{\NEGsym}{\overline{\;\phantom{\cdot}\;}}
%\newcommand{\INF}[1]{\overset{\infty}{#1}}
\newcommand{\INF}[1]{\widehat{#1}}
%\newcommand{\ORsymInf}{\INF{\ORsym}}
\newcommand{\ORsymInf}{\oplus}
%\newcommand{\ANDsymInf}{\INF{\ANDsym}}
\newcommand{\ANDsymInf}{\oslash}
%\newcommand{\SEQsymInf}{\INF{\SEQsym}} 
\newcommand{\SEQsymInf}{\odot} 
\newcommand{\OR}{\mathrel{+}}
\newcommand{\AND}{\mathrel{|}} 
\newcommand{\SEQ}{\mathrel{\cdot}}
%\newcommand{\ORInf}{\mathrel{\INF{\scriptstyle{+}}}}
\newcommand{\ORInf}{\mathrel{\oplus}}
%\newcommand{\ANDInf}{\mathrel{\INF{\scriptstyle{|}}}}
\newcommand{\ANDInf}{\mathrel{\oslash}}
%\newcommand{\SEQInf}{\mathrel{\INF{\scriptstyle{\cdot}}}}
\newcommand{\SEQInf}{\mathrel{\odot}}

\newcommand{\uSTAR}{^\STARsym} %unary star
\newcommand{\bSTAR}{\mathrel{\STARsym}} %unary star
\newcommand{\STAR}{\uSTAR} \newcommand{\CIRC}{\mathrel{\circ}}
%\newcommand{\EnrichedRE}{E$\omega${}RE\xspace}
\newcommand{\EnrichedRE}{\ensuremath{\textup{EnRE}}\xspace}
\newcommand{\EnRE}{\EnrichedRE}
\newcommand{\VeryEnrichedRE}{kE$\omega${}RE\xspace}
%\newcommand{\Vars}{\mathcal{V}}
\newcommand{\ElemRE}{\ensuremath{\textup{ElemRE}}\xspace}
%\newcommand{\Bool}{\mathcal{B}}
\newcommand{\wRE}[1]{\ensuremath{\omega\text{RE}(#1)}}
\newcommand{\finRE}[1]{\ensuremath{\text{RE}(#1)}}
\newcommand{\RE}{\textup{RE}\xspace} % regular expressions
\newcommand{\ERE}{\textup{ERE}\xspace} % Extended Regular Expressions RE(+,*,;,|,~,[w]) 
\newcommand{\SERE}{\textup{SERE}\xspace} %Semi-Extended Regular Expressions RE(+,*,;,|, ,[w])

\newcommand{\SigmaElem}{\Sigma_\text{elem}}
\newcommand{\SigmaEnriched}{\Sigma_\text{rich}}
\newcommand{\SigmaEnrichedInf}{\Sigma^{\omega}_\text{rich}}
\newcommand{\SigmaRegexp}{\Sigma_\text{regexp}}

\newcommand{\NBW}{NBW\xspace} % non-deterministic B\"uchi automata on words 
\newcommand{\DBW}{DBW\xspace} % deterministic B\"uchi automata on words 
\newcommand{\ABW}{ABW\xspace} % alternating B\"uchi automata on words 
\newcommand{\BABW}{BABW\xspace} %
\newcommand{\ETL}{\text{ETL}\xspace} % Extended Temporal Logic ETL
\newcommand{\ETLr}{\ensuremath{\text{ETL}_r}\xspace} % ETL (regular condition) 
\newcommand{\ETLl}{\ensuremath{\text{ETL}_l\xspace}} % ETL (loop condition) 
\newcommand{\ETLf}{\ensuremath{\text{ETL}_f\xspace}}
% ETL (finite condition)
\newcommand{\ETLa}{\ensuremath{\text{ETL}_a\xspace}} % ETL(alternating condition)
\newcommand{\ETLta}{\ensuremath{\text{ETL}_2a}\xspace} % ETL (two way alternating condition) 
\newcommand{\Release}{\mathbin{\mathcal{R}}}

\newcommand{\ALL}{\ensuremath{\mathbf{all}}\xspace}

%\newcommand{\Alphabet}{\Gamma}
\newcommand{\Alphabet}{\Sigma}

\newcommand{\Eqs}{\mathcal{E}}


\newcommand{\DefOR}{\ensuremath{\;\;\;\big|\;\;\;}}
\renewcommand{\SEQsym}{\ensuremath{;}}
\renewcommand{\SEQ}{\mathrel{\SEQsym}}

\newcommand{\REOR}{+}
\newcommand{\RESEQsym}{\ensuremath{;}}
\newcommand{\RESeq}{\mathrel{\RESEQsym}}
\newcommand{\REStar}{*}
