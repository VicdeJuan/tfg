% Definiciones y constantes de estilo
% Clase del documento

%
% Paquetes necesarios
%

\usepackage{eurosym} 			% Símbolo del euro
\usepackage[utf8]{inputenc} 	% Codificación UTF8
\usepackage[spanish,english]{babel} 	% Caracteres del español
\usepackage{listings} 		% Código, algoritmos, etc.
\usepackage{color} 			% Definición de colores
\usepackage[table,xcdraw]{xcolor} 		% Extensión del paquete color
\usepackage{anysize} 		% Márgenes
\usepackage{fancyhdr} 		% Cabecera y pie de página
\usepackage{quotchap} 		% Estilo título capítulos
\usepackage{algorithmic} 	% Algoritmos (expresarlos mejor)
\usepackage{titlesec} 		% Títulos de secciones
\usepackage{amsmath} 		% Fórmulas matemáticas
\usepackage{enumerate} 		% Enumeraciones
\usepackage{emptypage} 		% Páginas en blanco
\usepackage{float} 			% Separación entre cajas
\usepackage{graphicx} 		% Imágenes
\usepackage{array} 			% Mejora de las tablas
\usepackage{mdwmath} 		% Mejora de los símbolos matemáticos
\usepackage[caption=false,font=footnotesize]{subfig} 		% Separar figuras en subfiguras
\usepackage{pdfpages} 		% Incluir pdfs externos
\usepackage{fancybox} 		% Mejoras sobre las cajas
\usepackage{appendix} 		% Apéndices
\usepackage{bookmark} 		% Marcadores (para el pdf)
\usepackage{enumitem} 		% Estilo de enumeraciones
\usepackage{setspace} 		% Espacio entre líneas y párrafos
\usepackage[acronym]{glossaries} 		% Glosario/Acrónimos
\usepackage[T1]{fontenc} 	% Fuentes
\usepackage[sorting=none,natbib=true,backend=bibtex,bibencoding=ascii]{biblatex} 		% Bibliografía
\usepackage{csquotes} 		% Fix biblatex+babel warning
\usepackage{exmath}
\usepackage{MathUnicode}
\usepackage{imakeidx}
\usepackage{definitions}
\usepackage{breqn}
\usepackage{hyperref}

\usepackage{babel}

\usetikzlibrary{arrows}
\tikzset{
every picture/.append style={
  execute at begin picture={\deactivatequoting},
  execute at end picture={\activatequoting}
  }
}
%\tikzset{every node/.append style={minimum size=0.5cm, draw,circle,font=\sffamily\Large\bfseries,inner sep=0.05cm}}%



%\PrerenderUnicode{ÁáÉéÍíÓóÚúÑñ} % Para que salgan las tildes y demás mierdas en el título.

% Enlaces
\hypersetup{hidelinks,pageanchor=true,colorlinks,citecolor=Fuchsia,urlcolor=black,linkcolor=Cerulean}

% Euro (€)
\DeclareUnicodeCharacter{20AC}{\euro}

% Inclusión de gráficos
\graphicspath{{./graphics/}}

% Extensiones de gráficos
\DeclareGraphicsExtensions{.pdf,.jpeg,.jpg,.png}

% Definiciones de colores (para hidelinks)
\definecolor{LightCyan}{rgb}{0,0,0}
\definecolor{Cerulean}{rgb}{0,0,0}
\definecolor{Fuchsia}{rgb}{0,0,0}

% Keywords (español e inglés)
\def\keywordsEn{\vspace{.5em}
{\textbf{\textit{Key words ---}}\,\relax%
}}
\def\endkeywordsEn{\par}

\def\keywordsEs{\vspace{.5em}
{\textbf{\textit{Palabras clave ---}}\,\relax%
}}
\def\endkeywordsEs{\par}


% Abstract (español e inglés)
\def\abstractEs{\vspace{.5em}
{\textbf{\textit{Resumen ---}}\,\relax%
}}
\def\endabstractEs{\par}

\def\abstractEn{\vspace{.5em}
{\textbf{\textit{Abstract ---}}\,\relax%
}}
\def\endabstractEn{\par}

% Estilo páginas de capítulos
\fancypagestyle{plain}{
\fancyhf{}
\fancyfoot[CO]{\footnotesize\emph{\workname}}
\fancyfoot[RO]{\thepage}
\renewcommand{\footrulewidth}{.6pt}
\renewcommand{\headrulewidth}{0.0pt}
}

% Estilo resto de páginas
\pagestyle{fancy}

% Estilo páginas impares
\fancyfoot[CO]{\footnotesize\emph{\workname}}
\fancyfoot[RO]{\thepage}
\rhead[]{\leftmark}

% Estilo páginas pares
\fancyfoot[CE]{\emph{\pieparcen}}
\fancyfoot[LE]{\thepage}
\fancyfoot[RE]{\pieparizq}
\lhead[\leftmark]{}

% Guía del pie de página
\renewcommand{\footrulewidth}{.6pt}

% Nombre de los bloques de código
\renewcommand{\lstlistingname}{Code}

% Estilo de los lstlistings
\lstset{
    frame=tb,
    breaklines=true,
    postbreak=\raisebox{0ex}[0ex][0ex]{\ensuremath{\color{gray}\hookrightarrow\space}}
}

% Definiciones de funciones para los títulos
\newlength\salto
\setlength{\salto}{3.5ex plus 1ex minus .2ex}
\newlength\resalto
\setlength{\resalto}{2.3ex plus.2ex}

% Estilo de los acrónimos

\pretolerance=2000
\tolerance=3000

% Texto índice de tablas
%\addto\captionsspanish{
%\def\tablename{Tabla}
%\def\listtablename{\'Indice de tablas}
%}




% Definiciones de comandos
\newcommand{\thisworkm}{bachelor thesis}
\newcommand{\thisworkM}{Bachelor thesis}
\newcommand{\nombreautor}{Víctor de Juan Sanz}
\newcommand{\nombretutor}{César Sánchez Sánchez}
\newcommand{\nombretrabajo}{Pruebas de primer orden de programas concurrentes}
\newcommand{\workname}{First order proofs for concurrent programs}
\newcommand{\fecha}{\today}
\newcommand{\grado}{Double Degree in Computer Science and Mathematics}
\newcommand{\degree}{Doble Grado en Ingeniería Informática y Matemáticas}
% Descomentar si tu trabajo tiene un ponente
\newcommand{\nombreponente}{Juan de Lara}
% Descomentar si tu trabajo está asociado a un grupo de investigación
% \newcommand{\grupoInvestigacion}{TODO: Grupo de investigación}
\newcommand{\departamento}{Computer Science}
\newcommand{\facultad}{Escuela Politécnica Superior}
\newcommand{\universidad}{Universidad Autónoma de Madrid}
\newcommand{\pieparizq}{}
\newcommand{\pieparcen}{\thisworkM}
\newcommand{\logoizq}{Logo_EPS}
\newcommand{\logoder}{Logo_UAM}
\newcommand{\correo}{vic.juan@estudiante.uam.es}

% Glosario y acrónimos
\makeglossaries
% Acrónimos

% TODO: Añadir aquí los acrónimos
% Ejemplo de acrónimo
\newacronym{FPGA}{FPGA}{Field-Programmable Gate Array}
\newacronym{FOL}{FOL}{First order logic}
\newacronym{SOL}{SOL}{Second order logic}
\newacronym{iff}{iff}{if and only if}
\newacronym{SPL}{SPL}{Simplified Programming Language}
\newacronym{VC}{VC}{Verification condition}
\newacronym{LTL}{LTL}{Linear Temporal Logic}
% Glosario

% TODO: Añadir aquí las definiciones del glosario
% Ejemplo de glosario
\newglossaryentry{bitstream}{name={bitstream},description={En este contexto se refiere al binario que configura el Hardware de la FPGA}}


% Rerefencias
\def\citeapos#1{\citetitle{#1} (\citeauthor{#1}, \citeyear{#1}) \cite{#1}}
\def\citetool#1{\citetitle{#1} \cite{#1}}
\bibliography{src/bibliography}


\setcounter{tocdepth}{3}


% Inicio del documento
\begin{document}

% Elección del idioma (español)
\selectlanguage{english}

%
% Portada
%
\pagenumbering{gobble}
%
% Portada
%

% Universidad, Facultad
\begin{titlepage}
\selectlanguage{spanish}
\begin{center}
\textbf{\begin{huge}
\universidad \\
\end{huge}}
\bigskip 
\begin{LARGE}
\facultad \\
\end{LARGE}
\end{center}

\bigskip
\bigskip

%
% Imágenes (logos) izquierdo y derecho
%
\begin{figure}[h]
	\begin{center}
		\includegraphics[scale=0.35]{\logoizq}
    \hspace{1cm}
		\includegraphics[scale=0.4]{\logoder}
	\end{center}	
\end{figure}

\bigskip
\bigskip
\bigskip

% Grado
\begin{center}
\begin{large}
\textbf{\grado}\\
\end{large}
\end{center}

\bigskip

\textbf{\begin{center}
\begin{huge}
\MakeUppercase{Trabajo de Fin de Grado}
\end{huge}
\end{center}}

\bigskipF
\bigskip

% Nombre del TFG
\begin{center}
\textbf{\begin{large}
\MakeUppercase{\nombretrabajo}\\
\end{large}}
\end{center}

% Nombre del autor
\vspace{\fill}
\begin{center}
\textbf{\nombreautor}\\
% Tutor
\textbf{Tutor: \nombretutor}\\
% Ponente, si está definido en main.tex
\ifcsname nombreponente\endcsname
\textbf{Ponente: \nombreponente}\\
\fi

\bigskip

% Fecha
\textbf{\fecha}\\
\end{center}
\end{titlepage}

% Universidad, Facultad
\begin{titlepage}
\selectlanguage{english}
\newpage
\newpage
\begin{center}
\textbf{\begin{huge}
\universidad \\
\end{huge}}
\bigskip 
\begin{LARGE}
\facultad \\
\end{LARGE}
\end{center}

\bigskip
\bigskip

%
% Imágenes (logos) izquierdo y derecho
%
\begin{figure}[h]
	\begin{center}
		\includegraphics[scale=0.35]{\logoizq}
    \hspace{1cm}
		\includegraphics[scale=0.4]{\logoder}
	\end{center}	
\end{figure}

\bigskip
\bigskip
\bigskip

% Grado
\begin{center}
\begin{large}
\textbf{\degree}\\
\end{large}
\end{center}

\bigskip

\textbf{\begin{center}
\begin{huge}
\MakeUppercase{Bachelor thesis}
\end{huge}
\end{center}}

\bigskip
\bigskip

% Nombre del TFG
\begin{center}
\textbf{\begin{large}
\MakeUppercase{\workname}\\
\end{large}}
\end{center}

% Nombre del autor
\vspace{\fill}
\begin{center}
\textbf{\nombreautor}\\
% Tutor
\textbf{Tutor: \nombretutor}\\
% Ponente, si está definido en main.tex
\ifcsname nombreponente\endcsname
\textbf{Ponente: \nombreponente}\\
\fi

\bigskip

% Fecha
\textbf{\fecha}\\
\end{center}
\end{titlepage}

% Primera página
\pagenumbering{Alph}
\thispagestyle{empty}
\par\vspace*{\fill}
\begin{flushleft}
\begin{scriptsize}
\end{scriptsize}\end{flushleft}
\newpage
\thispagestyle{empty}
\begin{center}

% Nombre del trabajo
\textbf{\begin{large}
\MakeUppercase{\nombretrabajo}\\*
\end{large}}
\vspace*{0.2cm}
\vspace{5cm}

% Nombre del autor y del tutor
\large Autor: \nombreautor \\*
\large Tutor: \nombretutor \\*
\ifcsname nombreponente\endcsname
\large Ponente: \nombreponente\\
\fi

\vfill

% Grupo de investigación, departamento, facultad, universidad y fecha
\ifcsname grupoInvestigacion\endcsname
\grupoInvestigacion \\
\fi
\departamento \\
\facultad \\
\universidad \\
\vspace{1cm}
\fecha \\

\clearpage

\end{center}
\normalsize

%  % % % % % % % % % % % % % English version

%
% Portada
%



% Primera página
\pagenumbering{Alph}
\thispagestyle{empty}
\par\vspace*{\fill}
\begin{flushleft}
\begin{scriptsize}
\end{scriptsize}\end{flushleft}
\newpage
\thispagestyle{empty}
\begin{center}

% Nombre del trabajo
\textbf{\begin{large}
\MakeUppercase{\workname}\\*
\end{large}}
\vspace*{0.2cm}
\vspace{5cm}

% Nombre del autor y del tutor
\large Author: \nombreautor \\*
\large Tutor: \nombretutor \\*
\ifcsname nombreponente\endcsname
\large Ponente: \nombreponente\\
\fi

\vfill

% Grupo de investigación, departamento, facultad, universidad y fecha
\ifcsname grupoInvestigacion\endcsname
\grupoInvestigacion \\
\fi
\departamento \\
\facultad \\
\universidad \\
\vspace{1cm}
\fecha \\

\clearpage

\end{center}
\normalsize




\hypersetup{pageanchor=true}

% Estilo de párrafo de los capítulos
\setlength{\parskip}{0.75em}
\renewcommand{\baselinestretch}{1.25}
% Interlineado simple
\spacing{1}

%
% Agradecimientos
%
\pagenumbering{Roman}
\setcounter{page}{0}
% -*- root: ../main.tex -*-
\chapter*{Appreciations}



% Cita
\begin{flushright}
\textit{Reason and understanding concern two levels of concept.}
Kurt Gödel
\end{flushright}
  

%
% Resumen
%
% Resumen en inglés
\chapter*{Abstract}


\begin{abstractEn}

%250-500 words
We study the uniform verification problem for infinite state processes. The problem consists on proving the parallel composition of an arbitrary number of processes running the same program satisfies a temporal property. The linked list theory with insert,remove and search procedure in a most general client is the selected theory to prove its validity.

Those proofs can be done using first order logic or model checking. The approach choosen by Alejandro Sánchez and César Sánchez in his \citeapos{thesisAle} was model searching. This work has been developed to complement \citeapos{thesisAle} by proving some of the theories they used with first order logic.

...
\citetool{spass} has been choosen.

\end{abstractEn}

% Palabras clave en inglés
\begin{keywordsEn}
TODO: Palabras clave en inglés, separadas por coma.
\end{keywordsEn}

% Resumen en español
\chapter*{Resumen}

\begin{abstractEs}
Abstract in Spanish.
\end{abstractEs}

% Palabras clave en español
\begin{keywordsEs}
TODO: Palabras clave en inglés, separadas por coma.
\end{keywordsEs}


%
% Glosario
%
\printglossary[title=Glossary,toctitle=Glossary]
\printglossary[title=Acronym,toctitle=Acronym,type=\acronymtype]

% Estilo de párrafo de los índices
\setlength{\parskip}{1pt}
\renewcommand{\baselinestretch}{1}

%
% Tabla de contenidos
%
\tableofcontents
\listoftables
\listoffigures
\cleardoublepage

% Estilo de párrafo de los capítulos
\setlength{\parskip}{0.75em}
\renewcommand{\baselinestretch}{1.25}
% Interlineado simple
\spacing{1}
% Numeración contenido
\pagenumbering{arabic}
\setcounter{page}{1}

%
% Introducción
%
% -*- root: ../main.tex -*-
\chapter{Introduction\label{chap:introduction}}

\paragraph{Abstract}

In this chapter we will introduce this bachelor thesis.
%
We will cover the motivation that led the development of this \thisworkm and which objectives we pursue.

We will discuss its scope, the document structure and finally we will introduce the preliminaries, describing some concepts needed during the whole bachelor thesis. 

\section{Motivation}

\label{Motivation}
Last May $\text{2}^{\text{nd}}$ a Japanese satellite was lost in space. This satellite cost \$248 million.
%
Research revealed the cause was a software error \cite{japaneseSatellite}. 
%
Last year, another software error was found in Boeing-787 aircrafts.
%
Apparently, the plane’s electrical generators fall into a failsafe mode if kept continuously powered 
%on
for 248 days. \cite{boein787}
%
Fortunately this software error was discovered without relevant economic or human consequences.
%
Between 1985 and 1987, six people died because of a software malfunction of an x-ray machine \cite{xraykill}.
%
This are just a few example that justify software verification is a very important problem.
%
One needs to be sure that the software being developed, in particular critical software, is correct.
%
One would like it to work as expected with no bugs at all.
%
There are some critical software as the ones developed for aeroplanes, spaceships, nuclear reactors which cannot have any errors while there are some other software which errors are more tolerable.


The usual approach to software reliability its testing.
% 
The normal way to verify and validate a software is running test to find, whenever is possible, all the errors.
% 
When the software is finished (or even while it is being developed) one can test it to check its correctness.
%
How can one be sure that all functionalities have been proven? Maybe some cases were missed and some bugs have not been found so the software is not correct even though it passed all the test.
%
As we said before, some systems can tolerate some level of incorrectness but there are some others which can not.


On the other hand, as a mathematician, I am very used to mathematical proofs of theorems and I know logic is a very powerful tool.
%
We wonder if it were possible to verify and validate software using those powerful tools.
%
And the answer is affirmative.
%
One can prove software correctness in a formal way.
% 
Software correctness can be proven in the same way as the Gauss theorem can be proven.
%
One just need the appropriate framework, tools and of course, knowledge.


I have found this topic very useful and we wanted to explore a very formal verification of using some logical theories.


\section{Objectives}

The goal is to prove the correctness of an implementation of a concurrent linked list. \footnote{The specification of the implementation is on \ref{def:problem}}

We achieve to prove with mathematical certain the correctness of the program. 
%
We achieve to prove that a list is always preserved as a list and that it is always ordered, regardless the number of threads executing the program.
%
Thus, there are 2 steps to prove. 
%
The first one is to prove that with just one process using the list, these properties are preserved always. 
%
The other one is with multiple processes using the same list. 
%
In particular, in a grain-lock implementation.

But to achieve any formal proof, we need some axioms as a basis.
%
We can't define absolute truth, we can just proof that something is true, according to facts we already know. 
%
We can prove some theorem, but we must use the axioms as a starting point. 
%
So to achieve the verification, we need to define the axioms for the theory of linked list.

It is essential to build a framework in which this verification can be automated, so \gls{FOL} will be used. 
%
Additionally, it is to be hoped that this \gls{FOL} proofs can be generated, stored and checked by third-parties.


\section{Scope}

% Contexto
\todo{Scope?}


\section{Document Structure}

Start by defining some necessary concepts to understand the rest of the thesis.

Once the reader is familiarized with some basic concepts, the state of the art is covered. 
%
We show some of the actual technologies even commercial products used nowadays. 



Chapter 3 includes a preliminary section of what formal verification is and when and why a program can be formally verified and validated as correct. 
% 
After that general preliminaries, the \gls{FOL} theories used and the formalism necessary to do formal verification is explained.
%
The implementation of the linked list is defined in this chapter.

Once the goal and the formalism is defined, the development can be fully understood. 
%
In Chapter 4, we describe the methodology used the tools developed or used.

Next, we show the results of the work. 
%
This chapter describes the list of axioms needed, an analysis of the \gls{FOL} proofs generated and finally a time analysis.

Finally, the work is summarized in its conclusions. 
%
Does formal verification worth the try? 


\section{Preliminaries}

%\subsection{First order logic}

\paragraph{Notation}
\label{def:notation}
We assume the usual way of representing and working with \gls{FOL}, this is
\begin{itemize}
	\item \textbf{Symbols:} $\{),(, \implies, \dimplies, \orcond , \andcond \}$
	\item \textbf{Quantifiers:} $\{\forall, \exists\}$
	\item \textbf{Constants:} $\{\true,\false\}$
	where we define $\true$ as \textit{true} and $\false$ as \textit{false}.
\end{itemize}

One could consider $\exists x(P(x))$  as an abbreviation of $\neg (\forall x(\neg P(x)))$, but for better understanding we would use both quantifiers when needed.
We could also use $(a \orcond b)$ instead of $(\neg a \implies b)$ but, again, for the better understanding those abbreviations will be used.
The same happens with $\true \equiv \neg \false$, but it is clearer when we use both symbols and not just one of them.


\paragraph{Definitions}

We are going to define some very basic concepts, needed and used during the whole \thisworkmp.


Let $X$, $Y$ be two sets of any dimension.
%
A \concept[Function]{function} denoted by $\appl{f}{X}{Y}$ is a map which takes elements from $X$ and returns an element from $Y$.
%
A \concept[Predicate]{predicate} is a boolean-valued function, i.e., $\appl{P}{X}{\{\true,\false\}}$
%
We call \concept[Arity]{arity} to the number of arguments a function or a predicate takes.

A \concept{formula} is defined recursively as it follows: Constants, $\true,\false$,predicates and functions are formulas. Let $F_1,F_2$ be two formulas. Then, $F_1\implies F_2, F_1\dimplies F_2, F_1\orcond F_2, F_1\andcond F_2$ are formulas.
%
We say a formula $F$ is \concept{satisfiable} \gls{iff} there exists a model $I$ that makes the formula true ($I \vDash F$). 
%
We say a formula $F$ is \concept{valid} \gls{iff} for all interpretations $I$, $I\vDash F$.
\label{def:validity}
This 2 concepts are very important and they are very related. $F$ is valid \gls{iff} $\neg F$ is unsatisfiable. 


A first-order \concept{theory} is defined by the following components: 
\begin{itemize}
	\item Its \textbf{signature} $\Sigma$ is a set of constants, functions and predicate symbols, where functions and predicates have a fixed arity.
	\item Its set of \textbf{axioms} $\axioms$ is a set of \gls{FOL} closed formula in which only elements from $\Sigma$ appear.
\end{itemize}


There are some important properties that a theory may have. 

A theory $\Sigma$ is \concept[Completeness]{complete} \gls{iff} for every closed $\Sigma-$formula $\sigma$ we have $(\Sigma\vDash \sigma) \text{ or } (\Sigma\neg\vDash \sigma) $

A theory $\Sigma$ is \concept[Consistency]{consistent} if there is at least one $\Sigma-$interpretation.	Equivalently, a theory $\Sigma$ is consistent if $\Sigma \not\vDash \false$

If our theory is not consistent, we can have a formal proof of every formula, so we can prove any contradiction. 
%
We could prove that some program is both correct and incorrect at the same time, which gives no information. 
%
Thus \textbf{consistency is a fundamental property} of useful theories, like the ones used in this thesis.

\label{intr:consistency}
In the other hand, completeness is very desirable, but may not be possible to achieve because of the incompleteness theorem by Gödel.
%
It is not possible to have a complete and consistent theory that includes basic arithmetical truths. 
%
One could expect that the theory needed to prove programs correctness would not be complete, because of the inclusion of basic arithmetical truths. 
%
Normally, consistency is basic while completeness is only desirable.

Another property of theories is the \textbf{decidability}. 
%
We say a theory $\Sigma$ is \concept{decidable} if $\Sigma \vDash F$ is decidable, for every $\Sigma-$formula 
where a $\Sigma-$formula $F$ is decidable if there is an \textbf{algorithm} that \textbf{always terminates} with ``yes'' if $F$ is valid in $\Sigma$ ($\Sigma$-valid) or ``no'' if $F$ is not $\Sigma-$valid.


Decidability is a stronger property than completeness. 
%
As completeness, decidability is a very desirable property but because \gls{FOL} (with no axioms) is undecidable in general, we may not have decidability in the theory we are working on.



\begin{example}
\concept{Theory\IS of equality}

\label{theory:equality}

We are going to define the theory of equality, because it is the simplest first-order theory.
%
The signature of the theory is:

\[\Sigma_e:\{=,a,b,c,...\}\]

and it's axioms are:

\begin{description}
	\item[Reflexivity:	] $\forall x. x=x$
	\item[Symmetry:	] $\forall x,y. x=y \implies y=x$
	\item[Transitivity:	] $\forall x,y,z. x=y \andcond y=z \implies x=z$
	\item[Function congruence:] For each function $f$ of arity $n$
	\[\forall \gor{x},\gor{y}. \left( \bigwedge_{i=1}^n x_i = y_i \right) \implies f(\gor{x}) = f(\gor{y})\]
	\item[Predicate congruence:]  For each predicate $P$ of arity $n$
	\[\forall \gor{x},\gor{y}. \left( \bigwedge_{i=1}^n x_i = y_i \right) \implies P(\gor{x}) \dimplies P(\gor{y})\]

	This 2 ``axioms'' are not axioms but \concept[Axiom schema]{axiom schemas}, because there is one axiom for each function $f$ or predicate $P$.
\end{description}

\gls{FOL} with equality is a decidable theory as Leopold Löwenheim proved in 1915 \cite{EqualityIsDecidable}. 
%
It is also consistent and complete.
\end{example}

%\subsection{Second-order logic} 
In \gls{SOL} the quantifiers can be used related to functions and/or predicates. This gives lot of possibilities to reason about the universe and problems but adds lot of complexity. 

In the theory of equality (\ref{theory:equality}) we could have defined axioms of congruence by:

\[\forall f \left( \forall \gor{x},\gor{y} \left( \bigwedge_{i=1}^n x_i = y_i \right) \implies f(\gor{x}) = f(\gor{y}) \right)\]
\[\forall P \left( \forall \gor{x},\gor{y} \left( \bigwedge_{i=1}^n x_i = y_i \right) \implies P(\gor{x}) \dimplies P(\gor{y}) \right)\]

This is a simpler way of writing but a more complex way of reasoning.

\gls{SOL} is a very powerful tool in order to express formulas. 
%
However, because it is not possible to reason automatically in theories of second order, we have not used any second-order logic and we do not develop it further.

%\subsection{Temporal logic} 
%\newcommand{\LTL}{\mbox{LTL}}
\newcommand{\U}{\;\mathcal{U}\;}

\begin{defn}[\gls{LTL}]
	We define \gls{LTL} inductively:

	\begin{itemize}
		\item $\phi$ is a variable, then $\phi\in \LTL$
		\item $\phi\in \LTL$, then $\square \phi,\diamondsuit \phi, \bigcirc \phi \in \LTL$
		\item $\phi,\psi\in \LTL$ then $(\phi \U \psi)\in \LTL$
	\end{itemize}

	Temporal quantifiers are explained in table \ref{tabl:ltl}.

\end{defn}


\begin{table}[hbtp]
\centering
\begin{tabular}{c|lr}
$\square \phi$ & $\phi$ has to hold \textbf{always} & \includegraphics[scale=0.5]{graphics/Ltlalways.png}\\ 
$\diamondsuit \phi$ & $\phi$ has to hold \textbf{eventually} & \includegraphics[scale=0.5]{graphics/Ltlevently.png}\\ 
$\bigcirc \phi$ & $\phi$ has to hold \textbf{next} & \includegraphics[scale=0.5]{graphics/Ltlnext.png}\\ 
$\psi\U\phi$ & $\psi$ has to hold at least \textbf{until} $\phi$ holds & \includegraphics[scale=0.5]{graphics/Ltluntil.png}\\ 
\end{tabular}
\caption{\gls{LTL} quantifiers}
\label{tabl:ltl}
\end{table}


Using \gls{LTL} one can express temporal properties such as 
%
\textit{pointer p is never null}: $\square p (\neq \mbox{null})$ or 
%
\textit{program P finishes}, equivalently, 
%
	program counter reaches the return statement at line $l_k$: $\diamondsuit\pc = l_k$.



\begin{example}
Temporal logic studies problems like the one exposed below.

\label{Collatz:conjecture}

Let $f$ be the following function.

\[
f(n) = \left\{
	\begin{array}{cc}
		\rfrac{n}{2} & \text{ if } (n\text{ mod } 2) \equiv 0\\
		3n+1 & \text{ if } (n\text{ mod } 2) \equiv 1
	\end{array}
\right.
\]

We can form a sequence applying repeatedly $f$. If we take the sequence starting in $n=13$ we get:
%
\[ 
	13 \overset{\displaystyle\;\; 3n+1\;\;}{\longrightarrow}
	40 \overset{\displaystyle\;\;\rfrac{n}{2}\;\;}{\longrightarrow}
	20 \overset{\displaystyle\;\;\rfrac{n}{2}\;\;}{\longrightarrow}
	10 \overset{\displaystyle\;\;\rfrac{n}{2}\;\;}{\longrightarrow}
	5 \overset{\displaystyle\;\; 3n+1\;\;}{\longrightarrow}
	16 \overset{\displaystyle\;\;\rfrac{n}{2}\;\;}{\longrightarrow}
	8 \overset{\displaystyle\;\;\rfrac{n}{2}\;\;}{\longrightarrow}
	4 \overset{\displaystyle\;\;\rfrac{n}{2}\;\;}{\longrightarrow}
	2 \overset{\displaystyle\;\;\rfrac{n}{2}\;\;}{\longrightarrow}
	1
\]
%
One could ask if this sequence eventually stops regardless the integer chosen to begin. This question is called \concept{Collatz conjecture} and has not been solved yet. 
%
For the moment, an infinite sequence has not been found, but the temporal logic problem has not been proven either.
%

We could write in \gls{FOL} the formula for this conjecture as $[ \forall x \exists n (\underbrace{f \circ f \circ  ... \circ f}_{n})(x) = 1]$. However, the temporal formula for this problem is: $\diamondsuit \left(f(n) = 1\right)$
\end{example}


With \gls{LTL} one can express properties a program must hold to work properly. 
%
This properties may be \concept[Safety]{safety properties} or \concept[Liveness]{liveness properties}.

\label{def::safety}
\textbf{Safety properties} are ... \todo{Defn}

\label{def::liveness}
\textbf{Liveness properties} are ... \todo{Defn}


\gls{LTL} is a much wider field but just this introduction is needed to talk about formal verification.


%
% Estado del arte
%
% -*- root: ../main.tex -*-
% -*- dic: en_GB  -*-
\chapter{State of the art\label{sec:estado_del_arte}}

\paragraph{Abstract}

Formal verification is needed in some parts of the industry and this is a fact. 
%
Some examples are covered, in order to encourage the reader about formal verification.

There are some tools that aims to offer solutions to the need of formal verification. 
%
One of the tools covered is developed by Microsoft, which shows that this topic is not irrelevant.
%
Additionally, we cover the different ways this \textit{formal verification} can be achieved.



\section{The need of formal verification}

As it was shown in the introduction \ref{Motivation} there are some systems in which software errors are totally inadmissible.
%
People developing software for critical systems need an effective way to check the correctness of the software.
%
In addition, they need some guarantee that the compiler they use really generates an executable which does \textbf{exactly} the same.
%
In order to solve this issues there are some tools which have been developed.

\subsection{Solutions}

\subsubsection{Compcert}

School of Computing in the University of Utah claims \cite{CCompilerMotivation}

\textit{We created a tool that generates random C programs, and then spent two and a half years using it to find compiler bugs. So far, we have reported more than 325 previously unknown bugs to compiler developers. Moreover, every compiler that we tested has been found to crash and also to silently generate wrong code when presented with valid inputs.}

So there is a real need of a C verified compiler and that is what \concept{Compcert} is. 
%
Comcpert can compile programs using almost all of the ISO C90 / ANSI C while it provides all the necessary tools to formally verify the software developed.
%
The project began in 03/2008 and is still under development. Last version was released in 12/2015 and it is under development.

Other compilers \ref{Dargaye-these} are in process to be formally verified.


\subsubsection{VCC}

\gls{VCC} is a tool developed by Microsoft which allows to develop Verified C Code \cite{VCC}. 
%
The aim is to offer a tool to critical developers in which the formal verification needed in critical software is integrated and is easy to use.
%
This project impulse the use of formal verification among the developers.

VCC is sound, which means that if VCC verifies a program, it really is correct, with 2 possible problems.
%
VCC is not verified itself, which means it can have bugs, so the program is correct modulo bugs in VCC itself.
%
Additionally, a verified C compiler should be used in order to have a verified executable, because if the compiler is not correct and inserts a bug, then the verified program may fail.
%
Microsoft has think of that, so that problem is solved in latest versions of VCC. 

This tool can be downloaded at \href{http://research.microsoft.com/en-us/projects/vcc/}{reseach.microsoft.com}





\section{Types of formal verification}

There are basically two different ways of formally verify a program. 
%
Both of the consist on translating the program into logic a formula and try to prove it valid \cite{ScalableTechniques}.
\todo{I didn't read the book but it looks interesting}
%
The first way is search models and the other one is try to obtain a formal proof. 
%
Model searching is easily automatizable but formal proofs can not be automatized for some logic, as \gls{SOL}.


\subsection{Model checking}

The approach of model checking is the one used in \citeapos{thesisAle}. 
%
Another tool that is based on model checking is Spin.

\subsubsection{Spin}

Spin targets the efficient verification of multi-threaded software. 
%
The tool checks the logical consistency of a specification and reports on deadlocks, race conditions, different types of incompleteness, and unwarranted assumptions about the relative speeds of processes.
%
Spin provides direct support to C code by including a tool which translate C code to \gls{PROMELA}.
%
Spin also provides a \gls{LTL} model checking system.


Two examples of inspiring applications of Spin in the last few years are the verification of the control algorithms for the new flood control barrier built near Rotterdam. 
%
The verification work was carried out by the Dutch firm CMG (Computer Management Group) in collaboration with the Formal Methods group at the University of Twente.

\subsection{Formal proof}

The other approach is to use an \gls{ATP} which allows automatic proving. 
%
With the theory and the formula, the \gls{ATP} tries to find a proof. 
%
As it was explained \doubt{in/at} \ref{intr:consistency}, we may need to use a theory which is not decidable. 
%
This makes this way hard. 
%
The \gls{ATP} may be searching for the proof for ever.

This is the approach we have taken in this bachelor thesis.
%
There are different \gls{ATP}, but in all of them, a framework must be build in order to use them.
%
Essentially, they are theorem prover with no interface difference apart from the syntax. 
%
Internally, they search the proofs in different ways.

Some examples of \gls{ATP} are Vampire\cite{vampire}, Issabelle \cite{issabelle} and \spass \cite{spass}

You build your own framework using Issabelle(https://isabelle.in.tum.de/), 

Spass (ref) or others.




%
% Diseño
%
% -*- root: ../main.tex -*-
\chapter{Design\label{chap:design}}

\paragraph{Abstract} 
%In this chapter we cover the description of the theory used. As we mentioned before is the linked list theory, but one may implement this theory in many different ways. 

%Once the theory we are working with is defined, we will expose the goals we intend to prove and which are the intermediate steps to prove the safety of this theory.


\section{Program correctness}


We are finally ready to apply this concepts to a real word problem. In this \thisworkm we apply those concepts to prove some properties of programs.
%
The remaining task is to define the framework and the conventions we use to prove formally properties of programs.

The way of proving correctness is by proving properties. 
%
There are \concept{liveness}, \concept{safety} and \concept{functional} properties. 
%
Safety properties refer, informally, to "bad things never happens". Proving \textit{ variable x is never 0 } is a safety property. 
%
Proving its validity can assure that a division by zero error does never occur. 
%
Whether a program finishes or not is a liveness property and receiving an output for a concrete input is a functional property. 

This properties are written in some logic. Liveness properties need \gls{LTL} but \gls{FOL} is enough to prove safety properties. 
%
As the properties are expressed formally in logic, it is necessary to define a formal representation of a program.


\label{def:SPL}
% -*- root: ../main.tex -*-
\subsection{Formal representation of a program}

This \gls{SPL} and its formal representation is the language chosen to write the programs to be formally verified.
%
It has been chosen by \citep{thesisAle} because its simplicity and \doubt{expressiveness} in order to write concurrent programs. 
%
Because its simplicity it is a great option to do formal verification with it.


\subsubsection{Preliminaries (Notation, definition)}

A program is a series of state changes. 
%
There is the \pc variable, which has the information of the line to execute next (\concept{Program\IS counter}). 
%
Additionally, there are the steps of the program. Each step modifies the state of the program by modifying the values of the variables.
%
A step can be easily expressed in \gls{FOL} defining the \concept{post-state} variables, which are the new values the variables will have after the execution of the line.
%
The formula gathering the information of the execution of a line, using the \pc and \pc' (the post-state \pc) and all the other variables is called \concept[Transition relation]{transition relation}.


\subsubsection{Possible instructions}

In order to express correctly the transition relation corresponding to certain line, we need to know how to translate program lines into \gls{FOL} logic.
%
As we are going to work with programs used by more than one thread, we need one program counter for each thread executing.
%
We parametrize the program counter. 
%
This is, defining the \textbf{program counter} as a \textbf{function} that given a thread, returns its program counter. 
%
We could have define as one variable per thread.
%
It would be equivalent. 


We proceed to define in general terms how to build transition relation.
%
For this definitions we use the letter $T$ to refer a thread.
\begin{description}
\item [Assignments:]
		The transition relation for a variable assignment consists of the 
		update of the program counter for the running thread and the 
		corresponding modification to the variable being assigned.
%
			\[
			\begin{array}[t]{p{8em}@{\hspace{6em}}p{\longtablesize}}
				\hline
				Statement & Transition relation \\ \hline\hline
				$\begin{array}[t]{l@{\hspace{0.3em}}c@{\hspace{1em}}l}
					l_1 & : & \mathtt{v := 2} \\
					l_2 & : & \mathtt{\cdots}
				\end{array}$
				&
				$\begin{array}[t]{ll}
					 \pc(T) = l_1 \andcond
					 \pc'(T) = l_2 \andcond
					 \mathtt{v}' = 2
				 \end{array}$ \\ 
				 \hline
			\end{array}
			\]

	\item [Conditionals:]
		We present now the two possible kinds of conditional statements in 
		SPL.
%
		In the first case, if condition $c$ does not hold, the execution 
		proceeds from the statement following the end of the conditional.
%
		\[
        \begin{array}[t]{p{8em}@{\hspace{6em}}p{\longtablesize}}
				\hline
				Statement & Transition relation \\ \hline\hline
				$\begin{array}[t]{l@{\hspace{0.3em}}c@{\hspace{1em}}l}
					\ell_1 & : & \mathtt{\textbf{if } c \textbf{ then }} \\ \\
					\ell_2 & : & \mathtt{\cdots} \\
						&& \mathtt{\vdots} \\
					\ell_n & : & \mathtt{\textbf{end if}} \\
					\ell_{n+1} & : & \mathtt{\cdots}
				\end{array}$
				&
				$\begin{array}[t]{ll}
					(\pc(T) = \ell_1 \andcond \;\;\mathtt{c} \andcond \pc'(T) = \ell_2) \; \Or \\
					(\pc(T) = \ell_1 \andcond \lnot \mathtt{c} \andcond \pc'(T) = \ell_{n+1})
				 \end{array}$ \\ \hline
			 \end{array}
		 \]
%
		In the second case, if condition $c$ does not hold, the execution 
		continues at the first statement in the \textbf{else} section of the 
		conditional statement.
%
		\[
				\begin{array}[t]{p{8em}@{\hspace{6em}}p{\longtablesize}}
				\hline
				Statement & Transition relation \\ \hline\hline
				$\begin{array}[t]{l@{\hspace{0.3em}}c@{\hspace{1em}}l}
					\ell_1 & : & \mathtt{\textbf{if } c \textbf{ then }} \\ \\
					\ell_2 & : & \mathtt{\cdots} \\
					\mathtt{\vdots} \\
					\ell_n & : & \mathtt{\textbf{else}} \\
					\ell_{n+1} & : & \mathtt{\cdots} \\
					\mathtt{\vdots} \\
					\ell_m & : & \mathtt{\textbf{end if}} \\
					\ell_{m+1} & : & \mathtt{\cdots}
				\end{array}$
				&
				$\begin{array}[t]{ll}
					(\pc(T) = \ell_1 \andcond \;\;\mathtt{c} \andcond \pc'(T) = \ell_2) \; \Or \\
					(\pc(T) = \ell_1 \andcond \lnot \mathtt{c} \andcond \pc'(T) = \ell_{n+1})
						& \text{for line } \ell_1 \\ \\ \phantom{\vdots} \\

						\pc(T) = \ell_n \andcond \pc'(T) = \ell_{m+1} & \text{for line } \ell_n
				 \end{array}$ \\ \hline
			\end{array}
		\]
%
	\item [Loops:]
		We consider the only loop statement available in SPL which executes 
	the statements in the body as long as the loop condition holds.
%
		\[
				\begin{array}[t]{p{8em}@{\hspace{6em}}p{\longtablesize}}
				\hline
				Statement & Transition relation \\ \hline\hline
				$\begin{array}[t]{l@{\hspace{0.3em}}c@{\hspace{1em}}l}
					\ell_1 & : & \mathtt{\textbf{while } c \textbf{ do }} \\
					\ell_2 & : & \mathtt{\cdots} \\
					\mathtt{\vdots} \\
					\ell_n & : & \mathtt{\textbf{end while}} \\
					\ell_{n+1} & : & \mathtt{\cdots}
				\end{array}$
				&
				$\begin{array}[t]{ll}
						(\pc(T) = \ell_1 \andcond \;\;\texttt{c} \andcond \pc'(T) = \ell_2) \; 
						\Or \\
						(\pc(T) = \ell_1 \andcond \lnot \texttt{c} \andcond \pc'(T) = 
					\ell_{n+1})
					& \text{for line $\ell_1$} \\ \\
					\pc(T) = \ell_n \andcond \pc'(T) = \ell_1 &
						\text{for line $\ell_n$}
				 \end{array}$ \\ \hline
			 \end{array}
		\]
%
			\item [Non deterministic choice:]
		The transition relation for the non-deterministic choice statement can 
		be expressed as follows:
%
		\[
			\begin{array}[t]{p{8em}@{\hspace{6em}}p{\longtablesize}}
				\hline
				Statement & Transition relation \\ \hline\hline
				$\begin{array}[t]{l@{\hspace{0.3em}}c@{\hspace{1em}}l}
					\ell_1 & : & \mathtt{\Nondet} \\
					\ell_2 & : & \mathtt{\;\;\;\;\;\; \cdots} \\
					\ell_3 & : & \mathtt{\textbf{or } \; \cdots} \\
					\mathtt{\vdots} \\
					\ell_n & : & \mathtt{\textbf{or } \; \cdots} \\
					\ell_{n+1} & : & \mathtt{\NondetEnd} \\
				\end{array}$
				&
				$\begin{array}[t]{ll}
					\pc(T) = \ell_1 \andcond
					\bigvee\limits_{i = 2..n} \pc'(T) = \ell_i
				 \end{array}$ \\ \hline
			 \end{array}
		\]
%
	\item [Lock and unlock:]
	Even though these are not SPL statements, as they will be widely used, it is necessary to define theirs transition relations.
%
		\[
			\begin{array}[t]{p{8em}@{\hspace{6em}}p{\longtablesize}}
				\hline
				Statement & Transition relation \\ \hline\hline
				$\begin{array}[t]{l@{\hspace{0.3em}}c@{\hspace{1em}}l}
					\ell_1 & : & \mathtt{\fLock(l)} \\
					\ell_2 & : & \mathtt{\cdots}
				\end{array}$
				&
				$\begin{array}[t]{ll}
					\pc(T) = \ell_1 \andcond
						\mathtt{l} = \oslash \andcond
						\mathtt{l}' = T \andcond \pc'(T) = \ell_2
				 \end{array}$ \\ \hline\hline
				$\begin{array}[t]{l@{\hspace{0.3em}}c@{\hspace{1em}}l}
					\ell_1 & : & \mathtt{\fUnlock(l)} \\
					\ell_2 & : & \mathtt{\cdots}
				\end{array}$
				&
				$\begin{array}[t]{ll}
					\pc(T) = \ell_1 \andcond
						\mathtt{l}' = \oslash \andcond \pc'(T) = \ell_2
				 \end{array}$ \\ \hline
			 \end{array}
		\]
%
\end{description}



\subsection{Partial correctness (Safety)}

A function (or the whole program) is \textbf{partially correct} if when the function's precondition is satisfied on entry, its postcondition is satisfied when the function returns (if it ever does).
%
We present the \textbf{inductive assertion method} for proving partial correctness.


Let $\tau$ be the \gls{FOL} property to study. 
%
The procedure is the following:
%
First each function is reduced to a finite set of \gls{FOL} formulae called \concept[Verification condition]{\gls{VC}}.
%
\footnote{This reduction is done with the basic reducing cases we studied in \ref{def:SPL}.}
%
We achieve to prove that $\tau$ is valid in every state of the execution.
%
\textbf{Induction} is the methodology used.
%
First, we assert $\tau$ is valid before the program starts (induction base).
%
Then, we assume $\tau$ in the precondition and prove $\tau'$ valid.



Some examples are studied next to clarify the procedure.

\begin{example}



We study the loop version of the factorial function.


\[
	\begin{array}{l@{\hspace{0.3em}}c@{\hspace{1em}}l}
	\hline
		l_1 & : & \mathtt{x := 10} \\
		l_2 & : & \mathtt{f := 1} \\
		l_3 & : & \mathtt{\textbf{while } (x\geq 1) \textbf{ do }} \\
		l_4 & : & \mathtt{\;\;f = f*x} \\
		l_5 & : & \mathtt{\;\;x=x-1} \\ 	
		l_6 & : & \mathtt{\textbf{end while}}\\
		l_7 & : & \mathtt{\cdots}\\
	\hline
	\end{array}
\]
\label{simple:example}




We achieve to proof two formulae.

\[\tau_1 \equiv (l_5 \to \mathtt{x}\geq 1) \;\; \wedge \;\; \tau_2 \equiv \mathtt{x} \geq 0\]

First, we reduce the program to its \VC.


\[
	\begin{array}{l}
		 \psi_1 \equiv\pc(T) = l_1 \andcond \pc\prime (T) = l_2 \andcond \mathtt{f\prime =f} \andcond \mathtt{x}\prime  = 10\\
		 \psi_2 \equiv\pc(T) = l_2 \andcond \pc\prime (T) = l_3 \andcond \mathtt{f}\prime  = 1 \andcond x\prime =x\\
		 \psi_3 \equiv\pc(T) = l_3 \andcond \pc\prime (T) = l_4 \andcond \mathtt{f\prime =f} \andcond x\prime \geq 1\\
		 \psi_4 \equiv\pc(T) = l_4 \andcond \pc\prime (T) = l_5 \andcond \mathtt{f}\prime  = \mathtt{f*x} \andcond \mathtt{x\prime =x}\\
		 \psi_5 \equiv\pc(T) = l_5 \andcond \pc\prime (T) = l_3 \andcond \mathtt{f\prime =f} \andcond \mathtt{x\prime =x-1}\\
		 \psi_6 \equiv\pc(T) = l_3 \andcond \pc\prime (T) = l_7 \andcond \mathtt{f\prime =f} \andcond \mathtt{x<1}
	\end{array}
\]

We need to prove, for $i=1,2$:

\[
	\left\{
		\begin{array}{lr}
			\psi_1 \andcond \tau_i \to \tau_i\prime  &
			\psi_2 \andcond \tau_i \to \tau_i\prime \\
			\psi_3 \andcond \tau_i \to \tau_i\prime  &
			\psi_4 \andcond \tau_i \to \tau_i\prime \\
			\psi_5 \andcond \tau_i \to \tau_i\prime  &
			\psi_6 \andcond \tau_i \to \tau_i\prime 
		\end{array}
	\right.
\]

\begin{center}\rule{4cm}{0.4pt}  $\tau_1$  \rule{4cm}{0.4pt}\end{center}
	
	 $\psi_1 \andcond \tau_1 \to \tau_1\prime $:
%	\begin{dmath*}[indentstep={0em}]
	\begin{equation*}
		(
			\underbrace{\pc(T) = l_1 \andcond \pc\prime (T) = l_2 \andcond \mathtt{f\prime =f} \andcond \mathtt{x}\prime  = 10}_{\psi_1} \andcond (\underbrace{\pc(T) = l_5 \to \mathtt{x}\geq 1}_{\tau_1})
		) 
				\to(\underbrace{\pc\prime (T) = l_5 \to \mathtt{x}\prime  \geq 1}_{\tau_1\prime })\\\\
	\end{equation*}
%	\end{dmath*}


	The formula is valid because $\pc(T) = l_2 \neq l_5$ thus the $\tau\prime _1$ is true.

	 $\psi_2 \andcond \tau_1 \to \tau_1\prime $:
%	\begin{dmath*}[indentstep={0em}]
	\begin{equation*}
		(
			\underbrace{\pc(T) = l_2 \andcond \pc\prime (T) = l_3 \andcond \mathtt{f}\prime  = 1 \andcond x\prime =x}_{\psi_2} \andcond (\underbrace{\pc(T) = l_5 \to \mathtt{x}\geq 1}_{\tau_1})
		) 
			\to(\underbrace{\pc\prime (T) = l_5 \to \mathtt{x}\prime  \geq 1}_{\tau_1\prime })\\\\
	\end{equation*}
%	\end{dmath*}


	The formula is valid because $\pc(T) = l_3 \neq l_5$ thus the $\tau\prime _1$ is true.

	 $\psi_3 \andcond \tau_1 \to \tau_1\prime $:
%	\begin{dmath*}[indentstep={0em}]
	\begin{equation*}
		(
			\underbrace{\pc(T) = l_3 \andcond \pc\prime (T) = l_4 \andcond \mathtt{f\prime =f} \andcond x\prime \geq 1}_{\psi_3} \andcond (\underbrace{\pc(T) = l_5 \to \mathtt{x}\geq 1}_{\tau_1})
		) 
			\to(\underbrace{\pc\prime (T) = l_5 \to \mathtt{x}\prime  \geq 1}_{\tau_1\prime })\\\\
	\end{equation*}
%	\end{dmath*}

		The formula is valid because $\pc(T) = l_4 \neq l_5$ thus the $\tau\prime _1$ is true.

	 \;$\psi_4 \andcond \tau_1 \to \tau_1\prime $: 
%	\begin{dmath*}[indentstep={0em}]
	\begin{equation*}
		(
			\underbrace{\pc(T) = l_4 \andcond \pc\prime (T) = l_5 \andcond \mathtt{f}\prime  = \mathtt{f*x} \andcond \mathtt{x\prime =x}}_{\psi_4} \andcond (\underbrace{\pc(T) = l_5 \to \mathtt{x}\geq 1}_{\tau_1})
		) 
			\to(\underbrace{\pc\prime (T) = l_5 \to \mathtt{x}\prime  \geq 1}_{\tau_1\prime })\\\\
	\end{equation*}
%	\end{dmath*}

	The formula is equivalent (applying resolution) to

%	\begin{dmath*}[indentstep={0em}]
	\begin{equation*}
		(
			\mathtt{x\prime =x} \andcond  \mathtt{x}\geq 1
		) 
		\to (\mathtt{x}\prime \geq 1)
	\end{equation*}
%	\end{dmath*}


	Which is valid because of equality congruence.

	 $\psi_5 \andcond \tau_1 \to \tau_1\prime $:
%	\begin{dmath*}[indentstep={0em}]
	\begin{equation*}
		(
			\underbrace{\pc(T) = l_5 \andcond \pc\prime (T) = l_3 \andcond \mathtt{f\prime =f} \andcond \mathtt{x\prime =x-1}}_{\psi_5} \andcond (\underbrace{\pc(T) = l_5 \to \mathtt{x}\geq 1}_{\tau_1})
		) 
			\to(\underbrace{\pc\prime (T) = l_5 \to \mathtt{x}\prime  \geq 1}_{\tau_1\prime })\\\\
	\end{equation*}
%	\end{dmath*}


	The formula is valid because $\pc(T) = l_3 \neq l_5$ thus the $\tau\prime _1$ is true.

	 $\psi_6 \andcond \tau_1 \to \tau_1\prime $:
%	\begin{dmath*}[indentstep={0em}]
	\begin{equation*}
		(
			\underbrace{\pc(T) = l_3 \andcond \pc\prime (T) = l_7 \andcond \mathtt{f\prime =f} \andcond \mathtt{x<1}}_{\psi_6} \andcond \underbrace{\pc(T) = l_5 \to \mathtt{x} \geq 1}_{\tau_1}
		) 
			\to (\underbrace{\pc\prime (T) = l_5 \to \mathtt{x}\prime  \geq 1}_{\tau_1\prime })\\\\
	\end{equation*}
%	\end{dmath*}


	The formula is valid because $\pc(T) = l_7 \neq l_5$ thus the $\tau\prime _1$ is true.


\paragraph{Conclusion:} we have proven that $\pc(T) = l_5 \to x \geq 1$. 
%
This is called an \concept[Invariant]{invariant} because it is true during all the execution. 
%
This transition has been chosen specially because it is needed in the proof of $\tau_2$.

\begin{center}\rule{4cm}{0.4pt}  $\tau_2$  \rule{4cm}{0.4pt}\end{center}

	\; $\psi_1 \andcond \tau_2 \to \tau_2\prime $:	
%	\begin{dmath*}[indentstep={0em}]
	\begin{equation*}
		(
			\underbrace{\pc(T) = l_1 \andcond \pc\prime (T) = l_2 \andcond \mathtt{f\prime =f} \andcond \mathtt{x}\prime  = 10}_{\psi_1} \andcond \underbrace{\mathtt{x} \geq 0}_{\tau_2}
		) 
				\to  \underbrace{\mathtt{x}\prime  \geq 0}_{\tau_2\prime }\\\\
	\end{equation*}
%	\end{dmath*}


	The formula is valid because $x\prime =10 \to x\prime \geq 0$.

	\; $\psi_2 \andcond \tau_2 \to \tau_2\prime $:	
%	\begin{dmath*}[indentstep={0em}]
	\begin{equation*}
		(
			\underbrace{\pc(T) = l_2 \andcond \pc\prime (T) = l_3 \andcond \mathtt{f}\prime  = 1 \andcond x\prime =x}_{\psi_2} \andcond \underbrace{\mathtt{x} \geq 0}_{\tau_2}
		) 
			\to \underbrace{\mathtt{x}\prime  \geq 0}_{\tau_2\prime }\\\\
	\end{equation*}
%	\end{dmath*}



	The formula is valid because of the congruence of equality used in  $x\prime =x \andcond x\geq 0 \to x\prime \geq 0$ 

	\; $\psi_3 \andcond \tau_2 \to \tau_2\prime $:
%	\begin{dmath*}[indentstep={0em}]
	\begin{equation*}
		(
			\underbrace{\pc(T) = l_3 \andcond \pc\prime (T) = l_4 \andcond \mathtt{f\prime =f} \andcond x\prime \geq 1}_{\psi_3} \andcond \underbrace{\mathtt{x} \geq 0}_{\tau_2}
		) 
			\to \underbrace{\mathtt{x}\prime  \geq 0}_{\tau_2\prime }\\\\
	\end{equation*}
%	\end{dmath*}


	The formula is valid because $x\prime \geq 1 \to x\prime \geq 0$.
	\; $\psi_4 \andcond \tau_2 \to \tau_2\prime $:	
%	\begin{dmath*}[indentstep={0em}]
	\begin{equation*}
		(
			\underbrace{\pc(T) = l_4 \andcond \pc\prime (T) = l_5 \andcond \mathtt{f}\prime  = \mathtt{f*x} \andcond \mathtt{x\prime =x}}_{\psi_4} \andcond \underbrace{\mathtt{x} \geq 0}_{\tau_2}
		) 
			\to \underbrace{\mathtt{x}\prime  \geq 0}_{\tau_2\prime }\\\\
	\end{equation*}
%	\end{dmath*}


	The formula is valid because of the congruence of equality used in  $x\prime =x \andcond x\geq 0 \to x\prime \geq 0$ 

	\; $\psi_5 \andcond \tau_2 \to \tau_2\prime $:	
%	\begin{dmath*}[indentstep={0em}]
	\begin{equation*}
		(
			\underbrace{\pc(T) = l_5 \andcond \pc\prime (T) = l_3 \andcond \mathtt{f\prime =f} \andcond \mathtt{x\prime =x-1}}_{\psi_5} \andcond \underbrace{\mathtt{x} \geq 0}_{\tau_2}
		) 
			\to \underbrace{\mathtt{x}\prime  \geq 0}_{\tau_2\prime }\\\\
	\end{equation*}
%	\end{dmath*}


	The formula has some more difficulty. 
	%
	Inside the loop $x$ should be greater than 1.
	%
	However, that information is not in the implication to prove.
	
	The solution is use some \concept{support}.
	%
	A support formula is a formula added to the precedent of an implication to give more information. 
	%
	This addition does not change the validity of the formula.
	%
	We could equivalently prove

	\[
		(\psi_5 \andcond \tau_1 \andcond \tau_2 \to \tau_2\prime ) rightarrow (\psi_5\andcond \tau_2\to\tau_2\prime )
	\]

	And this is exactly the solution to proof this \gls{VC}

	

%	\begin{dmath*}[indentstep={0em}]
	\begin{equation*}
		(
			\underbrace{\pc(T) = l_5 \andcond \pc\prime (T) = l_3 \andcond \mathtt{f\prime =f} \andcond \mathtt{x\prime =x-1}}_{\psi_5} \andcond \underbrace{\mathtt{x} \geq 0}_{\tau_2} \andcond \underbrace{\pc(T) = l_5 \to \mathtt{x} \geq 1}_{\tau_1}
		) 
			\to \underbrace{\mathtt{x}\prime  \geq 0}_{\tau_2\prime }\\\\
	\end{equation*}
%	\end{dmath*}


	And this formula is valid. Applying resolution we get an equivalent valid formula:

	\[
		( \mathtt{x\prime =x-1} \andcond \mathtt{x\prime }\geq 1) \to \mathtt{x} \geq 0
	\]


	\; $\psi_6 \andcond \tau_2 \to \tau_2\prime $:
%	\begin{dmath*}[indentstep={0em}]
	\begin{equation*}
		(
			\underbrace{\pc(T) = l_3 \andcond \pc\prime (T) = l_7 \andcond \mathtt{f\prime =f} \andcond \mathtt{x<1} \andcond \mathtt{x\prime =x} }_{\psi_6} \andcond \underbrace{\mathtt{x} \geq 0}_{\tau_2}
		) 
			\to \underbrace{\mathtt{x}\prime  \geq 0}_{\tau_2\prime }\\\\
	\end{equation*}
%	\end{dmath*}


	The formula has some more difficulty too.
	%
	One can think that the unique possible value of $x$ should be $0$ because of the content of the loop.
	%
	However, that information is not within the formula.
	%
	As we did before, some support is needed to prove this formula.
	%
	The support needed is $\tau_3 \equiv \pc(T) = l_3 \to \mathtt{x} \geq 0$.
	%
	The proof of this invariant is not included because it does not give new relevant information. 
	%
	Using this invariant, we have:

	

%	\begin{dmath*}[indentstep={0em}]
	\begin{equation*}
		(
			\underbrace{\pc(T) = l_3 \andcond \pc\prime (T) = l_7 \andcond \mathtt{f\prime =f} \andcond \mathtt{x<1} \andcond \mathtt{x\prime =x}}_{\psi_6} \andcond \underbrace{\mathtt{x} \geq 0}_{\tau_2} \andcond \underbrace{\pc(T) = l_3 \to \mathtt{x}\geq 0 }_{\tau_3}
		) 
			\to \underbrace{\mathtt{x}\prime  \geq 0}_{\tau_2\prime }\\\\
	\end{equation*}
%	\end{dmath*}

	
	And this formula is valid. Applying resolution we get an equivalent valid formula:

	\[
		(
			\mathtt{x\prime =x}  \andcond \mathtt{x}\geq 0 \to \mathtt{x\prime }\geq 0
		)
	\]



\end{example}




\section{Parametrized systems}

The correctness of just one thread executing a program it is an easy problem because it runs sequentiality.
%
Multiple threads executing the same program is a different and more difficult problem to solve.
%
If the number of threads executing is not bounded is another important and difficult step.
%
If the number of threads is bound, one could unroll the formula for all the threads in the problem and try to prove it. 
%
This is why we focus the unbounded case, which is the usual scenario.

We are going to study those cases. 
%
To do so, we need to parametrize the program executed by multiple threads.
%
Typically the threads would be $i$,$j$,$k_0$,$k_1$,... 
%
\subsubsection{Arbitrary number of threads}

For example, the web servers may not have a bound of the number of clients they can accept.
%
Can we prove correctness when an unbounded number of processes are using the same global variabless?

A recent research \citeapos{paperParametrizedInvariants} has proven a very important result. 
%
We will enunciate it and discuss it because it is fundamental for this work. 
%
We will not prove any of the results proven in \citep{paperParametrizedInvariants}.

Before we enunciate the theorem, we need some previous concepts.
%
We need to extend the concept of support to parametrized formulas.


\begin{defn}[Support]
  Let $\psi$, $A$ and $B$ be parametrized formulas, and let $S$ be the
  set of possible substitutions from the set of parametrized variables in $\psi$ ($\Var(\psi)$) into the set of parametrized variables of $(A\Into B)$ ($\Var(A\Into B)$).
%
  We say that $\psi$ supports $(A\Into B)$, whenever
%
  \[ \big( (\bigwedge_{\sigma\in S} \sigma(\psi)\big) \andcond A\big) \Into B \hspace{4em} \text{is valid} \]
%
  We use $\psi\supports(A\Into B)$ as a short notation for
  $\big((\bigwedge_{\sigma\in S} \sigma(\psi)) \andcond A\big) \Into B$.  
\end{defn}

Note that if $S'\subseteq S$ is a subset of the substitutions, and 
%
  \[ \big( (\bigwedge_{\sigma\in S'} \sigma(\psi)\big) \andcond A\big) \Into B \hspace{4em} \text{is valid} \]
%
then
%
  \[ \big( (\bigwedge_{\sigma\in S} \sigma(\psi)\big) \andcond A\big) \Into B \hspace{4em} \text{is also valid} \]
%
  Essentially, if one is successful proving the validity of a formula obtained by removing some of the conjuncts from the antecedent, the validity of the full formula holds.
%
  Hence, in practice, it is enough to consider only some of the partial substitutions to show that a support formula is valid.


\begin{itheorem}[Bound an arbitrary number of threads]
	Let $\varphi$ be a thread-parametrized formula, where $\overline{k}=\Var(\varphi)$. 
	%
	Let $\tau$ be a transition of $P$ and $\ThetaParam$ the initial condition.

	To show that $P$ satisfies $\Always\varphi$:
	\hspace{-1em}
	\[ 
		\begin{array}{r@{\;\;}lr@{\;}@{\;}cl@{\hspace{1em}}l}
			\Premise{S1}. & & \ThetaParam(\overline{k}) &\supports & \varphi & \\

			\Premise{S2}. & \varphi \supports & \tau^{(i)} &\Into& \varphi'  & \text{forall $\tau$ and all $i\in \overline{k}$}\\
			\Premise{S3}. & \varphi\supports & \big(\bigwedge\limits_{x\in\Var(\varphi)} j\neq x \andcond \tau^{(j)} &\Into& \varphi' \big)& \text{forall $\tau$ and one fresh $j\notin\overline{k}$}\\ \hline
			& \multicolumn{4}{c}{\hspace{3em} \Always \varphi} &
		\end{array}
	\]
\label{thm:biggest}
\end{itheorem}



Using this powerful result, we have reduced an arbitrary number of processes sharing the same variables to a finite number of threads sharing the variable. 
%
The proof of this result can be found in \citeapos{paperParametrizedInvariants}.
%
We will refer to $\Premise{S1}$ as \concept{initiation} because it depends on the initial condition.
%
$\Premise{S2}$ will be referred as \concept{self-consecution} because it depends on the execution of the one of the threads in the formula.
%
Finally, $\Premise{S3}$ will be referred as \concept{others-consecution} because it depends on the execution of threads which do not appear in the formula. 


\begin{example}
Threads sharing and array. Each thread has an index. How to assure no other thread writes in my index?
\end{example}

%%%%%%%%%%%%%%%%%%%%%%%%%%%%%%%%%%%%%%%%%%%%%%%%%%%%%%%%%%%%%%

%%%%%%%%%%%%%%%%%%%%%%%%%%%%%%%%%%%%%%%%%%%%%%%%%%%%%%%%%%%%%%

%%%%%%%%%%%%%%%%%%%%%%%%%%%%%%%%%%%%%%%%%%%%%%%%%%%%%%%%%%%%%%

%%%%%%%%%%%%%%%%%%%%%%%%%%%%%%%%%%%%%%%%%%%%%%%%%%%%%%%%%%%%%%

%%%%%%%%%%%%%%%%%%%%%%%%%%%%%%%%%%%%%%%%%%%%%%%%%%%%%%%%%%%%%%

%%%%%%%%%%%%%%%%%%%%%%%%%%%%%%%%%%%%%%%%%%%%%%%%%%%%%%%%%%%%%%

%%%%%%%%%%%%%%%%%%%%%%%%%%%%%%%%%%%%%%%%%%%%%%%%%%%%%%%%%%%%%%

\section{Linked list theory}

\subsection{Description}

In order to work with a linked list in a context with multiple thread using the same list 
there are two approach. 
%
A thread could lock the entire list, work with it and then release it. 
%
There could be some optimizations in this approach, such as a writer-reader system.
%
However, this is extremely \doubt{unefficient} although it could more secure in terms of preventing deadlocks.
%
The other approach is locking and unlocking each node of the list, so multiple threads can work simultaneously using the same list while they don't need to use the same node.
%
This approach takes us to a lock-coupling linked list.



\begin{defn}[Lock-coupling linked list]
A lock-coupling concurrent list~\cite{herlihy08art,vafeiadis06proving} is 
a concurrent data type that implements a set by maintaining in the heap an 
ordered single-linked list with non-repeating elements.
%
Each node in the list is protected by a lock which guarantees that a 
single thread can access a list node at the same time.
%
\end{defn}

The way a thread iterates over the list is the following.
%
The thread acquires the lock of the node
that it visits and after that tries to acquire the next node.
The first lock is only released the lock of the
second node has been successfully acquired.
%
This technique of protecting cells with locks (instead of protecting
the whole data-structure with a single coarse-grain lock) is known as
\concept{fine-grained locking}.


The nodes of a concurrent lock-coupling list are instances of the following 
\ListNode class:
%
\[
  \begin{array}{ll}
	  \class & \Node  \;\{\\
	  		&\begin{array}{l}
				Elem \:\;\; data; \;\\
				Addr \:\;\; next; \;\\
				Lock \:\;\; lock; \;
			\end{array}\\
		&\}
  \end{array}
\]
%
Where:
\begin{itemize}
		\item \fData: the value stored in the node, This value is used to keep 
			the list ordered.
		\item \fNext: a pointer that stores the address of the next node in 
			the list.
		\item \fLock: which contains the lock protecting the node.
\end{itemize}

We assume that the operating system provides the atomic operations \fLock 
and \fUnlock. 

\concept{Ghost variable} are variables which are not properly in the program but are added to 
achieve the verification needed.
Our implementation of concurrent lock-coupling lists has 4 global variables.
%
Two of them are global addresses \head and \tail, and the other two are ghost global variables \region and \elements:
%
\begin{center}
	\includegraphics[scale=\figscale]{graphics/_lists_classes}
\end{center}
%
The declared global program variables are:

\head, an address, which points to the first node of the list which has the lowest possible value ($-\infty$).

\tail, an address, which points to the last node of the list  which has the lowest possible value ($+\infty$).

\region, a set of addresses, which is used to keep track of the portion of the heap whose cells form the list.

\elements, a set of elements, which represents the collection of elements stored in the list.

In figure \ref{fig:listcode} we present the program we are going to use, written in \gls{SPL}, with . We can see there are three procedures, \Search, \Insert and \Remove. 

We consider \head and \tail sentinel nodes which are neither removed nor 
modified and we assume that the list is initialized with \head and \tail 
already set.
%
The set \region is initialized containing solely the addresses of \head 
and \tail.
%
Similarly, the set \elements is initialized containing only the elements 
initially stored at the nodes pointed by \head and \tail.
%
There is also a function \concept{havocListElem}() which returns a random element. 


\begin{figure}[!htbp]
		\myframe{\includegraphics[scale=0.4]{graphics/_listcode}}
		\caption{Concurrent lock-coupling lists implementation.}
		\label{fig:listcode}
\end{figure}


\subsection{TL3}

\emph{Theory of Linked Lists with Locks} \TLLpL, is the theory we use for describing linked-list heap memory layout.
%
\TLLpL is a multi-sorted first-order theory.
%
It is multi-sorted because it has multiple types for its variables (address, element,...).
%
It is a first-order theory because only variables are quantifiable, as \gls{FOL}.

In this section \TLLpL is defined with the purpose we have. 
%
A more complete and formal definition of \TLLpL can be found in \citeapos{paperAle} and \citep[6.2]{thesisAle}.

Although some functions are originally defined \citep{thesisAle} in suffix notation (\fNext field for instance), preffix-notation has been used to described the theory. 
%
The reason for this modification is to be consistent with \spass syntax.
%
Furthermore, a subset of \TLLpL has been used. 
%
In the same way \gls{FOL} can be expressed with $\neg,\vee$ but sometimes $\implies$ is included but $\dimplies$ is not,
%
few functions have not been used because they can be expressed using others functions in the theory. 
%
We proceed to describe \TLLpL.


\TLLpL is a compound of theories. The \textbf{sorts} used among this theories are: 
%
\cell (representing the nodes of the list),
%
\elem (representing elements),
%
\addr (representing address),
%
\tid (representing thread id),
%
\mem (representing the memory also called heap. It is represented as maps of \addr to \cell ),
%
\path (representing a finite sequence of address),
%
\sSetWhatever (representing a set of \tid,\addr or \elem).

For each sort, there is a theory containing its constants, functions and predicates. 
%
There is one more theory, $\Sigma_{Bridge}$ is a \emph{bridge theory} containing auxiliary
functions, for example, that allow to map paths of addresses to set of 
addresses, or to obtain the set of addresses reachable from a given 
address following a chain of \fNext fields.



\subsection{Signature}

We proceed to describe the signature of each theory, listing the sorts used and explaining its functions, predicates and constants. 
%
Every theory includes the equality theory \ref{theory:equality} 


%%%%%%%%%%%%%%%%%%%%%%%%%%%%%%%%%%%%%%%%%%%%%%%%

%					TID 

%%%%%%%%%%%%%%%%%%%%%%%%%%%%%%%%%%%%%%%%%%%%%%%%
\begin{center}\rule{4cm}{0.4pt} $\Sigma_{\tid}$ \rule{4cm}{0.4pt}\end{center}
%
The sort used is \tid. The "no-thread" value is represented with \fNoThread.
%
Apart from the equality theory, this theory does not have any other predicates or functions.


%%%%%%%%%%%%%%%%%%%%%%%%%%%%%%%%%%%%%%%%%%%%%%%%

%					ELEM

%%%%%%%%%%%%%%%%%%%%%%%%%%%%%%%%%%%%%%%%%%%%%%%%


\begin{center}\rule{4cm}{0.4pt} $\Sigma_{\elem}$ \rule{4cm}{0.4pt}\end{center}
%
The sort used is \elem. 
%
There is a total order which allows to order every set of \elem.
%
In addition, this sort is upper and lower bounded.
%
The signature is described \doubt{in/at} table \ref{table:elem_signature}.


\begin{table}[hbtp]
\centering
\begin{tabular}{rrl}
\fHighest & \elem & Maximum value an \elem can take.\\
\fLowest & \elem & Minimum value an \elem can take.\\
\hline\hline
\fLselem & \elem$\times$\elem & Total order relation between \elem.
\end{tabular}
\caption{\textbf{Signature of $\Sigma_{\ensuremath{\mathit{elem}}}$.} Top block contains functions, lower block contains predicates.}
\label{table:elem_signature}
\end{table}


%%%%%%%%%%%%%%%%%%%%%%%%%%%%%%%%%%%%%%%%%%%%%%%%

%					CELL

%%%%%%%%%%%%%%%%%%%%%%%%%%%%%%%%%%%%%%%%%%%%%%%%

\begin{center}\rule{4cm}{0.4pt} $\Sigma_{\cell}$ \rule{4cm}{0.4pt}\end{center}
%
The sorts used are \cell,\elem,\addr,\tid.
%
The signature is described \doubt{in/at} table \ref{table:cell_signature}.

\begin{table}[hbtp]
\begin{tabular}{rrl}
\fMkcell & $\elem\times\addr\times\tid \to \cell$ & Constructor\\
\fNext & $\cell \to \addr$ & Getter of \fNext field \\ 
\fData & $\cell \to \elem$ & Getter of \fData field \\ 
\fLockID & $\cell \to \tid$ & Getter of \fLockID field \\ 
\fLock & $\cell\times\tid\to\cell$ & Construct a new \cell with \fData and \fNext \\
&&\;\;\;								values of the given \cell, \\
&&\;\;\;				using the \tid for the \fLockID field.\\
\fError & $\cell$ & Constant value used to model \\ 
&&\;\;\;				incorrect memory deference.
\end{tabular}
\caption{\textbf{Functions of $\Sigma_{\cell}$} theory.}
\label{table:cell_signature}
\end{table}

The function \fUnlock could be considered. Actually, \cite{thesisAle} includes it in the theory but it has not been included in this work.
%
The reason is justified because to \fUnlock a \cell is equivalent to \fLock a \cell with \fNoThread value.



%%%%%%%%%%%%%%%%%%%%%%%%%%%%%%%%%%%%%%%%%%%%%%%%

%					MEMORY

%%%%%%%%%%%%%%%%%%%%%%%%%%%%%%%%%%%%%%%%%%%%%%%%


\begin{center}\rule{4cm}{0.4pt} $\Sigma_{\mem}$ \rule{4cm}{0.4pt}\end{center}
%
The sorts used are \mem,\cell and \addr. 
%
The functions are described \doubt{in/at} table \ref{table:memory_signature}.

\begin{table}[hbtp]
\begin{tabular}{rrl}
\fNull & $\addr$ & Null address \\
\fRd & $\mem\times\addr\to\cell$ & Models memory deference. \\
&&								\;\;\; Returns the value from the \mem the \cell \\
&&								\;\;\; stored in the \addr.\\
\fUpd & $\mem\times\addr\times\cell\to\mem$ & Creates a new \mem from the given one
\end{tabular}
\caption{\textbf{Functions of $\Sigma_{\mem}$} theory}
\label{table:memory_signature}
\end{table}

%%%%%%%%%%%%%%%%%%%%%%%%%%%%%%%%%%%%%%%%%%%%%%%%

%					SETADDR

%%%%%%%%%%%%%%%%%%%%%%%%%%%%%%%%%%%%%%%%%%%%%%%%


\begin{center}\rule{4cm}{0.4pt} $\Sigma_{\sSetAddr}$ \rule{4cm}{0.4pt}\end{center}
%
It models the usual set theory. 
%
Preffix version of each function and predicate has been preferred to be consistent with \ref{ax::fulllist}.
%
The signature is described \doubt{in/at} table \ref{table:setaddr_signature}.

Intersection function and subset predicate has not been included despite \citep{thesisAle} uses them. 
%
They were not used because they were unnecessary.

\begin{table}[hbtp]
\begin{tabular}{rrl}
\fEmptyset & \sSetAddr & Empty set\\
\fSingl & $\addr\to\sSetAddr $& Constructor of a single-element set.\\
\fUnion & $\sSetAddr\times\sSetAddr\to\sSetAddr$&\\
\fSetdiff & $\sSetAddr\times\sSetAddr\to\sSetAddr$&\\
\hline\hline
\pIn & $\sAddr\times\sSetAddr $& \\
\end{tabular}
\caption{\textbf{Signature of $\Sigma_{\sSetAddr}$.} Top block contains functions, lower block contains predicates.}
\label{table:setaddr_signature}
\end{table}


%%%%%%%%%%%%%%%%%%%%%%%%%%%%%%%%%%%%%%%%%%%%%%%%

%					SETELEM

%%%%%%%%%%%%%%%%%%%%%%%%%%%%%%%%%%%%%%%%%%%%%%%%


\begin{center}\rule{4cm}{0.4pt} $\Sigma_{\sSetElem}$ \rule{4cm}{0.4pt}\end{center}
%

Again, it models the usual set theory.
%
The signature is described \doubt{in/at} table \ref{table:setelem_signature}.

\begin{table}[hbtp]
\begin{tabular}{rrl}
\fEmptysetElem & $\sSet $& Empty set\\
\fSinglElem & $\elem\to\sSet $& Constructor of a single-element set.\\
\fUnionElem & $\sSet\times\sSet\to\sSet$&\\
\fSetdiffElem & $\sSet\times\sSet\to\sSet$&\\
\hline\hline
\pInElem & $\sAddr\times\sSet $& \\
\end{tabular}
\caption{\textbf{Signature of $\Sigma_{\sSet}$.} Top block contains functions, lower block contains predicates.}
\label{table:setelem_signature}
\end{table}


%%%%%%%%%%%%%%%%%%%%%%%%%%%%%%%%%%%%%%%%%%%%%%%%

%					SETTID

%%%%%%%%%%%%%%%%%%%%%%%%%%%%%%%%%%%%%%%%%%%%%%%%


\begin{center}\rule{4cm}{0.4pt} $\Sigma_{\sSetTid}$ \rule{4cm}{0.4pt}\end{center}
%


The signature is described \doubt{in/at} table \ref{table:settid_signature}.

\begin{table}[hbtp]
\begin{tabular}{rrl}
\fEmptysetTid & $\sSetAddr $& Empty set\\
\fSinglTid & $\addr\to\sSetAddr $& Constructor of a single-element set.\\
\fUnionTid & $\sSetAddr\times\sSetAddr\to\sSetAddr$&\\
\fSetdiffTid & $\sSetAddr\times\sSetAddr\to\sSetAddr$&\\
\hline\hline
\pInTid & $\sAddr\times\sSetAddr$ &\\
\end{tabular}
\caption{\textbf{Signature of $\Sigma_{\sSetAddr}$.} Top block contains functions, lower block contains predicates.}
\label{table:settid_signature}
\end{table}



%
% Desarrollo
%
% -*- root: ../main.tex -*-
\chapter{Development\label{sec:develpment}}

\paragraph{Abstract} In this chapter, the practical work will be exposed. Until now, just the theoretical fundamentals of the subject have been defined.

Here we will expose the rigorous methodology used to mathematically prove our goals. To achieve that, some tools have been developed, and they will be described too.

Finally, we will cover the axioms we needed to prove all the intermediate and the final steps. The completeness of this theory will be discussed at the end of the chapter. 

\section{Methodology used}



\subsection{Tools}

\subsubsection{Spass}

\spass (\citeapos{spass}) is an automated theorem prover for first-order logic with equality. 
%
\spass receives a \gls{FOL} formula to prove. 
Running SPASS on such a formula results in the final output \textit{SPASS beiseite: Proof found.} if the formula is valid or  \textit{SPASS beiseite: Completion found.} if the formula is not valid.
Because validity in first-order logic is undecidable, SPASS may run forever without producing any final result.
%
This last comment is a very important issue because some proofs has taken hours and one could not know if \spass will eventually stop or run forever.
%
As a curiosity, the longest time \spass was left running was 82 hours and it stopped because a proof was found.

\todo{Spass example structure file}

\subsubsection{Leap}

\leap is a tool for the verification of concurrent datatypes and parametrized systems composed by an unbounded number of threads that manipulate infinite data.

Leap receives as input a concurrent program description and a specification and automatically generates a finite set of verification conditions which are then discharged to specialized decision procedures. 
%
The validity of all discharged verification conditions implies that the program executed by any number of threads satisfies the specification. 
%
Currently, Leap includes not only decision procedures for integers and Booleans, but it also implements specific theories for heap memory layouts such as linked-lists and skiplists.

\leap uses decision procedures, based in model searching while \spass does automatic reasoning. 



\subsubsection{\gandalf}

\gandalf is the implemented tool to prove \gls{VC} with \spass. 
%
The process of converting the \leap-generated verification condition to \spass syntax, splitting different conjunctions in different files, calling \spass to try to prove and process and storing the results has been automatized basically in python combined with bash.
Plus, \spass reads prefix syntax while leap has prefix and infix syntax.

Additionally, \gandalf does some reduction of the formulas so \spass can finish in a reasonable time.
%
This is needed because \spass does not have information nor tactics about which axiom should be used first.



\subsubsection{Process}

Once the tools have been described, we can explain the process. 
%
The process followed is shown in the figure \ref{fig:process}

\todo{Figure Process}

The first step consists on translate the program given to its \gls{VC}. 
%
This is done by \leap. 
%
In the case of this problem, \leap generates at least 2 \gls{VC} for each transition of the program. 
%
The first \gls{VC} corresponds to the self-consecution \gls{VC} and the second corresponds to the others-consecution \gls{VC}. The initial transition is also generated.
%
This is done 6 times, one for each existing invariant of the problem (\ref{invariants}).

Once the \gls{VC} have been generated, the goal is to prove all of them using \spass. 
%
As \spass syntax (\cite{spasssyntax}) it is not \leap syntax (\cite{leapsyntax}) some parsing is needed. 
%
For example, \spass uses prefix notation while \leap uses infix notation for binary functions.
%
In order to solve this, it was necessary to learn Ocaml using \citeapos{ocamlbook}.
%
\leap was forked so it could write the \gls{VC} in prefix notation.
%
Some other \leap functionalities had to be changed to make compatible \leap output with \spass input.

In addition, one \spass problem has to be created for each \gls{VC}. 
%
The axiom list for each problem is determined as an argument. 
%
This is explained further in section \ref{sec:axiomgraph} inside chapter \ref{chap:results}.

Because of \spass lack of tactics and the size of the axiom list, some very easy transitions could take too much time.
%
Transitions which can be proven using \pc reasoning \spass could take minutes. 
%
In order to improve \spass performance, the problem is divided.

Let $\varphi$ be the \gls{VC} to prove. For the explanation, 
	\[
		\varphi \equiv \pc(i) = l_j \to \head \not = \tail
	\]
While proving the \gls{VC} for the thread $i$, $\pc(i)$ has a value which may be $l_j$ or not. 
%
In both cases, resolution can be applied to prove equivalently
	\[
		\psi \equiv \head \not = \tail
	\]
This resolution can be applied in self-consecution and in others-consecution. 
%
This resolution step is done by \gandalf. 
%
Some other obvious resolution is done, such as $\top \andcond \top \equiv \top$ and other tautologies.

Now \spass can prove a simpler problem and it takes \doubt{little time}.
%
To assure this is equivalent, another \spass problem is generated. 
%
This \spass problem aims to prove $\psi \to \varphi$.
%
In order to prove $\varphi$, modus ponens is used:
\[
	\begin{array}{c}
		\psi\\
		\psi \to \varphi\\\hline
		\varphi
	\end{array}
\]

Despite 2 \spass problems have to be solved instead of 1, in \ref{sec:timeanalysis} the reader can see the benefits of applying this \doubt{workaround}.
%
And this is a logic consequence. $\varphi$ is a complex problem for \spass. 
%
However, $\psi$ is a much simpler problem because it does not include any reasoning about the program counter. 
% How to express:
%%%%    I didn't check that is tautologic in every transition, but i am pretty sure it is.
In addition, the problem $\psi \to \varphi$ is tautologic in most of the transitions because it adds conjectures in the precedent of an implication, which can't make invalid a valid formula.
%
Plus, the list of axioms needed to prove this problem is small in compare to the axiom list needed to prove the original problem.

$\psi$ \spass problem should not have $\pc$ involved. How can the $\pc$ be removed while proving others-consecution?
%
Again, more \spass problems are generated. 
%
In this case, 55 new problems are generated and modus ponens is used:
\[
	\begin{array}{c}
		\psi_1\\
		\psi_2\\
		\vdots\\
		\psi_{55}\\
		(\bigwedge_i \psi_i) \to \varphi\\\hline
		\varphi
	\end{array}
\]
%%%%    I didn't check that is tautologic in every transition, but i am pretty sure it is.
In this case, the \spass problem $(\bigwedge_i \psi_i) \to \varphi$ is tautologic most of the time, but \spass takes much more time than before because of the problem size.




\subsubsection{Ocaml - parser}
\todo{Proof axioms with leap}
In order to improve Ocaml skills needed, a parser was implemented. 
%
This parser translates from \spass or \leap syntax to human syntax. 
%
The idea of this parser came at the 5th time the list of axioms seemed finished. 
%
The axiom list needed to be supervised because it is the basis of all the problem. 
%%Rewrite





\section{Linked list}
\label{proof:Preserve}

With everything exposed before we have been able to prove the invariance of \invPreserve invariant.
%
\todo{Difficult to write}


\subsection{Axioms}

The set of relevant axioms needed to prove all the invariants is presented in this part.
%
There are a lot of secondary axioms needed by \spass that have been omitted. 
%
The omitted axioms refer to the sorting (\spass is not multi-sorted as the theory we work with), to constants (\spass does not include arithmetic, so $0$,$1$,... must be defined as unique 0-ary functions). 
%
The set of axioms is:
		
\begin{description}
\item[union-def] 
\label{ax::union_def}

\begin{dmath}
((in(x,se) \vee in(x,se2)) \dimplies in(x,union(se,se2)))
\end{dmath}

\textcolor{red}{Auto-generated}
\item[Intr-def] 
\label{ax::Intr_def}

\begin{dmath}
((in(x,se) \wedge in(x,se2)) \dimplies in(x,intr(se,se2)))
\end{dmath}

\textcolor{red}{Auto-generated}
\item[SetDiff-def] 
\label{ax::SetDiff_def}

\begin{dmath}
((in(x,se) \wedge (\neg  (in(x,se2)))) \dimplies in(x,diff(se,se2)))
\end{dmath}

\textcolor{red}{Auto-generated}
\item[in-set--def] 
\label{ax::in_set__def}

\begin{dmath}
((\neg  (in(a,se))) \implies in(b,se) \implies (\neg  (b = a)))
\end{dmath}

\textcolor{red}{Auto-generated}
\item[a--in--singl-a] 
\label{ax::a__in__singl_a}

\begin{dmath}
(((\neg  (a = b)) \implies (\neg  (in(b,{ a })))) \wedge in(a,{ a }))
\end{dmath}

\textcolor{red}{Auto-generated}
\item[a-not--in-se-dif-a] 
\label{ax::a_not__in_se_dif_a}

\begin{dmath}
(in(a,se) \implies (\neg  (in(a,diff(se,{ a })))))
\end{dmath}

\textcolor{red}{Auto-generated}
\item[nextreg] 
\label{ax::nextreg}

\begin{dmath}
((in(a,se) \wedge se = addr2set(m,b) \wedge c = rd(m,a).next \wedge (\neg  (a = null))) \implies in(c,se))
\end{dmath}

\textcolor{red}{Auto-generated}
\item[data--def] 
\label{ax::data__def}

\begin{dmath}
mkcell(e,a,t).data = e
\end{dmath}

\textcolor{red}{Auto-generated}
\item[next--def] 
\label{ax::next__def}

\begin{dmath}
mkcell(e,a,t).next = a
\end{dmath}

\textcolor{red}{Auto-generated}
\item[lockid--def] 
\label{ax::lockid__def}

\begin{dmath}
mkcell(e,a,t).lockid = t
\end{dmath}

\textcolor{red}{Auto-generated}
\item[next-error--is--null] 
\label{ax::next_error__is__null}

\begin{dmath}
error.next = null
\end{dmath}

\textcolor{red}{Auto-generated}
\item[equality-bt-cell] 
\label{ax::equality_bt_cell}

\begin{dmath}
(c1 = c2 \implies (c1.data = c2.data \wedge c1.lockid = c2.lockid \wedge c1.next = c2.next))
\end{dmath}

\textcolor{red}{Auto-generated}
\item[upd--def--not-null] 
\label{ax::upd__def__not_null}

\begin{dmath}
((\neg  (a = null)) \implies upd(m,a,c) = m2 \implies rd(m2,a) = c)
\end{dmath}

\textcolor{red}{Auto-generated}
\item[upd--def--null] 
\label{ax::upd__def__null}

\begin{dmath}
(a = null \implies upd(m,a,c) = m2 \implies m2 = m)
\end{dmath}

\textcolor{red}{Auto-generated}
\item[upd--def--one-at-the-time] 
\label{ax::upd__def__one_at_the_time}

\begin{dmath}
(((\neg  (a = null)) \wedge (\neg  (a = b))) \implies upd(m,a,c) = m2 \implies rd(m,b) = rd(m2,b))
\end{dmath}

\textcolor{red}{Auto-generated}
\item[rd-mem--def] 
\label{ax::rd_mem__def}

\begin{dmath}
rd(m,null) = error
\end{dmath}

\textcolor{red}{Auto-generated}
\item[lowest--less-than-highest] 
\label{ax::lowest__less_than_highest}

\begin{dmath}
(\neg  (lowestElem = highestElem))
\end{dmath}

\textcolor{red}{Auto-generated}
\item[less-total] 
\label{ax::less_total}

\begin{dmath}
(\neg  ((x < y \wedge y < x)))
\end{dmath}

\textcolor{red}{Auto-generated}
\item[ls-xy--not-ls-yx] 
\label{ax::ls_xy__not_ls_yx}

\begin{dmath}
(x < y \implies (\neg  (y < x)))
\end{dmath}

\textcolor{red}{Auto-generated}
\item[addr2set--def] 
\label{ax::addr2set__def}

\begin{dmath}
(in(b,addr2set(m,a)) \dimplies reach(m,a,b,getp(m,a,b)))
\end{dmath}

\textcolor{red}{Auto-generated}
\item[addr2set-null--is--singl-null] 
\label{ax::addr2set_null__is__singl_null}

\begin{dmath}
addr2set(m,null) = { null }
\end{dmath}

\textcolor{red}{Auto-generated}
\item[if--addr2set--then--reach-getp] 
\label{ax::if__addr2set__then__reach_getp}

\begin{dmath}
(se = addr2set(m,a) \implies (in(b,se) \dimplies reach(m,a,b,getp(m,a,b))))
\end{dmath}

\textcolor{red}{Auto-generated}
\item[reach--a-a-epsilon--true] 
\label{ax::reach__a_a_epsilon__true}

\begin{dmath}
reach(m,a,a,epsilon)
\end{dmath}

\textcolor{red}{Auto-generated}
\item[getp--same-addr--i-epsilon] 
\label{ax::getp__same_addr__i_epsilon}

\begin{dmath}
getp(m,a,a) = epsilon
\end{dmath}

\textcolor{red}{Auto-generated}
\item[getp--recursive--base] 
\label{ax::getp__recursive__base}

\begin{dmath}
((\neg  (a = rd(m,a).next)) \implies getp(m,a,rd(m,a).next) = [ a ])
\end{dmath}

\textcolor{red}{Auto-generated}
\item[getp--recursive--step] 
\label{ax::getp__recursive__step}

\begin{dmath}
(((\neg  (getp(m,a,b) = epsilon)) \wedge (\neg  (a = b)) \wedge (\neg  (rd(m,a).next = b))) \implies append([ a ],getp(m,rd(m,a).next,b),getp(m,a,b)))
\end{dmath}

\textcolor{red}{Auto-generated}
\item[path2set-epsilon--is--empty] 
\label{ax::path2set_epsilon__is__empty}

\begin{dmath}
path2set(epsilon) = empty
\end{dmath}

\textcolor{red}{Auto-generated}
\item[path2set-recursive--def--base] 
\label{ax::path2set_recursive__def__base}

\begin{dmath}
path2set([ a ]) = { a }
\end{dmath}

\textcolor{red}{Auto-generated}
\item[path2set-recursive--def--step] 
\label{ax::path2set_recursive__def__step}

\begin{dmath}
(\neg  (getp(m,a,b) = epsilon)) \implies (path2set(getp(m,a,b)) = union({ a },path2set(getp(m,rd(m,a).next,b))))
\end{dmath}

\textcolor{red}{Auto-generated}

\end{description}



%
% Resultados
%
% -*- root: ../main.tex -*-
\chapter{Results\label{chap:results}}

\paragraph{Abstract} In this chapter we will expose the structure of the generated proofs. We will mention some of the most relevant and difficult intermediate steps that we needed to prove.

Finally, the we will compare the timings of using automatic theorem provers versus model searching done by \gls{leap}.

\section{Analysis of generated proofs}

\subsection{Special transitions}

\section{Time analysis}

\subsection{Proof generation}

\subsection{Proof checking}


\subsection{Leap times}




%
% Conclusiones
%
\chapter{Conclusions\label{sec:conclusiones}}

\section{Linked list theory axiomatizable}

\section{Refutation worth it}

\subsection{Spass Time compared to leap}

 



%
% Página en blanco
%
\cleardoublepage

%
% Bibliografía
%
\printbibliography[heading=bibintoc]

% No expandir elementos para llenar toda la página
\raggedbottom

%
% Apéndices
%
\appendix
\cleardoublepage
\addappheadtotoc
\appendixpage

%
% TODO: Apéndices del TFG
%
\include{src/appendixExample}

% Fin del documento


\end{document}
