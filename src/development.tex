% -*- root: ../main.tex -*-
\chapter{Development\label{sec:develpment}}

\paragraph{Abstract} In this chapter, the practical work will be exposed. Until now, just the theoretical fundamentals of the subject have been defined.

Here we will expose the rigorous methodology used to mathematically prove our goals. To achieve that, some tools have been developed, and they will be described too.

Finally, we will cover the axioms we needed to prove all the intermediate and the final steps. The completeness of this theory will be discussed at the end of the chapter. 

\section{Methodology used}



\subsection{Tools}

\subsubsection{Spass}

\spass (\citeapos{spass}) is an automated theorem prover for first-order logic with equality. 
%
\spass receives a \gls{FOL} formula to prove. 
Running SPASS on such a formula results in the final output \textit{SPASS beiseite: Proof found.} if the formula is valid or  \textit{SPASS beiseite: Completion found.} if the formula is not valid.
Because validity in first-order logic is undecidable, SPASS may run forever without producing any final result.
%
This last comment is a very important issue because some proofs has taken hours and one could not know if \spass will eventually stop or run forever.
%
As a curiosity, the longest time \spass was left running was 82 hours and it stopped because a proof was found.

\todo{Spass example structure file}

\subsubsection{Leap}

\leap is a tool for the verification of concurrent datatypes and parametrized systems composed by an unbounded number of threads that manipulate infinite data.

Leap receives as input a concurrent program description and a specification and automatically generates a finite set of verification conditions which are then discharged to specialized decision procedures. 
%
The validity of all discharged verification conditions implies that the program executed by any number of threads satisfies the specification. 
%
Currently, Leap includes not only decision procedures for integers and Booleans, but it also implements specific theories for heap memory layouts such as linked-lists and skiplists.

\leap uses decision procedures, based in model searching while \spass does automatic reasoning. 



\subsubsection{\gandalf}

\gandalf is the implemented tool to prove \gls{VC} with \spass. 
%
The process of converting the \leap-generated verification condition to \spass syntax, splitting different conjunctions in different files, calling \spass to try to prove and process and storing the results has been automatized basically in python combined with bash.
Plus, \spass reads prefix syntax while leap has prefix and infix syntax.

Additionally, \gandalf does some reduction of the formulas so \spass can finish in a reasonable time.
%
This is needed because \spass does not have information nor tactics about which axiom should be used first.



\subsubsection{Process}

Once the tools have been described, we can explain the process. 
%
The process followed is shown in the figure \ref{fig:process}

\todo{Figure Process}

The first step consists on translate the program given to its \gls{VC}. 
%
This is done by \leap. 
%
In the case of this problem, \leap generates at least 2 \gls{VC} for each transition of the program. 
%
The first \gls{VC} corresponds to the self-consecution \gls{VC} and the second corresponds to the others-consecution \gls{VC}. The initial transition is also generated.
%
This is done 6 times, one for each existing invariant of the problem (\ref{invariants}).

Once the \gls{VC} have been generated, the goal is to prove all of them using \spass. 
%
As \spass syntax (\cite{spasssyntax}) it is not \leap syntax (\cite{leapsyntax}) some parsing is needed. 
%
For example, \spass uses prefix notation while \leap uses infix notation for binary functions.
%
In order to solve this, it was necessary to learn Ocaml using \citeapos{ocamlbook}.
%
\leap was forked so it could write the \gls{VC} in prefix notation.
%
Some other \leap functionalities had to be changed to make compatible \leap output with \spass input.

In addition, one \spass problem has to be created for each \gls{VC}. 
%
The axiom list for each problem is determined as an argument. 
%
This is explained further in section \ref{sec:axiomgraph} inside chapter \ref{chap:results}.

Because of \spass lack of tactics and the size of the axiom list, some very easy transitions could take too much time.
%
Transitions which can be proven using \pc reasoning \spass could take minutes. 
%
In order to improve \spass performance, the problem is divided.

Let $\varphi$ be the \gls{VC} to prove. For the explanation, 
	\[
		\varphi \equiv \pc(i) = l_j \to \head \not = \tail
	\]
While proving the \gls{VC} for the thread $i$, $\pc(i)$ has a value which may be $l_j$ or not. 
%
In both cases, resolution can be applied to prove equivalently
	\[
		\psi \equiv \head \not = \tail
	\]
This resolution can be applied in self-consecution and in others-consecution. 
%
This resolution step is done by \gandalf. 
%
Some other obvious resolution is done, such as $\top \andcond \top \equiv \top$ and other tautologies.

Now \spass can prove a simpler problem and it takes \doubt{little time}.
%
To assure this is equivalent, another \spass problem is generated. 
%
This \spass problem aims to prove $\psi \to \varphi$.
%
In order to prove $\varphi$, modus ponens is used:
\[
	\begin{array}{c}
		\psi\\
		\psi \to \varphi\\\hline
		\varphi
	\end{array}
\]

Despite 2 \spass problems have to be solved instead of 1, in \ref{sec:timeanalysis} the reader can see the benefits of applying this \doubt{workaround}.
%
And this is a logic consequence. $\varphi$ is a complex problem for \spass. 
%
However, $\psi$ is a much simpler problem because it does not include any reasoning about the program counter. 
% How to express:
%%%%    I didn't check that is tautologic in every transition, but i am pretty sure it is.
In addition, the problem $\psi \to \varphi$ is tautologic in most of the transitions because it adds conjectures in the precedent of an implication, which can't make invalid a valid formula.
%
Plus, the list of axioms needed to prove this problem is small in compare to the axiom list needed to prove the original problem.

$\psi$ \spass problem should not have $\pc$ involved. How can the $\pc$ be removed while proving others-consecution?
%
Again, more \spass problems are generated. 
%
In this case, 55 new problems are generated and modus ponens is used:
\[
	\begin{array}{c}
		\psi_1\\
		\psi_2\\
		\vdots\\
		\psi_{55}\\
		(\bigwedge_i \psi_i) \to \varphi\\\hline
		\varphi
	\end{array}
\]
%%%%    I didn't check that is tautologic in every transition, but i am pretty sure it is.
In this case, the \spass problem $(\bigwedge_i \psi_i) \to \varphi$ is tautologic most of the time, but \spass takes much more time than before because of the problem size.




\subsubsection{Ocaml - parser}
\todo{Proof axioms with leap}
In order to improve Ocaml skills needed, a parser was implemented. 
%
This parser translates from \spass or \leap syntax to human syntax. 
%
The idea of this parser came at the 5th time the list of axioms seemed finished. 
%
The axiom list needed to be supervised because it is the basis of all the problem. 
%%Rewrite





\section{Linked list}
\label{proof:Preserve}

With everything exposed before we have been able to prove the invariance of \invPreserve invariant.
%
\todo{Difficult to write}


\subsection{Axioms}

The set of relevant axioms needed to prove all the invariants is presented in this part.
%
There are a lot of secondary axioms needed by \spass that have been omitted. 
%
The omitted axioms refer to the sorting (\spass is not multi-sorted as the theory we work with), to constants (\spass does not include arithmetic, so $0$,$1$,... must be defined as unique 0-ary functions). 
%
The set of axioms is:
		
\begin{description}
\item[union-def] 
\label{ax::union_def}

\begin{dmath}
((in(x,se) \vee in(x,se2)) \dimplies in(x,union(se,se2)))
\end{dmath}

\textcolor{red}{Auto-generated}
\item[Intr-def] 
\label{ax::Intr_def}

\begin{dmath}
((in(x,se) \wedge in(x,se2)) \dimplies in(x,intr(se,se2)))
\end{dmath}

\textcolor{red}{Auto-generated}
\item[SetDiff-def] 
\label{ax::SetDiff_def}

\begin{dmath}
((in(x,se) \wedge (\neg  (in(x,se2)))) \dimplies in(x,diff(se,se2)))
\end{dmath}

\textcolor{red}{Auto-generated}
\item[in-set--def] 
\label{ax::in_set__def}

\begin{dmath}
((\neg  (in(a,se))) \implies in(b,se) \implies (\neg  (b = a)))
\end{dmath}

\textcolor{red}{Auto-generated}
\item[a--in--singl-a] 
\label{ax::a__in__singl_a}

\begin{dmath}
(((\neg  (a = b)) \implies (\neg  (in(b,{ a })))) \wedge in(a,{ a }))
\end{dmath}

\textcolor{red}{Auto-generated}
\item[a-not--in-se-dif-a] 
\label{ax::a_not__in_se_dif_a}

\begin{dmath}
(in(a,se) \implies (\neg  (in(a,diff(se,{ a })))))
\end{dmath}

\textcolor{red}{Auto-generated}
\item[nextreg] 
\label{ax::nextreg}

\begin{dmath}
((in(a,se) \wedge se = addr2set(m,b) \wedge c = rd(m,a).next \wedge (\neg  (a = null))) \implies in(c,se))
\end{dmath}

\textcolor{red}{Auto-generated}
\item[data--def] 
\label{ax::data__def}

\begin{dmath}
mkcell(e,a,t).data = e
\end{dmath}

\textcolor{red}{Auto-generated}
\item[next--def] 
\label{ax::next__def}

\begin{dmath}
mkcell(e,a,t).next = a
\end{dmath}

\textcolor{red}{Auto-generated}
\item[lockid--def] 
\label{ax::lockid__def}

\begin{dmath}
mkcell(e,a,t).lockid = t
\end{dmath}

\textcolor{red}{Auto-generated}
\item[next-error--is--null] 
\label{ax::next_error__is__null}

\begin{dmath}
error.next = null
\end{dmath}

\textcolor{red}{Auto-generated}
\item[equality-bt-cell] 
\label{ax::equality_bt_cell}

\begin{dmath}
(c1 = c2 \implies (c1.data = c2.data \wedge c1.lockid = c2.lockid \wedge c1.next = c2.next))
\end{dmath}

\textcolor{red}{Auto-generated}
\item[upd--def--not-null] 
\label{ax::upd__def__not_null}

\begin{dmath}
((\neg  (a = null)) \implies upd(m,a,c) = m2 \implies rd(m2,a) = c)
\end{dmath}

\textcolor{red}{Auto-generated}
\item[upd--def--null] 
\label{ax::upd__def__null}

\begin{dmath}
(a = null \implies upd(m,a,c) = m2 \implies m2 = m)
\end{dmath}

\textcolor{red}{Auto-generated}
\item[upd--def--one-at-the-time] 
\label{ax::upd__def__one_at_the_time}

\begin{dmath}
(((\neg  (a = null)) \wedge (\neg  (a = b))) \implies upd(m,a,c) = m2 \implies rd(m,b) = rd(m2,b))
\end{dmath}

\textcolor{red}{Auto-generated}
\item[rd-mem--def] 
\label{ax::rd_mem__def}

\begin{dmath}
rd(m,null) = error
\end{dmath}

\textcolor{red}{Auto-generated}
\item[lowest--less-than-highest] 
\label{ax::lowest__less_than_highest}

\begin{dmath}
(\neg  (lowestElem = highestElem))
\end{dmath}

\textcolor{red}{Auto-generated}
\item[less-total] 
\label{ax::less_total}

\begin{dmath}
(\neg  ((x < y \wedge y < x)))
\end{dmath}

\textcolor{red}{Auto-generated}
\item[ls-xy--not-ls-yx] 
\label{ax::ls_xy__not_ls_yx}

\begin{dmath}
(x < y \implies (\neg  (y < x)))
\end{dmath}

\textcolor{red}{Auto-generated}
\item[addr2set--def] 
\label{ax::addr2set__def}

\begin{dmath}
(in(b,addr2set(m,a)) \dimplies reach(m,a,b,getp(m,a,b)))
\end{dmath}

\textcolor{red}{Auto-generated}
\item[addr2set-null--is--singl-null] 
\label{ax::addr2set_null__is__singl_null}

\begin{dmath}
addr2set(m,null) = { null }
\end{dmath}

\textcolor{red}{Auto-generated}
\item[if--addr2set--then--reach-getp] 
\label{ax::if__addr2set__then__reach_getp}

\begin{dmath}
(se = addr2set(m,a) \implies (in(b,se) \dimplies reach(m,a,b,getp(m,a,b))))
\end{dmath}

\textcolor{red}{Auto-generated}
\item[reach--a-a-epsilon--true] 
\label{ax::reach__a_a_epsilon__true}

\begin{dmath}
reach(m,a,a,epsilon)
\end{dmath}

\textcolor{red}{Auto-generated}
\item[getp--same-addr--i-epsilon] 
\label{ax::getp__same_addr__i_epsilon}

\begin{dmath}
getp(m,a,a) = epsilon
\end{dmath}

\textcolor{red}{Auto-generated}
\item[getp--recursive--base] 
\label{ax::getp__recursive__base}

\begin{dmath}
((\neg  (a = rd(m,a).next)) \implies getp(m,a,rd(m,a).next) = [ a ])
\end{dmath}

\textcolor{red}{Auto-generated}
\item[getp--recursive--step] 
\label{ax::getp__recursive__step}

\begin{dmath}
(((\neg  (getp(m,a,b) = epsilon)) \wedge (\neg  (a = b)) \wedge (\neg  (rd(m,a).next = b))) \implies append([ a ],getp(m,rd(m,a).next,b),getp(m,a,b)))
\end{dmath}

\textcolor{red}{Auto-generated}
\item[path2set-epsilon--is--empty] 
\label{ax::path2set_epsilon__is__empty}

\begin{dmath}
path2set(epsilon) = empty
\end{dmath}

\textcolor{red}{Auto-generated}
\item[path2set-recursive--def--base] 
\label{ax::path2set_recursive__def__base}

\begin{dmath}
path2set([ a ]) = { a }
\end{dmath}

\textcolor{red}{Auto-generated}
\item[path2set-recursive--def--step] 
\label{ax::path2set_recursive__def__step}

\begin{dmath}
(\neg  (getp(m,a,b) = epsilon)) \implies (path2set(getp(m,a,b)) = union({ a },path2set(getp(m,rd(m,a).next,b))))
\end{dmath}

\textcolor{red}{Auto-generated}

\end{description}

