% -*- root: ../main.tex -*-
% -*- dic: en_GB  -*-
\chapter{State of the art\label{sec:estado_del_arte}}

\paragraph{Abstract}

We cover here some of the basics needed to understand the real thesis of this \thisworkm. We will describe some fundamentals of the first order logic (\gls{FOL}), about his undeciability and, of course, how we can use the \gls{FOL} to prove properties of a program. Just two general properties will be covered, such as safety and liveness.

The thesis of this \thisworkm is the definition of a theory for linked list with which we can prove safety. Thus, we will define safety and some examples of how one can prove safety in simpler integer programs.

We include a brief logic of temporal logic needed for lifeness proving, but this topic is not very important in this \thisworkm.

Finally, we try to answer the question of parallelism. We will cover how can we prove safety for programs with multiple threads. We will describe some very important results used in this \thisworkm 
done by \citep{thesisAle}.



\section{The need of formal verification}

\paragraph{Software bugs}

Wikipedia list of Software bugs is very interesting. Some examples are explained here:

	- 6 people dead http://sunnyday.mit.edu/papers/therac.pdf (1985)
	- Plane almost crash https://www.theguardian.com/business/2015/may/01/us-aviation-authority-boeing-787-dreamliner-bug-could-cause-loss-of-control (2015)

\paragraph{Misscompiling}


\subsection{Solutions}

\subsubsection{Compcert}

C verified compiler:  http://compcert.inria.fr/compcert-C.html

http://compcert.inria.fr/motivations.html


\paragraph{Other compilers} are in process to be formally verified.

\subsubsection{VCC}

VCC: A Verifier for Concurrent C  http://research.microsoft.com/en-us/projects/vcc/
VCC is sound -- if VCC verifies your program, it really is correct (modulo bugs in VCC itself)

VCC: A Practical System for Verifying Concurrent C. Ernie Cohen, Markus Dahlweid, Mark Hillebrand, Dirk Leinenbach, Michał Moskal, Thomas Santen, Wolfram Schulte, Stephan Tobies. 22nd International Conference on Theorem Proving in Higher Order Logics (TPHOLs 2009). (LNCS 5674). http://research.microsoft.com/en-us/um/people/moskal/pdf/tphol2009.pdf (Provides a good overall system description of VCC; the paper to cite for VCC)


\subsubsection{Spin}

http://spinroot.com/spin/what.html

Model checking, embedded C code

Three examples of inspiring applications of Spin in the last few years include the verification of the control algorithms for the new flood control barrier built in the late nineties near Rotterdam in the Netherlands. The verification work was carried out by the Dutch firm CMG (Computer Management Group) in collaboration with the Formal Methods group at the University of Twente.