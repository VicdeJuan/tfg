% -*- root: ../main.tex -*-
\chapter{State of the art\label{sec:estado_del_arte}}

\paragraph{Abstract}

We cover here some of the basics needed to understand the real thesis of this \thisworkm. We will describe some fundamentals of the first order logic, about his undeciability and, of course, how we can use the first order logic to prove properties of a program. Just two general properties will be covered, such as safety and liveness.

The thesis of this \thisworkm is the definition of a theory for linked list with which we can prove safety. Thus, we will define safety and some examples of how one can prove safety in simpler integer programs.

We include a brief logic of temporal logic needed for lifeness proving, but this topic is not very important in this \thisworkm.

Finally, we try to answer the question of parallelism. We will cover how can we prove safety for programs with multiple threads. We will describe some very important results used in this \thisworkm done by \citep{thesisAle}.

\section{First order logic}



\section{Program invariants}



\subsection{Safety}

What is proving safety.

\subsection{Lifeness}

What is proving lifeness, for what we need temporal logic.

\subsubsection{Temporal logic}

Basics on temporal logics

\subsubsection{Lifeness examples}

Now we can give some examples of lifeness examples.


\subsection{Parametrized invariants}

\subsubsection{Finite number of threads}

\subsubsection{Arbitrary number of threads}

