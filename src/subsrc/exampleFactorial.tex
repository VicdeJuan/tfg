


We study the loop version of the factorial function.


\[
	\begin{array}{l@{\hspace{0.3em}}c@{\hspace{1em}}l}
	\hline
		l_1 & : & \mathtt{x := 10} \\
		l_2 & : & \mathtt{f := 1} \\
		l_3 & : & \mathtt{\textbf{while } (x\geq 1) \textbf{ do }} \\
		l_4 & : & \mathtt{\;\;f = f*x} \\
		l_5 & : & \mathtt{\;\;x=x-1} \\ 	
		l_6 & : & \mathtt{\textbf{end while}}\\
		l_7 & : & \mathtt{\cdots}\\
	\hline
	\end{array}
\]
\label{simple:example}




We achieve to proof two formulae.

\[\tau_1 \equiv (l_5 \to \mathtt{x}\geq 1) \;\; \tau_2 \equiv \mathtt{x} \geq 0\]

First, we reduce the program to its \VC.


\[
	\begin{array}{l}
		 \psi_1 \equiv\pc(T) = l_1 \andcond \pc'(T) = l_2 \andcond \mathtt{f'=f} \andcond \mathtt{x}' = 10\\
		 \psi_2 \equiv\pc(T) = l_2 \andcond \pc'(T) = l_3 \andcond \mathtt{f}' = 1 \andcond x'=x\\
		 \psi_3 \equiv\pc(T) = l_3 \andcond \pc'(T) = l_4 \andcond \mathtt{f'=f} \andcond x'\geq 1\\
		 \psi_4 \equiv\pc(T) = l_4 \andcond \pc'(T) = l_5 \andcond \mathtt{f}' = \mathtt{f*x} \andcond \mathtt{x'=x}\\
		 \psi_5 \equiv\pc(T) = l_5 \andcond \pc'(T) = l_3 \andcond \mathtt{f'=f} \andcond \mathtt{x'=x-1}\\
		 \psi_6 \equiv\pc(T) = l_3 \andcond \pc'(T) = l_7 \andcond \mathtt{f'=f} \andcond \mathtt{x<1}
	\end{array}
\]

We need to prove, for $i=1,2$:

\[
	\left\{
		\begin{array}{lr}
			\psi_1 \andcond \tau_i \to \tau_i' &
			\psi_2 \andcond \tau_i \to \tau_i'\\
			\psi_3 \andcond \tau_i \to \tau_i' &
			\psi_4 \andcond \tau_i \to \tau_i'\\
			\psi_5 \andcond \tau_i \to \tau_i' &
			\psi_6 \andcond \tau_i \to \tau_i'
		\end{array}
	\right.
\]

\begin{center}\rule{4cm}{0.4pt}  $\tau_1$  \rule{4cm}{0.4pt}\end{center}
	
	 $\psi_1 \andcond \tau_1 \to \tau_1'$:
%	\begin{dmath*}[indentstep={0em}]
	\begin{equation*}
		(
			\underbrace{\pc(T) = l_1 \andcond \pc'(T) = l_2 \andcond \mathtt{f'=f} \andcond \mathtt{x}' = 10}_{\psi_1} \andcond (\underbrace{\pc(T) = l_5 \to \mathtt{x}\geq 1}_{\tau_1})
		) 
				\to(\underbrace{\pc'(T) = l_5 \to \mathtt{x}' \geq 1}_{\tau_1'})\\\\
	\end{equation*}
%	\end{dmath*}


	The formula is valid because $\pc(T) = l_2 \neq l_5$ thus the $\tau'_1$ is true.

	 $\psi_2 \andcond \tau_1 \to \tau_1'$:
%	\begin{dmath*}[indentstep={0em}]
	\begin{equation*}
		(
			\underbrace{\pc(T) = l_2 \andcond \pc'(T) = l_3 \andcond \mathtt{f}' = 1 \andcond x'=x}_{\psi_2} \andcond (\underbrace{\pc(T) = l_5 \to \mathtt{x}\geq 1}_{\tau_1})
		) 
			\to(\underbrace{\pc'(T) = l_5 \to \mathtt{x}' \geq 1}_{\tau_1'})\\\\
	\end{equation*}
%	\end{dmath*}


	The formula is valid because $\pc(T) = l_3 \neq l_5$ thus the $\tau'_1$ is true.

	 $\psi_3 \andcond \tau_1 \to \tau_1'$:
%	\begin{dmath*}[indentstep={0em}]
	\begin{equation*}
		(
			\underbrace{\pc(T) = l_3 \andcond \pc'(T) = l_4 \andcond \mathtt{f'=f} \andcond x'\geq 1}_{\psi_3} \andcond (\underbrace{\pc(T) = l_5 \to \mathtt{x}\geq 1}_{\tau_1})
		) 
			\to(\underbrace{\pc'(T) = l_5 \to \mathtt{x}' \geq 1}_{\tau_1'})\\\\
	\end{equation*}
%	\end{dmath*}

		The formula is valid because $\pc(T) = l_4 \neq l_5$ thus the $\tau'_1$ is true.

	 \;$\psi_4 \andcond \tau_1 \to \tau_1'$: 
%	\begin{dmath*}[indentstep={0em}]
	\begin{equation*}
		(
			\underbrace{\pc(T) = l_4 \andcond \pc'(T) = l_5 \andcond \mathtt{f}' = \mathtt{f*x} \andcond \mathtt{x'=x}}_{\psi_4} \andcond (\underbrace{\pc(T) = l_5 \to \mathtt{x}\geq 1}_{\tau_1})
		) 
			\to(\underbrace{\pc'(T) = l_5 \to \mathtt{x}' \geq 1}_{\tau_1'})\\\\
	\end{equation*}
%	\end{dmath*}

	The formula is equivalent (applying resolution) to

%	\begin{dmath*}[indentstep={0em}]
	\begin{equation*}
		(
			\mathtt{x'=x} \andcond  \mathtt{x}\geq 1
		) 
		\to (\mathtt{x}'\geq 1)
	\end{equation*}
%	\end{dmath*}


	Which is valid because of equality congruence.

	 $\psi_5 \andcond \tau_1 \to \tau_1'$:
%	\begin{dmath*}[indentstep={0em}]
	\begin{equation*}
		(
			\underbrace{\pc(T) = l_5 \andcond \pc'(T) = l_3 \andcond \mathtt{f'=f} \andcond \mathtt{x'=x-1}}_{\psi_5} \andcond (\underbrace{\pc(T) = l_5 \to \mathtt{x}\geq 1}_{\tau_1})
		) 
			\to(\underbrace{\pc'(T) = l_5 \to \mathtt{x}' \geq 1}_{\tau_1'})\\\\
	\end{equation*}
%	\end{dmath*}


	The formula is valid because $\pc(T) = l_3 \neq l_5$ thus the $\tau'_1$ is true.

	 $\psi_6 \andcond \tau_1 \to \tau_1'$:
%	\begin{dmath*}[indentstep={0em}]
	\begin{equation*}
		(
			\underbrace{\pc(T) = l_3 \andcond \pc'(T) = l_7 \andcond \mathtt{f'=f} \andcond \mathtt{x<1}}_{\psi_6} \andcond \underbrace{\pc(T) = l_5 \to \mathtt{x} \geq 1}_{\tau_1}
		) 
			\to (\underbrace{\pc'(T) = l_5 \to \mathtt{x}' \geq 1}_{\tau_1'})\\\\
	\end{equation*}
%	\end{dmath*}


	The formula is valid because $\pc(T) = l_7 \neq l_5$ thus the $\tau'_1$ is true.


\paragraph{Conclusion:} we have proven that $\pc(T) = l_5 \to x \geq 1$. 
%
This is called an \concept[Invariant]{invariant} because it is true during all the execution. 
%
This transition has been chosen specially because it is needed in the proof of $\tau_2$.

\begin{center}\rule{4cm}{0.4pt}  $\tau_2$  \rule{4cm}{0.4pt}\end{center}

	\; $\psi_1 \andcond \tau_2 \to \tau_2'$:	
%	\begin{dmath*}[indentstep={0em}]
	\begin{equation*}
		(
			\underbrace{\pc(T) = l_1 \andcond \pc'(T) = l_2 \andcond \mathtt{f'=f} \andcond \mathtt{x}' = 10}_{\psi_1} \andcond \underbrace{\mathtt{x} \geq 0}_{\tau_2}
		) 
				\to  \underbrace{\mathtt{x}' \geq 0}_{\tau_2'}\\\\
	\end{equation*}
%	\end{dmath*}


	The formula is valid because $x'=10 \to x'\geq 0$.

	\; $\psi_2 \andcond \tau_2 \to \tau_2'$:	
%	\begin{dmath*}[indentstep={0em}]
	\begin{equation*}
		(
			\underbrace{\pc(T) = l_2 \andcond \pc'(T) = l_3 \andcond \mathtt{f}' = 1 \andcond x'=x}_{\psi_2} \andcond \underbrace{\mathtt{x} \geq 0}_{\tau_2}
		) 
			\to \underbrace{\mathtt{x}' \geq 0}_{\tau_2'}\\\\
	\end{equation*}
%	\end{dmath*}



	The formula is valid because of the congruence of equality used in  $x'=x \andcond x\geq 0 \to x'\geq 0$ 

	\; $\psi_3 \andcond \tau_2 \to \tau_2'$:
%	\begin{dmath*}[indentstep={0em}]
	\begin{equation*}
		(
			\underbrace{\pc(T) = l_3 \andcond \pc'(T) = l_4 \andcond \mathtt{f'=f} \andcond x'\geq 1}_{\psi_3} \andcond \underbrace{\mathtt{x} \geq 0}_{\tau_2}
		) 
			\to \underbrace{\mathtt{x}' \geq 0}_{\tau_2'}\\\\
	\end{equation*}
%	\end{dmath*}


	The formula is valid because $x'\geq 1 \to x'\geq 0$.
	\; $\psi_4 \andcond \tau_2 \to \tau_2'$:	
%	\begin{dmath*}[indentstep={0em}]
	\begin{equation*}
		(
			\underbrace{\pc(T) = l_4 \andcond \pc'(T) = l_5 \andcond \mathtt{f}' = \mathtt{f*x} \andcond \mathtt{x'=x}}_{\psi_4} \andcond \underbrace{\mathtt{x} \geq 0}_{\tau_2}
		) 
			\to \underbrace{\mathtt{x}' \geq 0}_{\tau_2'}\\\\
	\end{equation*}
%	\end{dmath*}


	The formula is valid because of the congruence of equality used in  $x'=x \andcond x\geq 0 \to x'\geq 0$ 

	\; $\psi_5 \andcond \tau_2 \to \tau_2'$:	
%	\begin{dmath*}[indentstep={0em}]
	\begin{equation*}
		(
			\underbrace{\pc(T) = l_5 \andcond \pc'(T) = l_3 \andcond \mathtt{f'=f} \andcond \mathtt{x'=x-1}}_{\psi_5} \andcond \underbrace{\mathtt{x} \geq 0}_{\tau_2}
		) 
			\to \underbrace{\mathtt{x}' \geq 0}_{\tau_2'}\\\\
	\end{equation*}
%	\end{dmath*}


	The formula has some more difficulty. 
	%
	Because we are inside the loop $x$ should be greater than 1.
	%
	However, we don't have that information in the formula.
	
	The solution is use some \concept{support}.
	%
	A support formula is a formula added to the precedent of an implication to give more information. 
	%
	This addition does not change the validity of the formula.
	%
	We could equivalently prove

	\[
		(\psi_5 \andcond \tau_1 \andcond \tau_2 \to \tau_2') rightarrow (\psi_5\andcond \tau_2\to\tau_2')
	\]

	And this is exactly the solution to proof this \gls{VC}

	

%	\begin{dmath*}[indentstep={0em}]
	\begin{equation*}
		(
			\underbrace{\pc(T) = l_5 \andcond \pc'(T) = l_3 \andcond \mathtt{f'=f} \andcond \mathtt{x'=x-1}}_{\psi_5} \andcond \underbrace{\mathtt{x} \geq 0}_{\tau_2} \andcond \underbrace{\pc(T) = l_5 \to \mathtt{x} \geq 1}_{\tau_1}
		) 
			\to \underbrace{\mathtt{x}' \geq 0}_{\tau_2'}\\\\
	\end{equation*}
%	\end{dmath*}


	And this formula is valid. Applying resolution we get an equivalent valid formula:

	\[
		( \mathtt{x'=x-1} \andcond \mathtt{x'}\geq 1) \to \mathtt{x} \geq 0
	\]


	\; $\psi_6 \andcond \tau_2 \to \tau_2'$:
%	\begin{dmath*}[indentstep={0em}]
	\begin{equation*}
		(
			\underbrace{\pc(T) = l_3 \andcond \pc'(T) = l_7 \andcond \mathtt{f'=f} \andcond \mathtt{x<1} \andcond \mathtt{x'=x} }_{\psi_6} \andcond \underbrace{\mathtt{x} \geq 0}_{\tau_2}
		) 
			\to \underbrace{\mathtt{x}' \geq 0}_{\tau_2'}\\\\
	\end{equation*}
%	\end{dmath*}


	The formula has some more difficulty too.
	%
	One can think that the unique possible value of $x$ should be $0$ because of the content of the loop.
	%
	However, that information is not within the formula.
	%
	As we did before, some support is needed to prove this formula.
	%
	The support needed is $\tau_3 \equiv \pc(T) = l_3 \to \mathtt{x} \geq 0$.
	%
	The proof of this invariant is not included because it does not give new relevant information. 
	%
	Using this invariant, we have:

	

%	\begin{dmath*}[indentstep={0em}]
	\begin{equation*}
		(
			\underbrace{\pc(T) = l_3 \andcond \pc'(T) = l_7 \andcond \mathtt{f'=f} \andcond \mathtt{x<1} \andcond \mathtt{x'=x}}_{\psi_6} \andcond \underbrace{\mathtt{x} \geq 0}_{\tau_2} \andcond \underbrace{\pc(T) = l_3 \to \mathtt{x}\geq 0 }_{\tau_3}
		) 
			\to \underbrace{\mathtt{x}' \geq 0}_{\tau_2'}\\\\
	\end{equation*}
%	\end{dmath*}

	
	And this formula is valid. Applying resolution we get an equivalent valid formula:

	\[
		(
			\mathtt{x'=x}  \andcond \mathtt{x}\geq 0 \to \mathtt{x'}\geq 0
		)
	\]


