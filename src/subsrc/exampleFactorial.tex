


We study the loop version of the factorial function.


\[
	\begin{array}{l@{\hspace{0.3em}}c@{\hspace{1em}}l}
	\hline
		l_1 & : & \mathtt{x := 10} \\
		l_2 & : & \mathtt{f := 1} \\
		l_3 & : & \mathtt{\textbf{while } (x\geq 1) \textbf{ do }} \\
		l_4 & : & \mathtt{\;\;f = f*x} \\
		l_5 & : & \mathtt{\;\;x=x-1} \\ 	
		l_6 & : & \mathtt{\textbf{end while}}\\
		l_7 & : & \mathtt{\cdots}\\
	\hline
	\end{array}
\]
\label{simple:example}




We seek to prove two formulae.

\[\varphi_1 \equiv (l_5 \orcond l_4 \to \mathtt{x}\geq 1) \;\; \wedge \;\; \varphi_2 \equiv \mathtt{x} \geq 0\]

First, we reduce the program to its \VC.


\[
	\begin{array}{l}
		 \tau_1 \equiv\pc(T) = l_1 \andcond \pc\prime (T) = l_2 \andcond \mathtt{f\prime =f} \andcond \mathtt{x}\prime  = 10\\
		 \tau_2 \equiv\pc(T) = l_2 \andcond \pc\prime (T) = l_3 \andcond \mathtt{f}\prime  = 1 \andcond x\prime =x\\
		 \tau_3 \equiv\pc(T) = l_3 \andcond \pc\prime (T) = l_4 \andcond \mathtt{f\prime =f} \andcond x\prime \geq 1\\
		 \tau_4 \equiv\pc(T) = l_4 \andcond \pc\prime (T) = l_5 \andcond \mathtt{f}\prime  = \mathtt{f*x} \andcond \mathtt{x\prime =x}\\
		 \tau_5 \equiv\pc(T) = l_5 \andcond \pc\prime (T) = l_3 \andcond \mathtt{f\prime =f} \andcond \mathtt{x\prime =x-1}\\
		 \tau_6 \equiv\pc(T) = l_3 \andcond \pc\prime (T) = l_7 \andcond \mathtt{f\prime =f} \andcond \mathtt{x<1}
	\end{array}
\]

We need to prove, for $i=1,2$:

\[
	\left\{
		\begin{array}{l}
			\tau_1 \andcond \varphi_i \to \varphi_i\prime  \\
			\tau_2 \andcond \varphi_i \to \varphi_i\prime \\
			\tau_3 \andcond \varphi_i \to \varphi_i\prime  \\
			\tau_4 \andcond \varphi_i \to \varphi_i\prime \\
			\tau_5 \andcond \varphi_i \to \varphi_i\prime  \\
			\tau_6 \andcond \varphi_i \to \varphi_i\prime 
		\end{array}
	\right.
\]

\begin{center}\rule{4cm}{0.4pt}  $\varphi_1$  \rule{4cm}{0.4pt}\end{center}
	
	 $\tau_1 \andcond \varphi_1 \to \varphi_1\prime $:
%	\begin{dmath*}[indentstep={0em}]
	\begin{align*}
		(
			\underbrace{\pc(T) = l_1 \andcond \pc\prime (T) = l_2 \andcond \mathtt{f\prime =f} \andcond \mathtt{x}\prime  = 10}_{\tau_1} \andcond (\underbrace{[\pc(T) = l_5 \orcond \pc(T) = l_4 ] \to \mathtt{x}\geq 1}_{\varphi_1})
		) \\
				\to(\underbrace{[\pc\prime (T) = l_5 \orcond \pc\prime(T) = l_4] \to \mathtt{x}\prime  \geq 1}_{\varphi_1\prime })\;\;\;\;
	\end{align*}
%	\end{dmath*}


	The formula is valid because $\pc(T) = l_2 \neq l_5 \andcond l_2 \neq l_4$ thus the $\varphi\prime _1$ is true.

	 $\tau_2 \andcond \varphi_1 \to \varphi_1\prime $:
%	\begin{dmath*}[indentstep={0em}]
	\begin{align*}
		(
			\underbrace{\pc(T) = l_2 \andcond \pc\prime (T) = l_3 \andcond \mathtt{f}\prime  = 1 \andcond x\prime =x}_{\tau_2} \andcond (\underbrace{[\pc(T) = l_5 \orcond \pc(T) = l_4 ] \to \mathtt{x}\geq 1}_{\varphi_1})
		) \\
			\to(\underbrace{[\pc\prime (T) = l_5 \orcond \pc\prime(T) = l_4] \to \mathtt{x}\prime  \geq 1}_{\varphi_1\prime })\;\;\;\;
	\end{align*}
%	\end{dmath*}


	The formula is valid because $\pc(T) = l_3 \neq l_5 \andcond l_2 \neq l_4$ thus the $\varphi\prime _1$ is true.

	 $\tau_3 \andcond \varphi_1 \to \varphi_1\prime $:
%	\begin{dmath*}[indentstep={0em}]
	\begin{align*}
		(
			\underbrace{\pc(T) = l_3 \andcond \pc\prime (T) = l_4 \andcond \mathtt{f\prime =f} \andcond \textcolor{orange}{x\prime \geq 1}}_{\tau_3} \andcond (\underbrace{[\pc(T) = l_5 \orcond \pc(T) = l_4 ] \to \mathtt{x}\geq 1}_{\varphi_1})
		) \\
			\to(\underbrace{[\pc\prime (T) = l_5 \orcond \pc\prime(T) = l_4] \to \mathtt{x}\prime  \geq 1}_{\varphi_1\prime })\;\;\;\;
	\end{align*}
%	\end{dmath*}

		The formula is valid because $x\prime \geq 1$ thus the $\varphi\prime _1$ is true.

	 \;$\tau_4 \andcond \varphi_1 \to \varphi_1\prime $: 
%	\begin{dmath*}[indentstep={0em}]
	\begin{align*}
		(
			\underbrace{\pc(T) = l_4 \andcond \pc\prime (T) = l_5 \andcond \mathtt{f}\prime  = \mathtt{f*x} \andcond \mathtt{x\prime =x}}_{\tau_4} \andcond (\underbrace{[\pc(T) = l_5 \orcond \pc(T) = l_4 ] \to \mathtt{x}\geq 1}_{\varphi_1})
		) \\
			\to(\underbrace{[\pc\prime (T) = l_5 \orcond \pc\prime(T) = l_4] \to \mathtt{x}\prime  \geq 1}_{\varphi_1\prime })\;\;\;\;
	\end{align*}
%	\end{dmath*}

	The formula is equivalent (applying resolution) to

%	\begin{dmath*}[indentstep={0em}]
	\begin{align*}
		(
			\mathtt{x\prime =x} \andcond  \mathtt{x}\geq 1
		) 
		\to (\mathtt{x}\prime \geq 1)
	\end{align*}
%	\end{dmath*}


	Which is valid because of equality congruence.

	 $\tau_5 \andcond \varphi_1 \to \varphi_1\prime $:
%	\begin{dmath*}[indentstep={0em}]
	\begin{align*}
		(
			\underbrace{[\pc(T) = l_5 \orcond \pc(T) = l_4 ]\andcond \pc\prime (T) = l_3 \andcond \mathtt{f\prime =f} \andcond \mathtt{x\prime =x-1}}_{\tau_5} \andcond (\underbrace{[\pc(T) = l_5 \orcond \pc(T) = l_4 ] \to \mathtt{x}\geq 1}_{\varphi_1})
		) \\
			\to(\underbrace{[\pc\prime (T) = l_5 \orcond \pc\prime(T) = l_4] \to \mathtt{x}\prime  \geq 1}_{\varphi_1\prime })\;\;\;\;
	\end{align*}
%	\end{dmath*}


	The formula is valid because $\pc(T) = l_3 \neq l_5 \andcond l_2 \neq l_4$ thus the $\varphi\prime _1$ is true.

	 $\tau_6 \andcond \varphi_1 \to \varphi_1\prime $:
%	\begin{dmath*}[indentstep={0em}]
	\begin{align*}
		(
			\underbrace{\pc(T) = l_3 \andcond \pc\prime (T) = l_7 \andcond \mathtt{f\prime =f} \andcond \mathtt{x<1}}_{\tau_6} \andcond \underbrace{\pc(T) = l_5 \to \mathtt{x} \geq 1}_{\varphi_1}
		) \\
			\to (\underbrace{[\pc\prime (T) = l_5 \orcond \pc\prime(T) = l_4] \to \mathtt{x}\prime  \geq 1}_{\varphi_1\prime })\;\;\;\;
	\end{align*}
%	\end{dmath*}


	The formula is valid because $\pc(T) = l_7 \neq l_5$ thus the $\varphi\prime _1$ is true.


\paragraph{Conclusion:} we have proven that $\pc(T) = l_5 \to x \geq 1$. 
%
This is called an \concept[Invariant]{invariant} because it is always true in all executions of the program. 
%
This invariant has been chosen specially because it is needed in the proof of $\varphi_2$.

\begin{center}\rule{4cm}{0.4pt}  $\varphi_2$  \rule{4cm}{0.4pt}\end{center}

	\; $\tau_1 \andcond \varphi_2 \to \varphi_2\prime $:	
%	\begin{dmath*}[indentstep={0em}]
	\begin{equation*}
		(
			\underbrace{\pc(T) = l_1 \andcond \pc\prime (T) = l_2 \andcond \mathtt{f\prime =f} \andcond \mathtt{x}\prime  = 10}_{\tau_1} \andcond \underbrace{\mathtt{x} \geq 0}_{\varphi_2}
		) 
				\to  \underbrace{\mathtt{x}\prime  \geq 0}_{\varphi_2\prime }\\\\
	\end{equation*}
%	\end{dmath*}


	The formula is valid because $x\prime =10 \to x\prime \geq 0$.

	\; $\tau_2 \andcond \varphi_2 \to \varphi_2\prime $:	
%	\begin{dmath*}[indentstep={0em}]
	\begin{equation*}
		(
			\underbrace{\pc(T) = l_2 \andcond \pc\prime (T) = l_3 \andcond \mathtt{f}\prime  = 1 \andcond x\prime =x}_{\tau_2} \andcond \underbrace{\mathtt{x} \geq 0}_{\varphi_2}
		) 
			\to \underbrace{\mathtt{x}\prime  \geq 0}_{\varphi_2\prime }\\\\
	\end{equation*}
%	\end{dmath*}



	The formula is valid because of the congruence of equality used in  $x\prime =x \andcond x\geq 0 \to x\prime \geq 0$ 

	\; $\tau_3 \andcond \varphi_2 \to \varphi_2\prime $:
%	\begin{dmath*}[indentstep={0em}]
	\begin{equation*}
		(
			\underbrace{\pc(T) = l_3 \andcond \pc\prime (T) = l_4 \andcond \mathtt{f\prime =f} \andcond x\prime \geq 1}_{\tau_3} \andcond \underbrace{\mathtt{x} \geq 0}_{\varphi_2}
		) 
			\to \underbrace{\mathtt{x}\prime  \geq 0}_{\varphi_2\prime }\\\\
	\end{equation*}
%	\end{dmath*}


	The formula is valid because $x\prime \geq 1 \to x\prime \geq 0$.
	\; $\tau_4 \andcond \varphi_2 \to \varphi_2\prime $:	
%	\begin{dmath*}[indentstep={0em}]
	\begin{equation*}
		(
			\underbrace{\pc(T) = l_4 \andcond \pc\prime (T) = l_5 \andcond \mathtt{f}\prime  = \mathtt{f*x} \andcond \mathtt{x\prime =x}}_{\tau_4} \andcond \underbrace{\mathtt{x} \geq 0}_{\varphi_2}
		) 
			\to \underbrace{\mathtt{x}\prime  \geq 0}_{\varphi_2\prime }\\\\
	\end{equation*}
%	\end{dmath*}


	The formula is valid because of the congruence of equality used in  $x\prime =x \andcond x\geq 0 \to x\prime \geq 0$ 

	\; $\tau_5 \andcond \varphi_2 \to \varphi_2\prime $:	
%	\begin{dmath*}[indentstep={0em}]
	\begin{equation*}
		(
			\underbrace{\pc(T) = l_5 \andcond \pc\prime (T) = l_3 \andcond \mathtt{f\prime =f} \andcond \mathtt{x\prime =x-1}}_{\tau_5} \andcond \underbrace{\mathtt{x} \geq 0}_{\varphi_2}
		) 
			\to \underbrace{\mathtt{x}\prime  \geq 0}_{\varphi_2\prime }\\\\
	\end{equation*}
%	\end{dmath*}


	The formula has some more difficulty. 
	%
	Inside the loop $x$ should be greater than 1.
	%
	However, that information is not encoded in the formula.
	
	The solution is use some \concept{support}.
	%
	A support formula is an invariant formula added to the antecedent of an implication to give more information. 
	%
	This addition does not change the validity of the formula.
	%
	We could equivalently prove

	\[
		(\tau_5 \andcond \varphi_1 \andcond \varphi_2 \to \varphi_2\prime ) \rightarrow (\varphi_2\to\varphi_2\prime )
	\]

	And this is exactly the solution to proof this \gls{VC}

	

%	\begin{dmath*}[indentstep={0em}]
	\begin{equation*}
		(
			\underbrace{\pc(T) = l_5 \andcond \pc\prime (T) = l_3 \andcond \mathtt{f\prime =f} \andcond \mathtt{x\prime =x-1}}_{\tau_5} \andcond \underbrace{\mathtt{x} \geq 0}_{\varphi_2} \andcond \underbrace{[\pc(T) = l_5 \orcond \pc(T) = l_4] \to \mathtt{x} \geq 1}_{\varphi_1}
		) 
			\to \underbrace{\mathtt{x}\prime  \geq 0}_{\varphi_2\prime }\\\\
	\end{equation*}
%	\end{dmath*}


	And this formula is valid. Applying resolution we get an equivalent valid formula:

	\[
		( \mathtt{x\prime =x-1} \andcond \mathtt{x\prime }\geq 1) \to \mathtt{x} \geq 0
	\]


	\; $\tau_6 \andcond \varphi_2 \to \varphi_2\prime $:
%	\begin{dmath*}[indentstep={0em}]
	\begin{equation*}
		(
			\underbrace{\pc(T) = l_3 \andcond \pc\prime (T) = l_7 \andcond \mathtt{f\prime =f} \andcond \mathtt{x<1} \andcond \mathtt{x\prime =x} }_{\tau_6} \andcond \underbrace{\mathtt{x} \geq 0}_{\varphi_2}
		) 
			\to \underbrace{\mathtt{x}\prime  \geq 0}_{\varphi_2\prime }\\\\
	\end{equation*}
%	\end{dmath*}
	

	
	And this formula is valid. Applying resolution we get an equivalent valid formula:

	\[
		(
			\mathtt{x\prime =x}  \andcond \mathtt{x}\geq 0 \to \mathtt{x\prime }\geq 0
		)
	\]

\paragraph{Conclusion} We have proof that $\varphi_1$ and $\varphi_2$ are invariants.
