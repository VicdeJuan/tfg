
%	\begin{axiomdescription}[equality-bt-cell]
%		\label{ax::equality_bt_cell}
%		\explanation{
%			\paragraph{Interpretation:}
%			Let c1,c2 be cells. We claim 2 cells are equal if all their fields are equal.
%		}
%		\begin{formula}
%			(c1 = c2 \implies (data(c1) = data(c2) \wedge c1.lockid = c2.lockid \wedge next(c1) = next(c2)))
%		\end{formula}
%	\end{axiomdescription}

%	\begin{axiomdescription}[equality-on-read]
%		\label{ax::equality_on_read}
%		\explanation{
%			\paragraph{Interpretation:}
%			Let a,b be addresses and m a memory. Then, reading \doubt{on/in} a memory two equal addresses return the same cell.
%		}
%		\begin{formula}
%			(a = b \implies rd(m,a) = rd(m,b))
%		\end{formula}
%	\end{axiomdescription}


%	UNUSED
%	\begin{axiomdescription}[set-exten]
%		\label{ax::set_exten}
%		\explanation{
%			\paragraph{Interpretation:}
%
%		}
%		\begin{formula}
%			se1 = se2 \implies (\forall a in(a,se1) \dimplies in(a,se2))
%		\end{formula}
%	\end{axiomdescription}


%	UNUSED
%	\begin{axiomdescription}[SetDiff-def]
%		\label{ax::SetDiff_def}
%		\explanation{
%			\paragraph{Interpretation:}
%			Difference of set definition.
%			%
%			A generic element is not in the difference of any set with $\fSingl(a)$
%		}
%		\begin{formula}
%			((in(x,se) \wedge (\neg\;  in(x,se2))) \dimplies in(x,diff(se,se2)))
%		\end{formula}
%	\end{axiomdescription}





%	UNUSED
%	\begin{axiomdescription}[less-total]
%		\label{ax::less_total}
%		\explanation{
%			\paragraph{Interpretation:}
%			The order relation \doubt{between/among} \elem is total.
%		}
%		\begin{formula}
%			(\neg\;  (x < y \wedge y < x))
%		\end{formula}
%	\end{axiomdescription}


%	\begin{axiomdescription}[not-in-region--not-change-heap-list]
%		\label{ax::not_in_region__not_change_heap_list}
%		\explanation{
%			\paragraph{Interpretation:}
%			Let $hp$ be an \addr and $hp$ a \mem. 
%			%
%			Then, modifying a \mem in an \addr which is not reachable from $hp$ preserves the order of the elements in the list.
%		}
%		\begin{formula}
%			((\neg\;  in(a,addr2set(hp,hd))) \implies orderlist(hp,hd) = orderlist(upd(hp,a,c),hd))
%		\end{formula}
%	\end{axiomdescription}


	\begin{axiomdescription}[just-tail--points--null]
		\label{ax::just_tail__points__null}
		\explanation{
			\paragraph{Interpretation:}
			The only node of the list pointing to \fNull is the sentinel node \tail. 
			%
			One may think that this axiom do not give new any new truth to the axiom set, because of the two previous ones.
			%
			Independently of this axiom being redundant, it provides \spass a performance improvement.
			%
			As it has been stated, this set of axioms is not the minimum set of axioms but is a set sufficient to prove all the invariants.
			\\
			Let $hd,tl$ be \addr reachable from $hd$ in the \mem $hp$, both different from \fNull, and $tl$ pointing to \fNull.
			%
			Let $d$ be a not \fNull \addr which points to \fNull 
		}
		\begin{formula}
			((in(tl,addr2set(hp,hd)) \wedge (\neg\;  hd = null) \wedge (\neg\;  tl = null) \wedge next(rd(hp,tl)) = null \wedge (\neg\;  d = null) \wedge next(rd(hp,d)) = null \wedge in(d,addr2set(hp,hd))) \implies d = tl)
		\end{formula}
	\end{axiomdescription}


	\begin{axiomdescription}[next-injective--if-ordered]
		\label{ax::next_injective__if_ordered}
		\explanation{
			\paragraph{Interpretation:}
			\fNext is biyective inside an orderlist, i.e., every node of the list has only another node of the list pointing to it.
			\\
			Let $hd,tl$ be \addr, sentinel nodes of an order list in the \mem $hp$.
			%
			As a sentinel node $tl$ points to \fNull.
			%
			Let $a,b,c,d$ be three \addr reachable from $hd$, every \addr different from \fNull and $a,c$ different from $tl$.
			\\
			If $c$ points to $a$, $d$ points to $a$ and $a=b$, \textbf{$c=d$}.
		}
		\begin{formula}
			((orderlist(hp,hd,tl) \wedge in(a,addr2set(hp,hd)) \wedge in(b,addr2set(hp,hd)) \wedge in(c,addr2set(hp,hd)) \wedge in(d,addr2set(hp,hd)) \wedge (\neg\;  tl = null) \wedge null = next(rd(hp,tl)) \wedge (\neg\;  c = null) \wedge (\neg\;  d = null) \wedge (\neg\;  a = null) \wedge (\neg\;  b = null) \wedge (\neg\;  a = tl) \wedge (\neg\;  b = tl) \wedge next(rd(hp,c)) = a \wedge next(rd(hp,d)) = b) \implies a = b \implies c = d)
		\end{formula}
	\end{axiomdescription}