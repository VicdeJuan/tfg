
%\chapter{\spass\; syntax file \& full list of axioms}
\label{spass:syntax_file}
\begin{lstlisting}

begin_problem(Template).


list_of_descriptions.
 name({*Template*}).
 author({*Victor de Juan*}).
 status(unknown).
 description({*desc*}).
end_of_list.



list_of_symbols.
functions[

%%  %%  static global variables:
    region_prime,region,heap_prime,heap,elements_prime,elements,i,
%%  %%  local in threads and in procedures variables:
    search_prev_prime_i,search_prev_i,search_curr_prime_i,search_curr_i,search_aux_prime_i,search_aux_i,search_e_prime_i,search_e_i,search_e_prime_i,search_e_i,search_e_prime_i,search_e_i,remove_prev_prime_i,remove_prev_i,remove_curr_prime_i,remove_curr_i,remove_aux_prime_i,remove_aux_i,remove_e_prime_i,remove_e_i,remove_e_prime_i,remove_e_i,remove_e_prime_i,remove_e_i,insert_prev_prime_i,insert_prev_i,insert_curr_prime_i,insert_curr_i,insert_aux_prime_i,insert_aux_i,insert_e_prime_i,insert_e_i,insert_e_prime_i,insert_e_i,insert_e_prime_i,insert_e_i,

%%  %%   numbers: 
    0,1,2,3,4,5,
%%  %%  functions:
    
%%   %%  mem

(null,0),(upd,3),(rd,2),
%%   %%  W_reach


%%   %%  W_bridge

(path2set,1),(addr2set,2),(getp,3),(fstlock,2),
%%   %%  path

(epsilon,0),(consPath,1),
%%   %%  addr

(freshaddr,0),
%%   %%  setaddr

(emptySet,0),(union,2),(setDiff,2),(singl,1),
%%   %%  elem

(highestElem,0),(lowestElem,0),(main_e_prime_i,0),(main_e_i,0),
%%   %%  setelem

(emptySetElem,0),(unionElem,2),(setDiffElem,2),(singlElem,1),(set2elem,2),
%%   %%  cell

(error,0),(mkcell,3),(data,1),(next,1),(lockid,1),(lock,2),(unlock,1),(head,0),(tail,0),(freshcell,0),
%%   %%  nat

(s,1),
%%   %%  tid

(nothread,0),(pc_prime_i,0),(pc_i,0),
%%   %%  settid

(emptySetTh,0),(unionTh,2),(setDiffTh,2),(singlTh,1)
].


predicates[
 %  program
    
%%   %%  mem
    
%%   %%  W_reach
    (reach,4),
%%   %%  W_bridge
    (orderlist,3),(initial,0),(search_result_prime_i,0),(search_result_i,0),
%%   %%  path
    (append,3),
%%   %%  addr
    
%%   %%  setaddr
    (in,2),(sub,2),
%%   %%  elem
    (ls_elem,2),
%%   %%  setelem
    (inElem,2),(subElem,2),
%%   %%  cell
    
%%   %%  nat
    (ls,2),
%%   %%  tid
    (ls_tid,2),
%%   %%  settid
    (inTh,2),(subTh,2)
].
sorts[
    mem,path,addr,setaddr,elem,setelem,cell,nat,tid,settid
].
end_of_list.


list_of_formulae(axioms).
    %% %% %% sorts_types
    formula(tid(i),i__is__Tid_tllign). formula(equal(i,0),i__def_tllign).

    formula(setaddr(region_prime),setaddr_prime__def_tllign).
    formula(setaddr(region),setaddr__def_tllign).
    formula(mem(heap_prime),mem_prime__def_tllign).
    formula(mem(heap),mem__def_tllign).
    formula(setelem(elements_prime),setelem_prime__def_tllign).
    formula(setelem(elements),setelem__def_tllign).

    %%   %% mem
    formula(addr(null),null__is__addr_tllign).
    formula(forall([mem(h),addr(a),cell(c)],mem(upd(h,a,c))),upd__is__mem_tllign).
    formula(forall([mem(h),addr(a)],cell(rd(h,a))),rd__is__cell_tllign).

    %%   %% W_reach

    %%   %% W_bridge
    formula(forall([path(p)],setaddr(path2set(p))),path2set__is__setaddr_tllign).
    formula(forall([mem(h),addr(a)],setaddr(addr2set(h,a))),addr2set__is__setaddr_tllign).
    formula(forall([mem(h),addr(a),addr(a1)],path(getp(h,a,a1))),getp__is__path_tllign).
    formula(forall([mem(h),path(p)],addr(fstlock(h,p))),fstlock__is__addr_tllign).

    %%   %% path
    formula(path(epsilon),epsilon__is__path_tllign).
    formula(forall([addr(a)],path(consPath(a))),consPath__is__path_tllign).

    %%   %% addr
    formula(addr(freshaddr),freshaddr__is__addr_tllign).

    %%   %% setaddr
    formula(setaddr(emptySet),emptySet__is__setaddr_tllign).
    formula(forall([setaddr(set_a),setaddr(set_a1)],setaddr(union(set_a,set_a1))),union__is__setaddr_tllign).
    formula(forall([setaddr(set_a),setaddr(set_a1)],setaddr(setDiff(set_a,set_a1))),setDiff__is__setaddr_tllign).
    formula(forall([addr(a)],setaddr(singl(a))),singl__is__setaddr_tllign).

    %%   %% elem
    formula(elem(highestElem),highestElem__is__elem_tllign).
    formula(elem(lowestElem),lowestElem__is__elem_tllign).
    formula(elem(main_e_prime_i),main_e_prime_i__is__elem_tllign).
    formula(elem(main_e_i),main_e_i__is__elem_tllign).

    %%   %% setelem
    formula(setelem(emptySetElem),emptySetElem__is__setelem_tllign).
    formula(forall([setelem(set_e),setelem(set_e1)],setelem(unionElem(set_e,set_e1))),unionElem__is__setelem_tllign).
    formula(forall([setelem(set_e),setelem(set_e1)],setelem(setDiffElem(set_e,set_e1))),setDiffElem__is__setelem_tllign).
    formula(forall([elem(e)],setelem(singlElem(e))),singlElem__is__setelem_tllign).
    formula(forall([setaddr(set_a),mem(h)],setelem(set2elem(set_a,h))),set2elem__is__setelem_tllign).

    %%   %% cell
    formula(cell(error),error__is__cell_tllign).
    formula(forall([elem(e),addr(a),tid(t)],cell(mkcell(e,a,t))),mkcell__is__cell_tllign).
    formula(forall([cell(c)],elem(data(c))),data__is__elem_tllign).
    formula(forall([cell(c)],addr(next(c))),next__is__addr_tllign).
    formula(forall([cell(c)],tid(lockid(c))),lockid__is__tid_tllign).
    formula(forall([cell(c),tid(t)],cell(lock(c,t))),lock__is__cell_tllign).
    formula(forall([cell(c)],cell(unlock(c))),unlock__is__cell_tllign).
    formula(addr(head),head__is__addr_tllign).
    formula(addr(tail),tail__is__addr_tllign).
    formula(cell(freshcell),freshcell__is__cell_tllign).

    %%   %% nat
    formula(forall([nat(n)],nat(s(n))),s__is__nat_tllign).

    %%   %% tid
    formula(tid(nothread),nothread__is__tid_tllign).
    formula(nat(pc_prime_i),pc_prime_i__is__nat_tllign).
    formula(nat(pc_i),pc_i__is__nat_tllign).

    %%   %% settid
    formula(settid(emptySetTh),emptySetTh__is__settid_tllign).
    formula(forall([settid(set_t),settid(set_t1)],settid(unionTh(set_t,set_t1))),unionTh__is__settid_tllign).
    formula(forall([settid(set_t),settid(set_t1)],settid(setDiffTh(set_t,set_t1))),setDiffTh__is__settid_tllign).
    formula(forall([tid(t)],settid(singlTh(t))),singlTh__is__settid_tllign).
    formula(not(or(equal(i,nothread) )),th_nothread_diff_i_tllign).
    % % % % % % % % % % Program axioms
    % % % % % % % Natural axioms
    % numbers:
    formula(equal(s(0),1),def_1_tllign).
    formula(equal(s(1),2),def_2_tllign).
    formula(equal(s(2),3),def_3_tllign).
    formula(equal(s(3),4),def_4_tllign).
    formula(equal(s(4),5),def_5_tllign).
    formula(and(not(equal(0,1)),not(equal(0,2)),not(equal(0,3)),not(equal(0,4)),not(equal(1,2)),not(equal(1,3)),not(equal(1,4)),not(equal(2,3)),not(equal(2,4)),not(equal(3,4))),nums_are_different_tllign).

    % < and s 
    formula(forall([nat(x),nat(y)],implies(equal(x,y),equal(s(x),s(y)))),s_injective_tllign).
    formula(forall([nat(x)],not(exists([nat(y)],equal(s(y),0)))),no_negative_numbers_tllign).
    formula(addr(search_prev_prime_i),search_prev__is__addr_tllign).    formula(addr(search_prev_i),search_prev__is__addr_tllign).  
    formula(addr(search_curr_prime_i),search_curr__is__addr_tllign).    formula(addr(search_curr_i),search_curr__is__addr_tllign).  
    formula(addr(search_aux_prime_i),search_aux__is__addr_tllign).  formula(addr(search_aux_i),search_aux__is__addr_tllign).    
    formula(elem(search_e_prime_i),search_e__is__elem_tllign).  formula(elem(search_e_i),search_e__is__elem_tllign).    
    formula(elem(search_e_prime_i),search_e__is__elem_tllign).  formula(elem(search_e_i),search_e__is__elem_tllign).    
    formula(elem(search_e_prime_i),search_e__is__elem_tllign).  formula(elem(search_e_i),search_e__is__elem_tllign).    
    formula(addr(remove_prev_prime_i),remove_prev__is__addr_tllign).    formula(addr(remove_prev_i),remove_prev__is__addr_tllign).  
    formula(addr(remove_curr_prime_i),remove_curr__is__addr_tllign).    formula(addr(remove_curr_i),remove_curr__is__addr_tllign).  
    formula(addr(remove_aux_prime_i),remove_aux__is__addr_tllign).  formula(addr(remove_aux_i),remove_aux__is__addr_tllign).    
    formula(elem(remove_e_prime_i),remove_e__is__elem_tllign).  formula(elem(remove_e_i),remove_e__is__elem_tllign).    
    formula(elem(remove_e_prime_i),remove_e__is__elem_tllign).  formula(elem(remove_e_i),remove_e__is__elem_tllign).    
    formula(elem(remove_e_prime_i),remove_e__is__elem_tllign).  formula(elem(remove_e_i),remove_e__is__elem_tllign).    
    formula(addr(insert_prev_prime_i),insert_prev__is__addr_tllign).    formula(addr(insert_prev_i),insert_prev__is__addr_tllign).  
    formula(addr(insert_curr_prime_i),insert_curr__is__addr_tllign).    formula(addr(insert_curr_i),insert_curr__is__addr_tllign).  
    formula(addr(insert_aux_prime_i),insert_aux__is__addr_tllign).  formula(addr(insert_aux_i),insert_aux__is__addr_tllign).    
    formula(elem(insert_e_prime_i),insert_e__is__elem_tllign).  formula(elem(insert_e_i),insert_e__is__elem_tllign).    
    formula(elem(insert_e_prime_i),insert_e__is__elem_tllign).  formula(elem(insert_e_i),insert_e__is__elem_tllign).    
    formula(elem(insert_e_prime_i),insert_e__is__elem_tllign).  formula(elem(insert_e_i),insert_e__is__elem_tllign).    


    % % % % Type equivalences: : : : : : 
    formula(forall([mem(x)],and(    not(path(x)),not(addr(x)),not(setaddr(x)),not(elem(x)),not(setelem(x)),not(cell(x)),not(nat(x)),not(tid(x)),not(settid(x)))),mem_is_not_other_type_tllign).
    formula(forall([path(x)],and(   not(mem(x)),not(addr(x)),not(setaddr(x)),not(elem(x)),not(setelem(x)),not(cell(x)),not(nat(x)),not(tid(x)),not(settid(x)))),path_is_not_other_type_tllign).
    formula(forall([addr(x)],and(   not(mem(x)),not(path(x)),not(setaddr(x)),not(elem(x)),not(setelem(x)),not(cell(x)),not(nat(x)),not(tid(x)),not(settid(x)))),addr_is_not_other_type_tllign).
    formula(forall([setaddr(x)],and(not(mem(x)),not(path(x)),not(addr(x)),not(elem(x)),not(setelem(x)),not(cell(x)),not(nat(x)),not(tid(x)),not(settid(x)))),setaddr_is_not_other_type_tllign).
    formula(forall([elem(x)],and(   not(mem(x)),not(path(x)),not(addr(x)),not(setaddr(x)),not(setelem(x)),not(cell(x)),nat(x),not(tid(x)),not(settid(x)))),elem_is_not_other_type_tllign).
    formula(forall([setelem(x)],and(not(mem(x)),not(path(x)),not(addr(x)),not(setaddr(x)),not(elem(x)),not(cell(x)),not(nat(x)),not(tid(x)),not(settid(x)))),setelem_is_not_other_type_tllign).
    formula(forall([cell(x)],and(   not(mem(x)),not(path(x)),not(addr(x)),not(setaddr(x)),not(elem(x)),not(setelem(x)),not(nat(x)),not(tid(x)),not(settid(x)))),cell_is_not_other_type_tllign).
    formula(forall([nat(x)],and(    not(mem(x)),not(path(x)),not(addr(x)),not(setaddr(x)),not(setelem(x)),not(cell(x)),not(settid(x)))),nat_is_not_other_type_tllign).
    formula(forall([tid(x)],and(    not(mem(x)),not(path(x)),not(addr(x)),not(setaddr(x)),not(elem(x)),not(setelem(x)),not(cell(x)),nat(x),not(settid(x)))),tid_is_not_other_type_tllign).
    formula(forall([settid(x)],and(not(mem(x)),not(path(x)),not(addr(x)),not(setaddr(x)),not(elem(x)),not(setelem(x)),not(cell(x)),not(nat(x)),not(tid(x)))),settid_is_not_other_type_tllign).






    % % % % Set


    formula(forall([setaddr(se),setaddr(se2),addr(x)],equiv(or(in(x,se),in(x,se2)),in(x,union(se,se2)))),union_def).
    formula(forall([setaddr(se),setaddr(se2)],equal(union(se,se2),union(se2,se))),union_conmutative).
    formula(forall([addr(b),addr(a),setaddr(se)],implies(not(in(a,se)),implies(in(b,se),not(equal(b,a))))),in_set__def).
    formula(forall([addr(a),addr(b)],and(implies(not(equal(a,b)),not(in(b,singl(a)))) ,in(a,singl(a)))),a__in__singl_a).
    formula(forall([addr(a)],not(in(a,emptySet))),emptySet_is_empty).
    formula(forall([setaddr(se1),setaddr(se2)],equiv(not(exists([addr(a)],equiv(in(a,se1),not(in(a,se2))))),equal(se1,se2))),set_eq_addr).
    formula(forall([setaddr(se1),setaddr(se2)],implies(equal(se1,se2),forall([addr(a)],equiv(in(a,se1),in(a,se2))))),set_extenaddr).
    formula(forall([setaddr(se1),setaddr(se2)],implies(forall([addr(a)],equiv(in(a,se1),in(a,se2))),    equal(se1,se2))),set_exten_invaddr).
    formula(forall([setaddr(se),setaddr(se2),addr(x)],equiv(and(in(x,se),not(in(x,se2))),in(x,setDiff(se,se2)))),SetDiff_def).
    formula(forall([addr(a),setaddr(se)],implies(in(a,se),not(in(a,setDiff(se,singl(a)))))),a_not__in_se_dif_a).





    % % % % Cell
    formula(forall([cell(c)],exists([elem(e),addr(a),tid(t)],equal(c,mkcell(e,a,t)))),mckcell__def_tllign).
    formula(forall([elem(e),addr(a),tid(t)],equal(data(mkcell(e,a,t)),e)),data__def).
    formula(forall([elem(e),addr(a),tid(t)],equal(next(mkcell(e,a,t)),a)),next__def).
    formula(forall([elem(e),addr(a),tid(t)],equal(lockid(mkcell(e,a,t)),t)),lockid__def).
    formula(equal(next(error),null),next_error__is__null).
    formula(forall([cell(c1),cell(c2)],implies(equal(c1,c2),and(equal(data(c1),data(c2)),equal(lockid(c1),lockid(c2)),equal(next(c1),next(c2))))),equality_bt_cell).
    formula(forall([mem(m),addr(a),addr(b)],implies(equal(a,b),equal(rd(m,a),rd(m,b)))),equality_on_read).


            % % % % mem
    formula(forall([mem(m),addr(a),addr(b),cell(c),mem(m2)],implies(not(equal(a,null)),implies(equal(upd(m,a,c),m2),equal(rd(m2,a),c)))),upd__def__not_null).
    formula(forall([mem(m),addr(a),addr(b),cell(c),mem(m2)],implies(and(not(equal(a,null)),not(equal(a,b))),implies(equal(upd(m,a,c),m2),equal(rd(m,b),rd(m2,b))))),upd__def__one_at_the_time).
    formula(forall([mem(m)],equal(rd(m,null),error)),rd_mem__def).

            % % % % elem
    formula(not(equal(lowestElem,highestElem)),lowest__less_than_highest).
    formula(forall([elem(e)],or(equal(e,lowestElem),ls_elem(lowestElem,e))),lowestElem__def_tll).
    formula(forall([elem(e)],or(equal(e,highestElem),ls_elem(e,highestElem))),highestElem__def_tll).
    formula(forall([elem(x),elem(y),elem(z)],implies(and(ls_elem(x,y),ls_elem(y,z)),ls_elem(x,z))),less_trans).
    formula(forall([elem(x),elem(y)],not(and(ls_elem(x,y),ls_elem(y,x)))),less_total).
    formula(forall([elem(x),elem(y)],equiv(ls_elem(x,y),and(not(equal(x,y)),not(ls_elem(y,x))))),ls_xy__not_ls_yx).
    % % % % % % Important axioms: 

    formula(forall([addr(c),addr(a),addr(b),mem(m),setaddr(se)],implies(and(in(a,se),equal(se,addr2set(m,b)),equal(c,next(rd(m,a))),not(equal(a,null))),in(c,se))),nextreg).
    formula(forall([tid(t),mem(hp_p),mem(hp),addr(a),addr(hd)],implies(and(equal(hp_p,upd(hp,a,mkcell(data(rd(hp,a)),next(rd(hp,a)),t)))),equal(addr2set(hp,hd),addr2set(hp_p,hd)))),lock_keeps_addr2set).
    formula(forall([tid(t),mem(hp_p),mem(hp),addr(tl),addr(a),addr(hd)],implies(and(equal(hp_p,upd(hp,a,mkcell(data(rd(hp,a)),next(rd(hp,a)),t)))),equiv(orderlist(hp,hd,tl),orderlist(hp_p,hd,tl)))),lock_keeps_orderlist).
    formula(forall([tid(t),mem(hp_p),mem(hp),addr(a),addr(hd)],implies(and(equal(hp_p,upd(hp,a,mkcell(data(rd(hp,a)),next(rd(hp,a)),t)))),equal(data(rd(hp,a)),data(rd(hp_p,a))))),lock_keeps_heap__data).
    formula(forall([tid(t),mem(hp_p),mem(hp),addr(a),addr(hd)],implies(and(equal(hp_p,upd(hp,a,mkcell(data(rd(hp,a)),next(rd(hp,a)),t)))),equal(next(rd(hp,a)),next(rd(hp_p,a))))),lock_keeps_heap__next).
    formula(forall([addr(a),addr(hd),cell(c),mem(hp)],implies(and(not(in(a,addr2set(hp,hd)))),equal(addr2set(hp,hd),addr2set(upd(hp,a,c),hd)))),not_in_region__not_change_heap_addr).
    formula(forall([addr(tl),addr(a),addr(hd),cell(c),mem(hp)],implies(and(not(in(a,addr2set(hp,hd)))),equiv(orderlist(hp,hd,tl),orderlist(upd(hp,a,c),hd,tl)))),not_in_region__not_change_heap_list).
    formula(forall([mem(hp),addr(hd),addr(tl),addr(nl)],implies(and(ls_elem(data(rd(hp,hd)),data(rd(hp,tl))),equal(next(rd(hp,hd)),tl),equal(next(rd(hp,tl)),nl)),orderlist(hp,hd,tl))),order_primim).
    formula(forall([mem(hp),addr(hd),addr(tl)],implies(and(ls_elem(data(rd(hp,hd)),data(rd(hp,tl))),equal(next(rd(hp,hd)),tl),equal(next(rd(hp,tl)),null)),equal(addr2set(hp,hd),union(union(singl(hd),singl(tl)),singl(null))))),addr2set_primim).
    formula(forall([addr(hd),addr(prev),addr(aux),addr(curr),mem(hp),mem(hp_p),setaddr(reg),setaddr(reg_p)],implies(and(equal(reg,addr2set(hp,hd)),equal(union(reg,singl(aux)),reg_p),equal(next(rd(hp,prev)),curr),not(equal(prev,curr)),equal(next(rd(hp,aux)),curr),not(equal(aux,null)),not(equal(prev,null)),not(equal(curr,null)),in(prev,addr2set(hp,hd)),equal(hp_p,upd(hp,prev,mkcell(data(rd(hp,prev)),aux,lockid(rd(hp,prev)))))),equal(reg_p,addr2set(hp_p,hd)))),insert__keeps_addr2set).
    formula(forall([addr(curr),addr(aux),addr(prev),addr(hd),mem(hp),mem(hp_p)],implies(and(equal(next(rd(hp,curr)),aux),equal(next(rd(hp,prev)),curr),not(equal(aux,next(rd(hp,prev)))),equal(hp_p,upd(hp,prev,mkcell(data(rd(hp,prev)),aux,lockid(rd(hp,prev))))),not(equal(aux,null)),in(curr,addr2set(hp,hd)),in(null,addr2set(hp,hd)),in(prev,addr2set(hp,hd))),equal(setDiff(addr2set(hp,hd),singl(curr)),addr2set(hp_p,hd)))),remove__keeps_addr2set).
    % % % Addr2set
    formula(forall([mem(m),addr(a)],equal(addr2set(m,a),union(singl(a),addr2set(m,next(rd(m,a)))))),addr2set_rec_def).
    formula(forall([mem(m)],equal(addr2set(m,null),singl(null))),addr2set_null__is__singl_null).





    % % % Orderlist
    formula(forall([addr(a),addr(b),addr(c),addr(d),addr(hd),addr(tl),mem(hp)], implies( and( orderlist(hp,hd,tl),in(a,addr2set(hp,hd)) , in(b,addr2set(hp,hd)) , in(c,addr2set(hp,hd)) , in(d,addr2set(hp,hd)),not(equal(tl,null)),equal(null,next(rd(hp,tl))),not(equal(c,null)),not(equal(d,null)),not(equal(a,null)),not(equal(b,null)),not(equal(a,tl)),not(equal(b,tl)),equal(next(rd(hp,c)),a),equal(next(rd(hp,d)),b)),implies(equal(a,b),equal(c,d)))),next_injective__if_ordered).
    formula(forall([addr(a),addr(tl),addr(hd),mem(hp)],implies(and(in(a,addr2set(hp,hd)),in(tl,addr2set(hp,hd)),not(equal(hd,null)),not(equal(tl,null)),not(equal(a,null)),equal(next(rd(hp,tl)),null)),not(equal(next(rd(hp,a)),a)))),next_is_not_same__if_ordered).
    formula(forall([addr(d),addr(tl),addr(hd),mem(hp)],implies(and(in(tl,addr2set(hp,hd)),not(equal(hd,null)),not(equal(tl,null)),equal(next(rd(hp,tl)),null),not(equal(d,null)),equal(next(rd(hp,d)),null),in(d,addr2set(hp,hd))),equal(d,tl))),just_tail__points__null).
    formula(forall([addr(hd),mem(hp),mem(hp_p),addr(aux),addr(prev),addr(curr),addr(tl)],implies(and(orderlist(hp,hd,tl),not(equal(tl,null)),equal(next(rd(hp,tl)),null),ls_elem(data(rd(hp,prev)),data(rd(hp,aux))),ls_elem(data(rd(hp,aux)),data(rd(hp,curr))),equal(next(rd(hp,aux)),curr),equal(next(rd(hp,prev)),curr),equal(hp_p,upd(hp,prev,mkcell(data(rd(hp,prev)),aux,lockid(rd(hp,prev)))))),orderlist(hp_p,hd,tl))),insert__keeps_orderlist).
    formula(forall([addr(hd),mem(hp),mem(hp_p),addr(aux),addr(prev),addr(curr),addr(tl)],implies(and(equal(aux,next(rd(hp,curr))),equal(curr,next(rd(hp,prev))),not(equal(aux,null)),equal(null,next(rd(hp,tl))),not(equal(aux,next(rd(hp,prev)))),equal(hp_p,upd(hp,prev,mkcell(data(rd(hp,prev)),aux,lockid(rd(hp,prev))))),in(prev,addr2set(hp,hd)),in(curr,addr2set(hp,hd)),in(null,addr2set(hp,hd)),in(aux,addr2set(hp,hd)),orderlist(hp,hd,tl)),orderlist(hp_p,hd,tl))),remove__keeps_orderlist).

end_of_list.



list_of_formulae(conjectures).
    formula(false).
end_of_list.



end_problem.



\end{lstlisting}

\chapter{Inductive Assertion Method}

This example is proposed to illustrate the inductive assertion method. 
%
A simple program, which one can see it should work perfectly is verified manually applying the method.
\label{app:exampleFactorial}
\begin{example}



We study the loop version of the factorial function.


\[
	\begin{array}{l@{\hspace{0.3em}}c@{\hspace{1em}}l}
	\hline
		l_1 & : & \mathtt{x := 10} \\
		l_2 & : & \mathtt{f := 1} \\
		l_3 & : & \mathtt{\textbf{while } (x\geq 1) \textbf{ do }} \\
		l_4 & : & \mathtt{\;\;f = f*x} \\
		l_5 & : & \mathtt{\;\;x=x-1} \\ 	
		l_6 & : & \mathtt{\textbf{end while}}\\
		l_7 & : & \mathtt{\cdots}\\
	\hline
	\end{array}
\]
\label{simple:example}




We achieve to proof two formulae.

\[\tau_1 \equiv (l_5 \to \mathtt{x}\geq 1) \;\; \wedge \;\; \tau_2 \equiv \mathtt{x} \geq 0\]

First, we reduce the program to its \VC.


\[
	\begin{array}{l}
		 \psi_1 \equiv\pc(T) = l_1 \andcond \pc\prime (T) = l_2 \andcond \mathtt{f\prime =f} \andcond \mathtt{x}\prime  = 10\\
		 \psi_2 \equiv\pc(T) = l_2 \andcond \pc\prime (T) = l_3 \andcond \mathtt{f}\prime  = 1 \andcond x\prime =x\\
		 \psi_3 \equiv\pc(T) = l_3 \andcond \pc\prime (T) = l_4 \andcond \mathtt{f\prime =f} \andcond x\prime \geq 1\\
		 \psi_4 \equiv\pc(T) = l_4 \andcond \pc\prime (T) = l_5 \andcond \mathtt{f}\prime  = \mathtt{f*x} \andcond \mathtt{x\prime =x}\\
		 \psi_5 \equiv\pc(T) = l_5 \andcond \pc\prime (T) = l_3 \andcond \mathtt{f\prime =f} \andcond \mathtt{x\prime =x-1}\\
		 \psi_6 \equiv\pc(T) = l_3 \andcond \pc\prime (T) = l_7 \andcond \mathtt{f\prime =f} \andcond \mathtt{x<1}
	\end{array}
\]

We need to prove, for $i=1,2$:

\[
	\left\{
		\begin{array}{lr}
			\psi_1 \andcond \tau_i \to \tau_i\prime  &
			\psi_2 \andcond \tau_i \to \tau_i\prime \\
			\psi_3 \andcond \tau_i \to \tau_i\prime  &
			\psi_4 \andcond \tau_i \to \tau_i\prime \\
			\psi_5 \andcond \tau_i \to \tau_i\prime  &
			\psi_6 \andcond \tau_i \to \tau_i\prime 
		\end{array}
	\right.
\]

\begin{center}\rule{4cm}{0.4pt}  $\tau_1$  \rule{4cm}{0.4pt}\end{center}
	
	 $\psi_1 \andcond \tau_1 \to \tau_1\prime $:
%	\begin{dmath*}[indentstep={0em}]
	\begin{equation*}
		(
			\underbrace{\pc(T) = l_1 \andcond \pc\prime (T) = l_2 \andcond \mathtt{f\prime =f} \andcond \mathtt{x}\prime  = 10}_{\psi_1} \andcond (\underbrace{\pc(T) = l_5 \to \mathtt{x}\geq 1}_{\tau_1})
		) 
				\to(\underbrace{\pc\prime (T) = l_5 \to \mathtt{x}\prime  \geq 1}_{\tau_1\prime })\\\\
	\end{equation*}
%	\end{dmath*}


	The formula is valid because $\pc(T) = l_2 \neq l_5$ thus the $\tau\prime _1$ is true.

	 $\psi_2 \andcond \tau_1 \to \tau_1\prime $:
%	\begin{dmath*}[indentstep={0em}]
	\begin{equation*}
		(
			\underbrace{\pc(T) = l_2 \andcond \pc\prime (T) = l_3 \andcond \mathtt{f}\prime  = 1 \andcond x\prime =x}_{\psi_2} \andcond (\underbrace{\pc(T) = l_5 \to \mathtt{x}\geq 1}_{\tau_1})
		) 
			\to(\underbrace{\pc\prime (T) = l_5 \to \mathtt{x}\prime  \geq 1}_{\tau_1\prime })\\\\
	\end{equation*}
%	\end{dmath*}


	The formula is valid because $\pc(T) = l_3 \neq l_5$ thus the $\tau\prime _1$ is true.

	 $\psi_3 \andcond \tau_1 \to \tau_1\prime $:
%	\begin{dmath*}[indentstep={0em}]
	\begin{equation*}
		(
			\underbrace{\pc(T) = l_3 \andcond \pc\prime (T) = l_4 \andcond \mathtt{f\prime =f} \andcond x\prime \geq 1}_{\psi_3} \andcond (\underbrace{\pc(T) = l_5 \to \mathtt{x}\geq 1}_{\tau_1})
		) 
			\to(\underbrace{\pc\prime (T) = l_5 \to \mathtt{x}\prime  \geq 1}_{\tau_1\prime })\\\\
	\end{equation*}
%	\end{dmath*}

		The formula is valid because $\pc(T) = l_4 \neq l_5$ thus the $\tau\prime _1$ is true.

	 \;$\psi_4 \andcond \tau_1 \to \tau_1\prime $: 
%	\begin{dmath*}[indentstep={0em}]
	\begin{equation*}
		(
			\underbrace{\pc(T) = l_4 \andcond \pc\prime (T) = l_5 \andcond \mathtt{f}\prime  = \mathtt{f*x} \andcond \mathtt{x\prime =x}}_{\psi_4} \andcond (\underbrace{\pc(T) = l_5 \to \mathtt{x}\geq 1}_{\tau_1})
		) 
			\to(\underbrace{\pc\prime (T) = l_5 \to \mathtt{x}\prime  \geq 1}_{\tau_1\prime })\\\\
	\end{equation*}
%	\end{dmath*}

	The formula is equivalent (applying resolution) to

%	\begin{dmath*}[indentstep={0em}]
	\begin{equation*}
		(
			\mathtt{x\prime =x} \andcond  \mathtt{x}\geq 1
		) 
		\to (\mathtt{x}\prime \geq 1)
	\end{equation*}
%	\end{dmath*}


	Which is valid because of equality congruence.

	 $\psi_5 \andcond \tau_1 \to \tau_1\prime $:
%	\begin{dmath*}[indentstep={0em}]
	\begin{equation*}
		(
			\underbrace{\pc(T) = l_5 \andcond \pc\prime (T) = l_3 \andcond \mathtt{f\prime =f} \andcond \mathtt{x\prime =x-1}}_{\psi_5} \andcond (\underbrace{\pc(T) = l_5 \to \mathtt{x}\geq 1}_{\tau_1})
		) 
			\to(\underbrace{\pc\prime (T) = l_5 \to \mathtt{x}\prime  \geq 1}_{\tau_1\prime })\\\\
	\end{equation*}
%	\end{dmath*}


	The formula is valid because $\pc(T) = l_3 \neq l_5$ thus the $\tau\prime _1$ is true.

	 $\psi_6 \andcond \tau_1 \to \tau_1\prime $:
%	\begin{dmath*}[indentstep={0em}]
	\begin{equation*}
		(
			\underbrace{\pc(T) = l_3 \andcond \pc\prime (T) = l_7 \andcond \mathtt{f\prime =f} \andcond \mathtt{x<1}}_{\psi_6} \andcond \underbrace{\pc(T) = l_5 \to \mathtt{x} \geq 1}_{\tau_1}
		) 
			\to (\underbrace{\pc\prime (T) = l_5 \to \mathtt{x}\prime  \geq 1}_{\tau_1\prime })\\\\
	\end{equation*}
%	\end{dmath*}


	The formula is valid because $\pc(T) = l_7 \neq l_5$ thus the $\tau\prime _1$ is true.


\paragraph{Conclusion:} we have proven that $\pc(T) = l_5 \to x \geq 1$. 
%
This is called an \concept[Invariant]{invariant} because it is true during all the execution. 
%
This transition has been chosen specially because it is needed in the proof of $\tau_2$.

\begin{center}\rule{4cm}{0.4pt}  $\tau_2$  \rule{4cm}{0.4pt}\end{center}

	\; $\psi_1 \andcond \tau_2 \to \tau_2\prime $:	
%	\begin{dmath*}[indentstep={0em}]
	\begin{equation*}
		(
			\underbrace{\pc(T) = l_1 \andcond \pc\prime (T) = l_2 \andcond \mathtt{f\prime =f} \andcond \mathtt{x}\prime  = 10}_{\psi_1} \andcond \underbrace{\mathtt{x} \geq 0}_{\tau_2}
		) 
				\to  \underbrace{\mathtt{x}\prime  \geq 0}_{\tau_2\prime }\\\\
	\end{equation*}
%	\end{dmath*}


	The formula is valid because $x\prime =10 \to x\prime \geq 0$.

	\; $\psi_2 \andcond \tau_2 \to \tau_2\prime $:	
%	\begin{dmath*}[indentstep={0em}]
	\begin{equation*}
		(
			\underbrace{\pc(T) = l_2 \andcond \pc\prime (T) = l_3 \andcond \mathtt{f}\prime  = 1 \andcond x\prime =x}_{\psi_2} \andcond \underbrace{\mathtt{x} \geq 0}_{\tau_2}
		) 
			\to \underbrace{\mathtt{x}\prime  \geq 0}_{\tau_2\prime }\\\\
	\end{equation*}
%	\end{dmath*}



	The formula is valid because of the congruence of equality used in  $x\prime =x \andcond x\geq 0 \to x\prime \geq 0$ 

	\; $\psi_3 \andcond \tau_2 \to \tau_2\prime $:
%	\begin{dmath*}[indentstep={0em}]
	\begin{equation*}
		(
			\underbrace{\pc(T) = l_3 \andcond \pc\prime (T) = l_4 \andcond \mathtt{f\prime =f} \andcond x\prime \geq 1}_{\psi_3} \andcond \underbrace{\mathtt{x} \geq 0}_{\tau_2}
		) 
			\to \underbrace{\mathtt{x}\prime  \geq 0}_{\tau_2\prime }\\\\
	\end{equation*}
%	\end{dmath*}


	The formula is valid because $x\prime \geq 1 \to x\prime \geq 0$.
	\; $\psi_4 \andcond \tau_2 \to \tau_2\prime $:	
%	\begin{dmath*}[indentstep={0em}]
	\begin{equation*}
		(
			\underbrace{\pc(T) = l_4 \andcond \pc\prime (T) = l_5 \andcond \mathtt{f}\prime  = \mathtt{f*x} \andcond \mathtt{x\prime =x}}_{\psi_4} \andcond \underbrace{\mathtt{x} \geq 0}_{\tau_2}
		) 
			\to \underbrace{\mathtt{x}\prime  \geq 0}_{\tau_2\prime }\\\\
	\end{equation*}
%	\end{dmath*}


	The formula is valid because of the congruence of equality used in  $x\prime =x \andcond x\geq 0 \to x\prime \geq 0$ 

	\; $\psi_5 \andcond \tau_2 \to \tau_2\prime $:	
%	\begin{dmath*}[indentstep={0em}]
	\begin{equation*}
		(
			\underbrace{\pc(T) = l_5 \andcond \pc\prime (T) = l_3 \andcond \mathtt{f\prime =f} \andcond \mathtt{x\prime =x-1}}_{\psi_5} \andcond \underbrace{\mathtt{x} \geq 0}_{\tau_2}
		) 
			\to \underbrace{\mathtt{x}\prime  \geq 0}_{\tau_2\prime }\\\\
	\end{equation*}
%	\end{dmath*}


	The formula has some more difficulty. 
	%
	Inside the loop $x$ should be greater than 1.
	%
	However, that information is not in the implication to prove.
	
	The solution is use some \concept{support}.
	%
	A support formula is a formula added to the precedent of an implication to give more information. 
	%
	This addition does not change the validity of the formula.
	%
	We could equivalently prove

	\[
		(\psi_5 \andcond \tau_1 \andcond \tau_2 \to \tau_2\prime ) rightarrow (\psi_5\andcond \tau_2\to\tau_2\prime )
	\]

	And this is exactly the solution to proof this \gls{VC}

	

%	\begin{dmath*}[indentstep={0em}]
	\begin{equation*}
		(
			\underbrace{\pc(T) = l_5 \andcond \pc\prime (T) = l_3 \andcond \mathtt{f\prime =f} \andcond \mathtt{x\prime =x-1}}_{\psi_5} \andcond \underbrace{\mathtt{x} \geq 0}_{\tau_2} \andcond \underbrace{\pc(T) = l_5 \to \mathtt{x} \geq 1}_{\tau_1}
		) 
			\to \underbrace{\mathtt{x}\prime  \geq 0}_{\tau_2\prime }\\\\
	\end{equation*}
%	\end{dmath*}


	And this formula is valid. Applying resolution we get an equivalent valid formula:

	\[
		( \mathtt{x\prime =x-1} \andcond \mathtt{x\prime }\geq 1) \to \mathtt{x} \geq 0
	\]


	\; $\psi_6 \andcond \tau_2 \to \tau_2\prime $:
%	\begin{dmath*}[indentstep={0em}]
	\begin{equation*}
		(
			\underbrace{\pc(T) = l_3 \andcond \pc\prime (T) = l_7 \andcond \mathtt{f\prime =f} \andcond \mathtt{x<1} \andcond \mathtt{x\prime =x} }_{\psi_6} \andcond \underbrace{\mathtt{x} \geq 0}_{\tau_2}
		) 
			\to \underbrace{\mathtt{x}\prime  \geq 0}_{\tau_2\prime }\\\\
	\end{equation*}
%	\end{dmath*}


	The formula has some more difficulty too.
	%
	One can think that the unique possible value of $x$ should be $0$ because of the content of the loop.
	%
	However, that information is not within the formula.
	%
	As we did before, some support is needed to prove this formula.
	%
	The support needed is $\tau_3 \equiv \pc(T) = l_3 \to \mathtt{x} \geq 0$.
	%
	The proof of this invariant is not included because it does not give new relevant information. 
	%
	Using this invariant, we have:

	

%	\begin{dmath*}[indentstep={0em}]
	\begin{equation*}
		(
			\underbrace{\pc(T) = l_3 \andcond \pc\prime (T) = l_7 \andcond \mathtt{f\prime =f} \andcond \mathtt{x<1} \andcond \mathtt{x\prime =x}}_{\psi_6} \andcond \underbrace{\mathtt{x} \geq 0}_{\tau_2} \andcond \underbrace{\pc(T) = l_3 \to \mathtt{x}\geq 0 }_{\tau_3}
		) 
			\to \underbrace{\mathtt{x}\prime  \geq 0}_{\tau_2\prime }\\\\
	\end{equation*}
%	\end{dmath*}

	
	And this formula is valid. Applying resolution we get an equivalent valid formula:

	\[
		(
			\mathtt{x\prime =x}  \andcond \mathtt{x}\geq 0 \to \mathtt{x\prime }\geq 0
		)
	\]



\end{example}



%%%%%%%%%%%%%%%%%%%%%%%%%%%%%%%%%%%%%%%%%%%%%%%%%%%%%%%%%%%%%%%%%%%%%%%%%%%%%%%%%%%%
%%%%%%%%%%%%%%%%%%%%%%%%%%%%%%%%%%%%%%%%%%%%%%%%%%%%%%%%%%%%%%%%%%%%%%%%%%%%%%%%%%%%
%%%%%%%%%%%%%%%%%%%%%%%%%%%%%%%%%%%%%%%%%%%%%%%%%%%%%%%%%%%%%%%%%%%%%%%%%%%%%%%%%%%%
%%%%%%%%%%%%%%%%%%%%%%%%%%%%%%%%%%%%%%%%%%%%%%%%%%%%%%%%%%%%%%%%%%%%%%%%%%%%%%%%%%%%
%%%%%%%%%%%%%%%%%%%%%%%%%%%%%%%%%%%%%%%%%%%%%%%%%%%%%%%%%%%%%%%%%%%%%%%%%%%%%%%%%%%%
%%%%%%%%%%%%%%%%%%%%%%%%%%%%%%%%%%%%%%%%%%%%%%%%%%%%%%%%%%%%%%%%%%%%%%%%%%%%%%%%%%%%
%%%%%%%%%%%%%%%%%%%%%%%%%%%%%%%%%%%%%%%%%%%%%%%%%%%%%%%%%%%%%%%%%%%%%%%%%%%%%%%%%%%%
%%%%%%%%%%%%%%%%%%%%%%%%%%%%%%%%%%%%%%%%%%%%%%%%%%%%%%%%%%%%%%%%%%%%%%%%%%%%%%%%%%%%
%%%%%%%%%%%%%%%%%%%%%%%%%%%%%%%%%%%%%%%%%%%%%%%%%%%%%%%%%%%%%%%%%%%%%%%%%%%%%%%%%%%%
%%%%%%%%%%%%%%%%%%%%%%%%%%%%%%%%%%%%%%%%%%%%%%%%%%%%%%%%%%%%%%%%%%%%%%%%%%%%%%%%%%%%
%%%%%%%%%%%%%%%%%%%%%%%%%%%%%%%%%%%%%%%%%%%%%%%%%%%%%%%%%%%%%%%%%%%%%%%%%%%%%%%%%%%%
%%%%%%%%%%%%%%%%%%%%%%%%%%%%%%%%%%%%%%%%%%%%%%%%%%%%%%%%%%%%%%%%%%%%%%%%%%%%%%%%%%%%
%%%%%%%%%%%%%%%%%%%%%%%%%%%%%%%%%%%%%%%%%%%%%%%%%%%%%%%%%%%%%%%%%%%%%%%%%%%%%%%%%%%%
%%%%%%%%%%%%%%%%%%%%%%%%%%%%%%%%%%%%%%%%%%%%%%%%%%%%%%%%%%%%%%%%%%%%%%%%%%%%%%%%%%%%
%%%%%%%%%%%%%%%%%%%%%%%%%%%%%%%%%%%%%%%%%%%%%%%%%%%%%%%%%%%%%%%%%%%%%%%%%%%%%%%%%%%%
%%%%%%%%%%%%%%%%%%%%%%%%%%%%%%%%%%%%%%%%%%%%%%%%%%%%%%%%%%%%%%%%%%%%%%%%%%%%%%%%%%%%
%%%%%%%%%%%%%%%%%%%%%%%%%%%%%%%%%%%%%%%%%%%%%%%%%%%%%%%%%%%%%%%%%%%%%%%%%%%%%%%%%%%%
%%%%%%%%%%%%%%%%%%%%%%%%%%%%%%%%%%%%%%%%%%%%%%%%%%%%%%%%%%%%%%%%%%%%%%%%%%%%%%%%%%%%
%%%%%%%%%%%%%%%%%%%%%%%%%%%%%%%%%%%%%%%%%%%%%%%%%%%%%%%%%%%%%%%%%%%%%%%%%%%%%%%%%%%%
%%%%%%%%%%%%%%%%%%%%%%%%%%%%%%%%%%%%%%%%%%%%%%%%%%%%%%%%%%%%%%%%%%%%%%%%%%%%%%%%%%%%
%%%%%%%%%%%%%%%%%%%%%%%%%%%%%%%%%%%%%%%%%%%%%%%%%%%%%%%%%%%%%%%%%%%%%%%%%%%%%%%%%%%%
%%%%%%%%%%%%%%%%%%%%%%%%%%%%%%%%%%%%%%%%%%%%%%%%%%%%%%%%%%%%%%%%%%%%%%%%%%%%%%%%%%%%
%%%%%%%%%%%%%%%%%%%%%%%%%%%%%%%%%%%%%%%%%%%%%%%%%%%%%%%%%%%%%%%%%%%%%%%%%%%%%%%%%%%%
%%%%%%%%%%%%%%%%%%%%%%%%%%%%%%%%%%%%%%%%%%%%%%%%%%%%%%%%%%%%%%%%%%%%%%%%%%%%%%%%%%%%
%%%%%%%%%%%%%%%%%%%%%%%%%%%%%%%%%%%%%%%%%%%%%%%%%%%%%%%%%%%%%%%%%%%%%%%%%%%%%%%%%%%%
%%%%%%%%%%%%%%%%%%%%%%%%%%%%%%%%%%%%%%%%%%%%%%%%%%%%%%%%%%%%%%%%%%%%%%%%%%%%%%%%%%%%
%%%%%%%%%%%%%%%%%%%%%%%%%%%%%%%%%%%%%%%%%%%%%%%%%%%%%%%%%%%%%%%%%%%%%%%%%%%%%%%%%%%%
%%%%%%%%%%%%%%%%%%%%%%%%%%%%%%%%%%%%%%%%%%%%%%%%%%%%%%%%%%%%%%%%%%%%%%%%%%%%%%%%%%%%
%%%%%%%%%%%%%%%%%%%%%%%%%%%%%%%%%%%%%%%%%%%%%%%%%%%%%%%%%%%%%%%%%%%%%%%%%%%%%%%%%%%%
%%%%%%%%%%%%%%%%%%%%%%%%%%%%%%%%%%%%%%%%%%%%%%%%%%%%%%%%%%%%%%%%%%%%%%%%%%%%%%%%%%%%
%%%%%%%%%%%%%%%%%%%%%%%%%%%%%%%%%%%%%%%%%%%%%%%%%%%%%%%%%%%%%%%%%%%%%%%%%%%%%%%%%%%%
%%%%%%%%%%%%%%%%%%%%%%%%%%%%%%%%%%%%%%%%%%%%%%%%%%%%%%%%%%%%%%%%%%%%%%%%%%%%%%%%%%%%
%%%%%%%%%%%%%%%%%%%%%%%%%%%%%%%%%%%%%%%%%%%%%%%%%%%%%%%%%%%%%%%%%%%%%%%%%%%%%%%%%%%%
%%%%%%%%%%%%%%%%%%%%%%%%%%%%%%%%%%%%%%%%%%%%%%%%%%%%%%%%%%%%%%%%%%%%%%%%%%%%%%%%%%%%
%%%%%%%%%%%%%%%%%%%%%%%%%%%%%%%%%%%%%%%%%%%%%%%%%%%%%%%%%%%%%%%%%%%%%%%%%%%%%%%%%%%%
%%%%%%%%%%%%%%%%%%%%%%%%%%%%%%%%%%%%%%%%%%%%%%%%%%%%%%%%%%%%%%%%%%%%%%%%%%%%%%%%%%%%
%%%%%%%%%%%%%%%%%%%%%%%%%%%%%%%%%%%%%%%%%%%%%%%%%%%%%%%%%%%%%%%%%%%%%%%%%%%%%%%%%%%%
%%%%%%%%%%%%%%%%%%%%%%%%%%%%%%%%%%%%%%%%%%%%%%%%%%%%%%%%%%%%%%%%%%%%%%%%%%%%%%%%%%%%
%%%%%%%%%%%%%%%%%%%%%%%%%%%%%%%%%%%%%%%%%%%%%%%%%%%%%%%%%%%%%%%%%%%%%%%%%%%%%%%%%%%%
%%%%%%%%%%%%%%%%%%%%%%%%%%%%%%%%%%%%%%%%%%%%%%%%%%%%%%%%%%%%%%%%%%%%%%%%%%%%%%%%%%%%






\chapter{Code}
\label{app:code}

The code is shown with annotations and not with line numbers in order to better understand the invariants.
%
These annotations are used to express the invariants. 
%
The lines of code wrapped by ':tag$\left[\right.$\;'  and ':tag$\left.\right]$\;' are used to define the preconditions of the invariants. 
%
In stead of: \textit{(\pc(i) $\leq$ n $\wedge$ \pc(i) $\geq$ m  $\to$ ...)} ; it is written: \textit{(@tag $\to$ )}.
%
This way is clearer to read the invariants within the code.


We consider \head and \tail sentinel nodes which are neither removed nor 
modified and we assume that the list is initialized with \head and \tail 
already set.
%
The set \region is initialized containing solely the addresses of \head 
and \tail.
%
Similarly, the set \elements is initialized containing only the elements 
initially stored at the nodes pointed by \head and \tail.
%
There is also a function \concept{havocListElem}() which returns a random element. 
\lstinputlisting[mathescape,language=SPL,
   backgroundcolor=\color{lightgray!70!white},
   extendedchars=true,
   basicstyle=\footnotesize\ttfamily,
   showstringspaces=false,
   showspaces=false,
   tabsize=2,
   breaklines=true,
   showtabs=false,
   captionpos=b]{src/invs/annotation.tex}


\chapter{Invariants}


\label{appendix::inv:full}



\begin{center}\rule{4cm}{0.4pt}  \textbf{Preserve}  \rule{4cm}{0.4pt}\end{center}
\label{inv::full:preserve}
\small{}
\lstinputlisting[mathescape,language=leap,backgroundcolor=\color{gray!10!white}]{src/invs/preserve.inv}
\normalsize{}


\begin{center}\rule{4cm}{0.4pt}  \textbf{Disjoint}  \rule{4cm}{0.4pt}\end{center}
\label{inv::full:disjoint}
\small{}
\lstinputlisting[mathescape,language=leap,backgroundcolor=\color{gray!10!white}]{src/invs/disjoint.inv}
\normalsize{}


\begin{center}\rule{4cm}{0.4pt}  \textbf{Order}  \rule{4cm}{0.4pt}\end{center}
\label{inv::full:order}
\small{}
\lstinputlisting[mathescape,language=leap,backgroundcolor=\color{gray!10!white}]{src/invs/order.inv}
\normalsize{}


\begin{center}\rule{4cm}{0.4pt}  \textbf{Locks}  \rule{4cm}{0.4pt}\end{center}
\label{inv::full:lock}
\small{}
\lstinputlisting[mathescape,language=leap,backgroundcolor=\color{gray!10!white}]{src/invs/lock.inv}
\normalsize{}


\begin{center}\rule{4cm}{0.4pt}  \textbf{Region}  \rule{4cm}{0.4pt}\end{center}
\label{inv::full:region}
\small{}
\lstinputlisting[mathescape,language=leap,backgroundcolor=\color{gray!10!white}]{src/invs/region.inv}
\normalsize{}


\begin{center}\rule{4cm}{0.4pt}  \textbf{Next}  \rule{4cm}{0.4pt}\end{center}
\label{inv::full:next}
\small{}
\lstinputlisting[mathescape,language=leap,backgroundcolor=\color{gray!10!white}]{src/invs/next.inv}
\normalsize{}


