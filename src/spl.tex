% -*- root: ../main.tex -*-
\subsection{Formal representation of a program}

This \gls{SPL} and its formal representation is the language chosen to write the programs to be formally verified.
%
It has been chosen by \citep{thesisAle} because its simplicity and \doubt{expressiveness} in order to write concurrent programs. 
%
Because its simplicity it is a great option to do formal verification with it.


\subsubsection{Preliminaries (Notation, definition)}

A program is a series of state changes. 
%
There is the \pc variable, which has the information of the line to execute next (\concept{Program\IS counter}). 
%
Additionally, there are the steps of the program. Each step modifies the state of the program by modifying the values of the variables.
%
A step can be easily expressed in \gls{FOL} defining the \concept{post-state} variables, which are the new values the variables will have after the execution of the line.
%
The formula gathering the information of the execution of a line, using the \pc and \pc' (the post-state \pc) and all the other variables is called \concept[Transition relation]{transition relation}.


\subsubsection{Possible instructions}

In order to express correctly the transition relation corresponding to certain line, we need to know how to translate program lines into \gls{FOL} logic.
%
As we are going to work with programs used by more than one thread, we need one program counter for each thread executing.
%
We parametrize the program counter. 
%
This is, defining the \textbf{program counter} as a \textbf{function} that given a thread, returns its program counter. 
%
We could have define as one variable per thread.
%
It would be equivalent. 


We proceed to define in general terms how to build transition relation.
%
For this definitions we use the letter $T$ to refer a thread.
\begin{description}
\item [Assignments:]
		The transition relation for a variable assignment consists of the 
		update of the program counter for the running thread and the 
		corresponding modification to the variable being assigned.
%
			\[
			\begin{array}[t]{p{8em}@{\hspace{6em}}p{\longtablesize}}
				\hline
				Statement & Transition relation \\ \hline\hline
				$\begin{array}[t]{l@{\hspace{0.3em}}c@{\hspace{1em}}l}
					l_1 & : & \mathtt{v := 2} \\
					l_2 & : & \mathtt{\cdots}
				\end{array}$
				&
				$\begin{array}[t]{ll}
					 \pc(T) = l_1 \andcond
					 \pc'(T) = l_2 \andcond
					 \mathtt{v}' = 2
				 \end{array}$ \\ 
				 \hline
			\end{array}
			\]

	\item [Conditionals:]
		We present now the two possible kinds of conditional statements in 
		SPL.
%
		In the first case, if condition $c$ does not hold, the execution 
		proceeds from the statement following the end of the conditional.
%
		\[
        \begin{array}[t]{p{8em}@{\hspace{6em}}p{\longtablesize}}
				\hline
				Statement & Transition relation \\ \hline\hline
				$\begin{array}[t]{l@{\hspace{0.3em}}c@{\hspace{1em}}l}
					\ell_1 & : & \mathtt{\textbf{if } c \textbf{ then }} \\ \\
					\ell_2 & : & \mathtt{\cdots} \\
						&& \mathtt{\vdots} \\
					\ell_n & : & \mathtt{\textbf{end if}} \\
					\ell_{n+1} & : & \mathtt{\cdots}
				\end{array}$
				&
				$\begin{array}[t]{ll}
					(\pc(T) = \ell_1 \andcond \;\;\mathtt{c} \andcond \pc'(T) = \ell_2) \; \Or \\
					(\pc(T) = \ell_1 \andcond \lnot \mathtt{c} \andcond \pc'(T) = \ell_{n+1})
				 \end{array}$ \\ \hline
			 \end{array}
		 \]
%
		In the second case, if condition $c$ does not hold, the execution 
		continues at the first statement in the \textbf{else} section of the 
		conditional statement.
%
		\[
				\begin{array}[t]{p{8em}@{\hspace{6em}}p{\longtablesize}}
				\hline
				Statement & Transition relation \\ \hline\hline
				$\begin{array}[t]{l@{\hspace{0.3em}}c@{\hspace{1em}}l}
					\ell_1 & : & \mathtt{\textbf{if } c \textbf{ then }} \\ \\
					\ell_2 & : & \mathtt{\cdots} \\
					\mathtt{\vdots} \\
					\ell_n & : & \mathtt{\textbf{else}} \\
					\ell_{n+1} & : & \mathtt{\cdots} \\
					\mathtt{\vdots} \\
					\ell_m & : & \mathtt{\textbf{end if}} \\
					\ell_{m+1} & : & \mathtt{\cdots}
				\end{array}$
				&
				$\begin{array}[t]{ll}
					(\pc(T) = \ell_1 \andcond \;\;\mathtt{c} \andcond \pc'(T) = \ell_2) \; \Or \\
					(\pc(T) = \ell_1 \andcond \lnot \mathtt{c} \andcond \pc'(T) = \ell_{n+1})
						& \text{for line } \ell_1 \\ \\ \phantom{\vdots} \\

						\pc(T) = \ell_n \andcond \pc'(T) = \ell_{m+1} & \text{for line } \ell_n
				 \end{array}$ \\ \hline
			\end{array}
		\]
%
	\item [Loops:]
		We consider the only loop statement available in SPL which executes 
	the statements in the body as long as the loop condition holds.
%
		\[
				\begin{array}[t]{p{8em}@{\hspace{6em}}p{\longtablesize}}
				\hline
				Statement & Transition relation \\ \hline\hline
				$\begin{array}[t]{l@{\hspace{0.3em}}c@{\hspace{1em}}l}
					\ell_1 & : & \mathtt{\textbf{while } c \textbf{ do }} \\
					\ell_2 & : & \mathtt{\cdots} \\
					\mathtt{\vdots} \\
					\ell_n & : & \mathtt{\textbf{end while}} \\
					\ell_{n+1} & : & \mathtt{\cdots}
				\end{array}$
				&
				$\begin{array}[t]{ll}
						(\pc(T) = \ell_1 \andcond \;\;\texttt{c} \andcond \pc'(T) = \ell_2) \; 
						\Or \\
						(\pc(T) = \ell_1 \andcond \lnot \texttt{c} \andcond \pc'(T) = 
					\ell_{n+1})
					& \text{for line $\ell_1$} \\ \\
					\pc(T) = \ell_n \andcond \pc'(T) = \ell_1 &
						\text{for line $\ell_n$}
				 \end{array}$ \\ \hline
			 \end{array}
		\]
%
			\item [Non deterministic choice:]
		The transition relation for the non-deterministic choice statement can 
		be expressed as follows:
%
		\[
			\begin{array}[t]{p{8em}@{\hspace{6em}}p{\longtablesize}}
				\hline
				Statement & Transition relation \\ \hline\hline
				$\begin{array}[t]{l@{\hspace{0.3em}}c@{\hspace{1em}}l}
					\ell_1 & : & \mathtt{\Nondet} \\
					\ell_2 & : & \mathtt{\;\;\;\;\;\; \cdots} \\
					\ell_3 & : & \mathtt{\textbf{or } \; \cdots} \\
					\mathtt{\vdots} \\
					\ell_n & : & \mathtt{\textbf{or } \; \cdots} \\
					\ell_{n+1} & : & \mathtt{\NondetEnd} \\
				\end{array}$
				&
				$\begin{array}[t]{ll}
					\pc(T) = \ell_1 \andcond
					\bigvee\limits_{i = 2..n} \pc'(T) = \ell_i
				 \end{array}$ \\ \hline
			 \end{array}
		\]
%
	\item [Lock and unlock:]
	Even though these are not SPL statements, as they will be widely used, it is necessary to define theirs transition relations.
%
		\[
			\begin{array}[t]{p{8em}@{\hspace{6em}}p{\longtablesize}}
				\hline
				Statement & Transition relation \\ \hline\hline
				$\begin{array}[t]{l@{\hspace{0.3em}}c@{\hspace{1em}}l}
					\ell_1 & : & \mathtt{\fLock(l)} \\
					\ell_2 & : & \mathtt{\cdots}
				\end{array}$
				&
				$\begin{array}[t]{ll}
					\pc(T) = \ell_1 \andcond
						\mathtt{l} = \oslash \andcond
						\mathtt{l}' = T \andcond \pc'(T) = \ell_2
				 \end{array}$ \\ \hline\hline
				$\begin{array}[t]{l@{\hspace{0.3em}}c@{\hspace{1em}}l}
					\ell_1 & : & \mathtt{\fUnlock(l)} \\
					\ell_2 & : & \mathtt{\cdots}
				\end{array}$
				&
				$\begin{array}[t]{ll}
					\pc(T) = \ell_1 \andcond
						\mathtt{l}' = \oslash \andcond \pc'(T) = \ell_2
				 \end{array}$ \\ \hline
			 \end{array}
		\]
%
\end{description}
