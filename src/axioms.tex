
\subsubsection{Common}
	This are some common axioms, included in every \spass problem generated. 

	Table \ref{table:axioms_common} reports the uses of each axiom when proving each invariant.


	\begin{axiomdescription}[next-error--is--null]
		\label{ax::next_error__is__null}
		\explanation{ The field \fNext of the \fError \cell is the \fNull address.
		}
		\begin{formula}
			\forall \ast \;\; next(error) = null
		\end{formula}
	\end{axiomdescription}

	\begin{table}[hbtp]
		\centering
		\begin{tabular}{c|cccccc}
			axiom & disjoint & locks & nexts & order & preserve & region 
			\\\hline
			next-error--is--null & 0 & 1 & 194 & 142 & 0 & 8 
			\\
		\end{tabular}
		\label{table:axioms_common}
		\caption{Uses of the common subset of axioms.}
	\end{table}



\subsubsection{Arithmetic}
	
	This subset of the axioms refers to the arithmetical axioms needed at the proofs.
	
	Table \ref{table:axiom_arithmetic} reports the uses of each axiom when \spass proves each invariant.

	\begin{axiomdescription}[nums-are-different]
		\label{ax::nums_are_different}
		\explanation{ All numbers are different one another. \footnote{This axiom is necessary because \spass only includes the equality theory. Thus, 0,1,...,55 are 0-ary functions that could potentially be the same number.}
		}
		\begin{formula}
			\forall \ast \;\; (0 \neq 1) \andcond (0 \neq 2) \andcond ... \andcond (1\neq 2) \andcond ... (54 \neq 55)
		\end{formula}
	\end{axiomdescription}


	\begin{table}[hbtp]
		\centering
		\begin{tabular}{c|cccccc}
		axiom & disjoint & locks & nexts & order & preserve & region 
		\\\hline
				nums-are-different & 118 & 1 & 109 & 1 & 0 & 44 
				\\
		\end{tabular}
		\caption{Uses of the arithmetic subset of axioms.}
		\label{table:axiom_arithmetic}
	\end{table}


\subsubsection{Set theory}

	This set of axioms models some rudiments of set theory. As there are 3 types of sets in \TLLpL, this subset of axioms are replicated for $\Sigma_{\sSetAddr},\Sigma_{\sSetElem}$ and $\Sigma_{\sSetTid}$.
	
	Table \ref{table:axiom_set} reports the uses of each group of 3 axioms in each invariant.

	\begin{axiomdescription}[union-def]
		\label{ax::union_def}
		\explanation{ Let $x$ be a generic, and $se1,se2$ set of generics. Then $x$ belongs to the union of $se1,se2$ \gls{iff} $x$ is in $se1$ or $se2$.
		}
		\begin{formula}
			\forall \ast \;\; ((in(x,se) \vee in(x,se2)) \dimplies in(x,union(se,se2)))
		\end{formula}
	\end{axiomdescription}

		\begin{axiomdescription}[set-equal]
		\label{ax::set_equal}
		\explanation{
			Two set are equal \gls{iff} there not exists a generic that is in one set but not in the other.
			%
			This axiom is called in literature \cite{yellowbook} \concept{set extensionability}.
		}
		\begin{formula}
			\forall \ast \;\; [\nexists \;a\;  (in(a,se2) \dimplies in(a,se1))] \dimplies se1 = se2
		\end{formula}
	\end{axiomdescription}

	\begin{axiomdescription}[union-conmutative]
		\label{ax::union_conmutative}
		\explanation{ 
			The union of set is a commutative operation.
		}
		\begin{formula}
			\forall \ast \;\; union(se,se2) = union(se2,se)
		\end{formula}
	\end{axiomdescription}

	\begin{axiomdescription}[in-set--def]
		\label{ax::in_set__def}
		\explanation{ 
			Let $a,b$ be generics and $se$ a set of generics. If $a$ is not in the set but $b$ is in the set, then $a$ and $b$ can not be equal.
		}
		\begin{formula}
			\forall \ast \;\; ((\neg\;  in(a,se)) \implies in(b,se) \implies (\neg\;  b = a))
		\end{formula}
	\end{axiomdescription}

	\begin{axiomdescription}[a--in--singl-a]
		\label{ax::a__in__singl_a}
		\explanation{
			Let $se$ be the singleton set built from the generic element $a$.
			%
			Then, there are no other address different from $a$ in $se$.
			%
			We call $se = \fSingl(a)$.
		}
		\begin{formula}
			\forall \ast \;\; (((\neg\;  a = b) \implies (\neg\;  in(b,\fSingl(a)))) \wedge in(a,\fSingl(a)))
		\end{formula}
	\end{axiomdescription}

	\begin{axiomdescription}[emptySet-is-empty]
		\label{ax::emptySet_is_empty}
		\explanation{
			The generic empty set does not contain any generic.
		}
		\begin{formula}
			\forall \ast \;\; (\neg\;  in(a,empty))
		\end{formula}
	\end{axiomdescription}



	\begin{axiomdescription}[set-exten-inv]
		\label{ax::set_exten_inv}
		\explanation{
			The inverse of set extensionability, described above  (\ref{ax::set_equal}).
		}
		\begin{formula}
			\forall \ast \;\; \left((\forall a\;\; in(a,se1) \dimplies in(a,se2)) \implies se1 = se2\right)
		\end{formula}
	\end{axiomdescription}


	\begin{axiomdescription}[a-not--in-se-dif-a]
		\label{ax::a_not__in_se_dif_a}
		\explanation{
			A generic element is not in the difference of any set with $\fSingl(a)$
		}
		\begin{formula}
			\forall \ast \;\; in(a,se) \implies (\neg\;  in(a,\mathit{diff}(se,\fSingl(a))))
		\end{formula}
	\end{axiomdescription}

	\begin{table}[hbtp]
		\centering
		\begin{tabular}{c|cccccc}
		axiom & disjoint & locks & nexts & order & preserve & region 
		\\\hline
				union-def & 0 & 1 & 223 & 0 & 1 & 15 
				\\
%				set-equal & 0 & 0 & 0 & 0 & 0 & 0 
%				\\
				union-conmutative & 0 & 1 & 2 & 0 & 0 & 1 
				\\
%				in-set--def & 0 & 0 & 0 & 0 & 0 & 0 
%				\\
				a--in--singl-a & 0 & 1 & 778 & 16 & 0 & 71 
				\\
				emptySet-is-empty & 0 & 1 & 139 & 0 & 0 & 2 
				\\
				% EXTENSION:
				set-exten-inv & 0 & 1 & 136 & 0 & 0 & 0 
				\\
				a-not--in-se-dif-a & 0 & 1 & 47 & 0 & 0 & 12 
				\\
		\end{tabular}
		\label{table:axiom_set}
		\caption{Uses of the set subset of axioms.}
	\end{table}


\subsubsection{Memory}

	These set of axioms model $\Sigma_{\mem}$, i.e. the axioms related with the updates of a heap memory \mem.
	%
	Table \ref{table:axiom_upd} reports the uses of each axiom when proving each invariant.

	\begin{axiomdescription}[rd-mem--def]
		\label{ax::rd_mem__def}
		\explanation{
			The \cell returned when reading the position \fNull of any memory $m$ is the \fError \cell.
		}
		\begin{formula}
			\forall \ast \;\; rd(m,null) = error
		\end{formula}
	\end{axiomdescription}

	\begin{axiomdescription}[upd--def--not-null]
		\label{ax::upd__def__not_null}
		\explanation{(Update definition)
			\\
			Let $a$ be an \addr different from \fNull, $m$ a \mem and $c$ a \cell. We call $m2$ the result of the \fUpd statement.
			%
			Then, updating the value of $m$ stored at address $a$ with a \cell $c$ implies that $c$ is the value returned when reading in the resulting memory the stored at $a$.
		}
		\begin{formula}
			\forall \ast \;\; ((\neg\;  a = null) \implies upd(m,a,c) = m2 \implies rd(m2,a) = c)
		\end{formula}
	\end{axiomdescription}

	\begin{axiomdescription}[upd--def--one-at-the-time]
		\label{ax::upd__def__one_at_the_time}
		\explanation{
			Let $a$ be an \addr different from \fNull, $m$ a memory. We call $m2$ the result of the \fUpd statement.
			%
			An \fUpd statement that only modifies the cell at $a$, preserves all other values.
		}
		\begin{formula}
			\forall \ast \;\; (((\neg\;  a = null) \wedge (\neg\;  a = b)) \implies upd(m,a,c) = m2 \implies rd(m,b) = rd(m2,b))
		\end{formula}
	\end{axiomdescription}

	\begin{table}[hbtp]
			\centering
			\begin{tabular}{c|cccccc}
			axiom & disjoint & locks & nexts & order & preserve & region 
			\\\hline
					rd-mem--def & 0 & 1 & 194 & 1729 & 1 & 8 
					\\
					upd--def--not-null & 0 & 1 & 283 & 1729 & 17 & 6 
					\\
					upd--def--one-at-the-time & 0 & 1 & 451 & 1803 & 19 & 10 
					\\
			\end{tabular}
		\label{table:axiom_upd}
		\caption{Uses of the subset of update axioms.}
	\end{table}


\subsubsection{Element Arithmetic}

	This set of axiom models $\Sigma_{\elem}$, i.e., the axioms related with the arithmetic of the \elem.
	%
	Table \ref{table:axiom_elemarith} reports the uses of each axiom when proving each invariant.

	\begin{axiomdescription}[less-trans]
		\label{ax::less_trans}
		\explanation{
			Transitivity of the order relation on \mathsf{elems}.
		}
		\begin{formula}
			\forall \ast \;\; ((x < y \wedge y < z) \implies x < z)
		\end{formula}
	\end{axiomdescription}



	\begin{axiomdescription}[ls-xy--not-ls-yx]
		\label{ax::ls_xy__not_ls_yx}
		\explanation{
			The order relation \mathsf{elems} is total.
		}
		\begin{formula}
			\forall \ast \;\; (\fLselem(x,y) \dimplies ((\neg\;  x = y) \wedge (\neg\;  \fLselem(y,x))))
		\end{formula}
	\end{axiomdescription}

	\begin{axiomdescription}[lowest--less-than-highest]
		\label{ax::lowest__less_than_highest}
		\explanation{
			The \fLowest is less than the \fHighest (according to the order relation defined among elements).
		}
		\begin{formula}
			\forall \ast \;\; \fLselem(lowestElem,highestElem)
		\end{formula}
	\end{axiomdescription}

	\begin{axiomdescription}[lowestElem--def-tll]
		\label{ax::lowestElem__def_tll}
		\explanation{(Definition of \fLowest)\\
			Any other element is great or equal than the \fLowest.
		}
		\begin{formula}
			\forall \ast \;\; (e = lowestElem \vee lowestElem < e)
		\end{formula}
	\end{axiomdescription}

	\begin{axiomdescription}[highestElem--def-tll]
		\label{ax::highestElem__def_tll}
		\explanation{(Definition of \fHighest)\\
			Any other element is less or equal than the \fHighest.
		}
		\begin{formula}
			\forall \ast \;\; (e = highestElem \vee e < highestElem)
		\end{formula}
	\end{axiomdescription}



	\begin{table}
	        \centering
	        \begin{tabular}{c|cccccc}
	        axiom & disjoint & locks & nexts & order & preserve & region 
	        \\\hline
	                less-trans & 0 & 1 & 0 & 1271 & 0 & 0 
	                \\
	                ls-xy--not-ls-yx & 0 & 1 & 6 & 1513 & 0 & 0 
	                \\
	                lowest--less-than-highest & 0 & 1 & 0 & 1535 & 0 & 0 
	                \\
	                lowestElem--def-tll & 0 & 1 & 13 & 1164 & 0 & 0 
	                \\
	                highestElem--def-tll & 0 & 1 & 6 & 1554 & 0 & 0 
	                \\
	        \end{tabular}
	\label{table:axiom_elemarith}
	\caption{Uses of the subset of the element arithmetic axioms.}
	\end{table}


\subsubsection{Addr2set}

This subset of axioms and the following are specific axioms introduced this thesis.
%
These are  more complex than the previous axioms because they have the substance of the theories of liked list in the heap.

	\begin{axiomdescription}[nextreg]
		\label{ax::nextreg}
		\explanation{
			Let $a$ be an \addr reachable from another \addr $b$. 
			%
			If $a$ is different from \fNull, then the node pointed by $a$ is also reachable from $b$.
		}
		\begin{formula}
			\forall \ast \;\; ((in(a,se) \wedge se = addr2set(m,b) \wedge c = next(rd(m,a)) \wedge (\neg\;  a = null))\implies in(c,se))
		\end{formula}
	\end{axiomdescription}

	\begin{axiomdescription}[lock-keeps-addr2set]
		\label{ax::lock_keeps_addr2set}
		\explanation{
			The set of \addr reachable from an \addr $hd$ is preserved after a \fLock statement targeting another (or the same) \addr $a$. 
			%
			This is true because a \fLock statement does not change connectivity properties but only the thread using the \addr. 
			%
			Even if the \fLock statement targets \fError, the reachable \addr are preserved.
		}
		\begin{formula}
			\begin{array}{r}
			\forall \ast \;\; (hp_p = upd(hp,a,mkcell(data(rd(hp,a)),next(rd(hp,a)),t))) 
			\\
			\implies (addr2set(p,hd) = addr2set(hp_p,hd))
			\end{array}
		\end{formula}
	\end{axiomdescription}

	\begin{axiomdescription}[addr2set-null-is-singl-null]
		\label{ax::addr2set_null__is__singl_null}
		\explanation{
			The set of \addr reachable from \fNull is the \sSetAddr with just the \addr \fNull.
			%
			This is true as a consequence of previous axioms, because the \fNext of \fError is \fNull for every memory.
		}
		\begin{formula}
			\forall \ast \;\; addr2set(m,null) = \fSingl(null)
		\end{formula}
	\end{axiomdescription}

	\begin{table}
        \centering
        \begin{tabular}{c|cccccc}
        axiom & disjoint & locks & nexts & order & preserve & region 
        \\\hline
                nextreg & 0 & 1 & 112 & 0 & 1 & 68 
                \\
                lock-keeps-addr2set & 0 & 0 & 0 & 0 & 17 & 0 
                \\
%                addr2set-null-is-singl-null & 0 & 0 & 0 & 0 & 0 & 0 
 %               \\
        \end{tabular}
	\label{table:axiom_addr2set}
	\caption{Uses of the subset of axioms related with $\Sigma_{\path}$, i.e., the reachability of the \addr.}
	\end{table}


\subsubsection{Important}

	
	
	\begin{axiomdescription}[lock-keeps-heap--data \& lock-keeps-heap--next]
		\label{ax::lock_keeps_heap__data}\label{ax::lock_keeps_heap__next}
		\explanation{
			In the same way a \fLock statement preserves the \sSetAddr reachable from a given \addr, the same principle applies to some other functions apart from \fAddrToSet.
			%
			A \fLock statement does not change the \fData and \fNext values of a \cell.
		}
		\begin{formula}
		\begin{array}{r}
			\forall \ast \;\; (hp_p = upd(hp,a,mkcell(data(rd(hp,a)),next(rd(hp,a)),t)))
			\\
			\implies (data(r(hp,a)) = data(rd(hp_p,a)) \andcond next(r(hp,a)) = next(rd(hp_p,a)))
		\end{array}
		\end{formula}
	\end{axiomdescription}


	\begin{axiomdescription}[addr2set-rec-def]
		\label{ax::addr2set_rec_def}
		\explanation{
			Recursive definition of \fAddrToSet.
			%
			As the \fAddrToSet is used to get the reachable \addr from another \addr $a$, it is to expect that the \fAddrToSet of the \fNext of $a$ would be the same \sSetAddr excluding $a$.
			%
			This definition additionally states that an \addr is reachable from itself.
		}
		\begin{formula}
			\forall \ast \;\; addr2set(m,a) = union(\fSingl(a),addr2set(m,next(rd(m,a))))
		\end{formula}
	\end{axiomdescription}

	\begin{axiomdescription}[addr2set-primim]
		\label{ax::addr2set_primim}
		\explanation{
			This axiom is the base of the recursion in the definition of \fAddrToSet. 
		}
		\begin{formula}
			\forall \ast \;\; ((data(rd(hp,hd)) < data(rd(hp,tl)) \wedge next(rd(hp,hd)) = tl \wedge next(r(hp,tl)) = null) \implies addr2set(hp,hd) = union(union({ hd },{ tl }),\fSingl(\fNull)))
		\end{formula}
	\end{axiomdescription}

	\begin{axiomdescription}[not-in-region--not-change-heap-addr]
		\label{ax::not_in_region__not_change_heap_addr}
		\explanation{
			Let $hd$ be an \addr and $hp$ a \mem. 
			%
			Then, modifying a \mem in an \addr which is not reachable from $hd$ preserves the \sSetAddr of reachable \addr from $hd$.
		}
		\begin{formula}
			\forall \ast \;\; ((\neg\;  in(a,addr2set(hp,hd))) \implies addr2set(hp,hd) = addr2set(upd(hp,a,c),hd))
		\end{formula}
	\end{axiomdescription}



	\begin{axiomdescription}[insert--keeps-addr2set]
		\label{ax::insert__keeps_addr2set}
		\explanation{
			This axiom allows to \fUpd a \mem preserving the \fAddrToSet.
			%
			This \fUpd correspond to an insertion in the list.
			\\
			Let $hp$ be a \mem. Let $hd$ be an \addr.
			%
			We call \region the set of address reachable from $hd$ as captured by \fAddrToSet.
			%
			Let $prev$ be an \addr reachable from $hd$ (in the \fAddrToSet of $hp,hd$) and $curr,aux$ two other \addr. 
			%
			Let the three addresses ($prev,curr,aux$) be different from one another and all of them different from null.
			\\
			If prev points curr and aux points curr, then making prev point to aux has the following effect on \region: aux now is in \region and every reachable address before the \fUpd is still reachable.
		}
		\begin{formula}
			\begin{array}{l}
				\forall \ast \;\; (reg = addr2set(hp,hd) \wedge union(reg,\fSingl(aux) }) = reg_p \wedge \\
				next(rd(hp,prev)) = curr \wedge (\neg\;  prev = curr) \wedge next(rd(hp,aux)) = curr \wedge (\neg\;  aux = null)\wedge \\
				(\neg\;  prev = null) \wedge (\neg\;  curr = null) \wedge
				in(prev,addr2set(hp,hd)) \wedge 
				\\ hp_p = upd(hp,prev,mkcell(data(rd(hp,prev)),aux,rd(hp,prev).lockid))) \\
				\quad\quad\quad\implies reg_p = addr2set(hp_p,hd)
			\end{array}
		\end{formula}
	\end{axiomdescription}

%%%%%%%%%%%%%%%%%%%%%R
%%%%%%%%%%%%%%%%%%%%%R
%%%%%%%%%%%%%%%%%%%%%R
%%%%%%%%%%%%%%%%%%%%%R
%%%%%%%%%%%%%%%%%%%%%R
%%%%%%%%%%%%%%%%%%%%%R


	\begin{axiomdescription}[remove--keeps-addr2set]
		\label{ax::remove__keeps_addr2set}
		\explanation{
			In a similar way to (\ref{ax::insert__keeps_addr2set}), this axioms allows to \fUpd a \mem preserving the \fAddrToSet, but this time corresponding to a removal in the list.
			\\
			Let $hp$ be a \mem. Let $hd$ be an \addr.
			%
			We call \region the set of addresses reachable from $hd$.
			%
			Let $prev$ be reachable from $hd$ and $curr,aux$ two other \addr,
			%
			such that the three address ($prev,curr,aux$) are different from each other and all of them are not null.
			%
			If prev points to curr, aux points to curr, then making prev point to aux has the following effect in the \region: curr is now not reachable but every \addr reachable before the \fUpd is still reachable.
		}
		\begin{formula}
			\forall \ast \;\; ((next(rd(hp,curr)) = aux \wedge next(rd(hp,prev)) = curr \wedge (\neg\;  aux = next(rd(hp,prev))) \wedge hp_p = upd(hp,prev,mkcell(data(rd(hp,prev)),aux,rd(hp,prev).lockid)) \wedge (\neg\;  aux = null) \wedge in(curr,addr2set(hp,hd)) \wedge in(null,addr2set(hp,hd)) \wedge in(prev,addr2set(hp,hd))) \implies \mathit{diff}(addr2set(hp,hd),\fSingl(curr)) = addr2set(hp_p,hd))
		\end{formula}
	\end{axiomdescription}


	\begin{axiomdescription}[order-primim]
		\label{ax::order_primim}
		\explanation{
			This axiom is the base of the recursion in the definition of \fOrderlist. 
			%
			Let $hp$ be a \mem and $hd,tl$ be two \addr. 
			%
			If the \elem stored in $hp$ at $hd$ is less than the \elem stored in $hp$ at $tl$ and $hd$ point to $tl$, then the list from $hd$ to $tl$ is ordered.
		}
		\begin{formula}
			\forall \ast \;\; (ls\_elem(data(rd(hp,hd)),data(rd(hp,tl))) \andcond tl = next(rd(hp,hd)) \andcond next(rd(hp,tl)) = \fNull ) \implies orderlist(hp,hd,tl) 
		\end{formula}
	\end{axiomdescription}

	\begin{axiomdescription}[insert--keeps-orderlist]
		\label{ax::insert__keeps_orderlist}
		\explanation{
			This axioms allows to update a \mem preserving the \fOrderlist predicate. 
			%
			This update preserves the predicate if certain conditions are satisfied.
			\\
			Let $hp$ be a \mem and $hd,tl$ be two \addr which satisfy $\fOrderlist(hp,hd,tl)$; let $tl$ be different from \fNull, but pointing to \fNull. Let $prev,aux,curr$ be \addr with ordered elements, i.e. the \elem stored in $hp$ at $prev$ is less than the one at $aux$, which is less than the one at $curr$. In addition, $curr$ is pointed by $aux$ and $prev$.
			\textbf{Then}, updating $hp$ so that $prev$ points to $aux$ preserve \fOrderlist$(hp',hd,tl)$.
		}
		\begin{formula}
			\forall \ast \;\; ((orderlist(hp,hd,tl) \wedge (\neg\;  tl = null) \wedge next(rd(hp,tl)) = null \wedge data(rd(hp,prev)) < data(rd(hp,aux)) \wedge data(rd(hp,aux)) < data(rd(hp,curr)) \wedge next(rd(hp,aux)) = curr \wedge next(rd(hp,prev)) = curr \wedge hp_p = upd(hp,prev,mkcell(data(rd(hp,prev)),aux,rd(hp,prev).lockid))) \implies orderlist(hp_p,hd,tl))
		\end{formula}
	\end{axiomdescription}

	\begin{axiomdescription}[remove--keeps-orderlist]
		\label{ax::remove__keeps_orderlist}
		\explanation{
			In a similar way to (\ref{ax::insert__keeps_orderlist}), this axiom allows to update a \mem preserving the \fOrderlist predicate, but this update corresponds to a removal in the list. 
			%
			This update preserves the predicate if certain conditions are satisfied.
			\\
			Let $hp$ be a \mem and $hd,tl$ be two \addr which satisfy $\fOrderlist(hp,hd,tl)$ and that $tl$ points to \fNull. Let $prev,aux,curr$ satisfy that $aux$ is pointed by $curr$, which is pointed by $prev$. In addition, $curr,prev,aux$ must be reachable from $hd$.
			\textbf{Then}, updating $hp$ so that $hp$ stores at $prev$ the same cell stored at $aux$ is an action that preserves \fOrderlist for the new $hp$ from $hd$ to $tl$.	
		}
		\begin{formula}
			\forall \ast \;\; ((aux = next(rd(hp,curr)) \wedge curr = next(rd(hp,prev)) \wedge (\neg\;  aux = null) \wedge null = next(rd(hp,tl)) \wedge (\neg\;  aux = next(rd(hp,prev))) \wedge hp_p = upd(hp,prev,mkcell(data(rd(hp,prev)),aux,rd(hp,prev).lockid)) \wedge in(prev,addr2set(hp,hd)) \wedge in(curr,addr2set(hp,hd)) \wedge in(null,addr2set(hp,hd)) \wedge in(aux,addr2set(hp,hd)) \wedge orderlist(hp,hd,tl)) \implies orderlist(hp_p,hd,tl))
		\end{formula}
	\end{axiomdescription}
	

	\begin{axiomdescription}[next-is-not-same--if-ordered]
		\label{ax::next_is_not_same__if_ordered}
		\explanation{
			Let $hd,tl$ and $a$ be \addr reachable from $hd$ in the \mem $hp$, with $hp,tl$ and $a$ different from \fNull. 
			%
			In addition, let $tl$ point to \fNull.
			\textbf{Then} a can not point to itselt.
		}
		\begin{formula}
			\forall \ast \;\; ((in(a,addr2set(hp,hd)) \wedge in(tl,addr2set(hp,hd)) \wedge (\neg\;  hd = null) \wedge (\neg\;  tl = null) \wedge (\neg\;  a = null) \wedge next(rd(hp,tl)) = null) \implies (\neg\;  next(rd(hp,a)) = a))
		\end{formula}
	\end{axiomdescription}


\begin{table}
        \centering
        \begin{tabular}{c|cccccc}
        axiom & disjoint & locks & nexts & order & preserve & region 
        \\\hline
                lock-keeps-heap--data & 0 & 0 & 0 & 98 & 0 & 0 
                \\
                lock-keeps-heap--next & 0 & 1 & 2749 & 0 & 0 & 1 
                \\
                addr2set-rec-def & 0 & 1 & 63 & 0 & 1 & 4 
                \\
                addr2set-primin & 0 & 0 & 0 & 0 & 1 & 0 
                \\
                not-in-region--not-change-heap-addr & 0 & 1 & 7 & 16 & 2 & 0 
                \\
                insert--keeps-addr2set & 0 & 1 & 52 & 0 & 0 & 53 
                \\
                remove--keeps-addr2set & 0 & 1 & 0 & 0 & 0 & 6 
                \\
  	            order-primin & 0 & 0 & 0 & 0 & 1 & 0 
                \\
                insert--keeps-orderlist & 0 & 1 & 5 & 0 & 0 & 5
                \\
                remove--keeps-orderlist & 0 & 1 & 5 & 0 & 0 & 5 
                \\
%                next-injective-if-ordered & 0 & 0 & 0 & 0 & 0 & 0 
 %               \\
                next-is-not-same--if-ordered & 0 & 1 & 7 & 0 & 0 & 0 
                \\
%                just-tail--points-null & 0 & 0 & 0 & 0 & 0 & 0 
 %               \\
        \end{tabular}
\label{table:axiom_important}
\caption{Uses of the important subset of axioms}
\end{table}


	%\vskip 1cm